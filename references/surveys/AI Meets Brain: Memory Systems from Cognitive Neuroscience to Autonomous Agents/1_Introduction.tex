\section{Introduction}

Memory is the cognitive nexus that interweaves past experience with future decision-making~\cite{brain_intro_memories}. 
In humans, it manifests as a dynamic neural process through which the brain stores and manages information~\cite{cog_brain_system}. 
Its profound significance lies in endowing individuals with the capacity to learn, adapt, and reshape their behavior, enabling humans to maintain coherence and foresight in an ever-changing environment~\cite{brain_intro_evolutionary,brain_intro_adaptive}.
As Large Language Models (LLMs) continue to evolve, endowing AI systems with human-like memory capabilities has emerged as a critical challenge~\cite{summary_ltm}. 
However, the natively stateless nature~\cite{survey_stateless} of LLMs renders each inference independent, preventing models from maintaining cross-session continuity or accumulating experience from historical interactions.
While modern LLMs have scaled parameters and context windows to massive sizes, knowledge update costs~\cite{summary_cost} and computational complexity~\cite{method_lost_in_middle,intro_on2} remain significant bottlenecks. 

With the rapid advancement of agents across diverse domains and tasks~\cite{method_cellagent,method_branch_and_browse,method_coge,method_memorb,method_sweexp,method_liddia,method_magis,method_experepair,method_fkn,method_recmind,method_toolmem}, memory systems have emerged as a critical factor in enhancing their performance by enabling information persistence~\cite{method_rgmem,method_pre_storage,method_pisa} and long-horizon planning~\cite{method_legomem,method_reasoningbank}.
Rather than serving as a passive repository for historical interactions, memory has evolved into a dynamic cognitive hub that underpins complex decision-making. 
However, despite significant progress in memory mechanism research in recent years, existing works~\cite{method_mem0,dy_memory_amem,method_mem1,method_memgpt,method_bot} tend to remain confined to a single disciplinary perspective or lack depth in biological research, making it difficult to achieve deep integration between cognitive science and artificial intelligence. 
This separation has deprived both fields of opportunities for deep mutual validation and inspiration in memory research.

To bridge this gap, our survey provides a comprehensive and unified review of memory systems, integrating insights from cognitive neuroscience with the rapidly evolving field of LLM-driven agents. 
We first establish a progressive research perspective on memory, transitioning from human brain to LLMs and ultimately to agents, systematically elucidating its definition and fundamental role (\S\ref{sec:Definitions}, \S\ref{sec:Utility}).
Building on the classical short- and long-term memory dichotomy in cognitive neuroscience, we propose a taxonomy that classifies agent memory along two dimensions, including nature- and scope-based classification (\S\ref{sec:Categorization}).
The former distinguishes procedural experience from conceptual knowledge, while the latter concerns memory persistence within or across trajectories.
Next, we examine memory storage from the perspectives of location and format (\S\ref{sec:Storage}).
In cognitive neuroscience, short-term memory relies on distributed sensory-frontoparietal networks while long-term memory depends on hippocampal-neocortical coordination. 
For agents, storage locations include the context window for temporary memory and external memory bank for persistent information. 
Regarding format, the brain employs persistent activity and synaptic connection weights for short-term retention and structured forms like cognitive maps for long-term memory, while agents utilize natural language text, graph structures preserving relational information, internalized parameters, and latent representation in high-dimensional vector spaces.

This survey further analyzes memory management mechanisms in both the human brain and agents, covering the complete closed-loop lifecycle of memory extraction, updating, retrieval, and utilization (\S\ref{sec:Management}). 
In cognitive neuroscience, new information is encoded and gradually stabilized into durable representations through hippocampal-neocortical coordination. 
When external cues trigger hippocampal replay, the information about past events carried by these representations is reinstated, and the retrieval process itself opens a plasticity window during which the underlying memory traces can be updated, strengthened, or weakened.
In agents, raw information is distilled into structured records through flat, hierarchical, or generative paradigms, dynamically refreshed within trajectories while maintained across them. 
These records are retrieved via similarity matching or multi-factor approaches, then incorporated into reasoning through contextual augmentation or parameter internalization.
Then, we comprehensively outline various benchmarks for evaluating agent memory capabilities, categorizing them into semantic-oriented benchmarks that examine internal state maintenance and higher-order cognitive abilities, and episodic-oriented benchmarks that assess performance in vertical domains such as web search, tool use, and environmental interaction (\S\ref{sec:Benchmarks for Agent Memory}).
Furthermore, we address the often-overlooked yet critical issue of memory security, providing a systematic investigation from both attack and defense perspectives (\S\ref{sec:Security}). 
On the attack side, existing methods focus on extracting sensitive information, implanting backdoors through malicious data, or degrading agent judgment by introducing noise and conflicting signals. 
On the defense side, countermeasures have been developed to purify retrieval sources, block harmful responses in real time, and safeguard sensitive data throughout the memory lifecycle.
Finally, we propose future research directions with particular focus on two key areas (\S\ref{sec:Future}). 
The first is multimodal memory systems capable of processing and integrating information across text, image, audio, video modalities. 
The second is agent skills that enable memory sharing and transfer across heterogeneous agents, transforming domain expertise into composable, reusable, and portable modular resources.
We hope this survey can facilitate cross-disciplinary research, offering newcomers an accessible introduction while providing seasoned researchers with insights.
