%%
%% This is file `sample-manuscript.tex',
%% generated with the docstrip utility.
%%
%% The original source files were:
%%
%% samples.dtx  (with options: `all,proceedings,bibtex,manuscript')
%% 
%% IMPORTANT NOTICE:
%% 
%% For the copyright see the source file.
%% 
%% Any modified versions of this file must be renamed
%% with new filenames distinct from sample-manuscript.tex.
%% 
%% For distribution of the original source see the terms
%% for copying and modification in the file samples.dtx.
%% 
%% This generated file may be distributed as long as the
%% original source files, as listed above, are part of the
%% same distribution. (The sources need not necessarily be
%% in the same archive or directory.)
%%
%%
%% Commands for TeXCount
%TC:macro \cite [option:text,text]
%TC:macro \citep [option:text,text]
%TC:macro \citet [option:text,text]
%TC:envir table 0 1
%TC:envir table* 0 1
%TC:envir tabular [ignore] word
%TC:envir displaymath 0 word
%TC:envir math 0 word
%TC:envir comment 0 0
%%
%%
%% The first command in your LaTeX source must be the \documentclass
%% command.
%%
%% For submission and review of your manuscript please change the
%% command to \documentclass[manuscript, screen, review]{acmart}.
%%
%% When submitting camera ready or to TAPS, please change the command
%% to \documentclass[sigconf]{acmart} or whichever template is required
%% for your publication.
%%
%%

\RequirePackage{rotating}

% \documentclass[manuscript,screen,review]{acmart}
% \documentclass[manuscript,screen]{acmart}
\documentclass[manuscript,screen,authorversion,nonacm]{acmart}
% nonacm is for arxiv version.


% \PassOptionsToPackage{numbers, compress}{natbib}
%%
%% \BibTeX command to typeset BibTeX logo in the docs
\AtBeginDocument{%
  \providecommand\BibTeX{{%
    Bib\TeX}}}

%% Rights management information.  This information is sent to you
%% when you complete the rights form.  These commands have SAMPLE
%% values in them; it is your responsibility as an author to replace
%% the commands and values with those provided to you when you
%% complete the rights form.
\setcopyright{none}
\copyrightyear{2024}
\acmYear{2024}
\acmDOI{XXXXXXX.XXXXXXX}

%% These commands are for a PROCEEDINGS abstract or paper.
% \acmConference[CSUR]{ACM Computing Surveys}{}{New York, NY}
\acmJournal{CSUR}
%%
%%  Uncomment \acmBooktitle if the title of the proceedings is different
%%  from ``Proceedings of ...''!
%%
%%\acmBooktitle{Woodstock '18: ACM Symposium on Neural Gaze Detection,
%%  June 03--05, 2018, Woodstock, NY}
% \acmISBN{978-1-4503-XXXX-X/18/06}


%%
%% Submission ID.
%% Use this when submitting an article to a sponsored event. You'll
%% receive a unique submission ID from the organizers
%% of the event, and this ID should be used as the parameter to this command.
%%\acmSubmissionID{123-A56-BU3}

%%
%% For managing citations, it is recommended to use bibliography
%% files in BibTeX format.
%%
%% You can then either use BibTeX with the ACM-Reference-Format style,
%% or BibLaTeX with the acmnumeric or acmauthoryear sytles, that include
%% support for advanced citation of software artefact from the
%% biblatex-software package, also separately available on CTAN.
%%
%% Look at the sample-*-biblatex.tex files for templates showcasing
%% the biblatex styles.
%%

%%
%% The majority of ACM publications use numbered citations and
%% references.  The command \citestyle{authoryear} switches to the
%% "author year" style.
%%
%% If you are preparing content for an event
%% sponsored by ACM SIGGRAPH, you must use the "author year" style of
%% citations and references.
%% Uncommenting
%% the next command will enable that style.
%%\citestyle{acmauthoryear}




\usepackage{paralist}
\usepackage[utf8]{inputenc} % allow utf-8 input
\usepackage[T1]{fontenc}    % use 8-bit T1 fonts
\usepackage{hyperref}       % hyperlinks
% \usepackage[colorlinks=true,citecolor=brown,urlcolor=gray]{hyperref}
\usepackage{url}            % simple URL typesetting
\usepackage{wrapfig,lipsum,booktabs}       % professional-quality tables
\usepackage{amsfonts}       % blackboard math symbols
\usepackage{nicefrac}       % compact symbols for 1/2, etc.
\usepackage{microtype}      % microtypography
% \usepackage[dvipsnames]{xcolor}         % colors
\usepackage{colortbl}
% \usepackage{amssymb}
\usepackage{pifont}
\usepackage{rotating}
\usepackage{makecell}

\usepackage{bbm}

\usepackage{caption}
\usepackage{subcaption}
\usepackage{amsmath}
\usepackage{xspace}
\usepackage{multirow}

\usepackage{algorithm}
\usepackage{algpseudocode}

\usepackage[titletoc]{appendix}

\usepackage{pdflscape}





%%% 
\newcommand{\haizhou}[1]{\textcolor{cyan}{[Haizhou: #1]}}
\newcommand{\wenyuan}[1]{\textcolor{magenta}{[Wenyuan: #1]}}
\newcommand{\hao}[1]{\textcolor{violet}{[Hao: #1]}}
\newcommand{\zihao}[1]{\textcolor{blue}{[Zihao: #1]}}
\newcommand{\sayna}[1]{\textcolor{Mahogany}{[Sayna: #1]}}
\newcommand{\zifeng}[1]{\textcolor{Salmon}{[Zifeng: #1]}}

\newcommand{\todo}[1]{\textcolor{red}{[TODO: #1]}}
\newcommand{\tocite}{\textcolor{Tan}{[cite]}\xspace}
\newcommand{\toref}{\textcolor{OrangeRed}{[ref]}\xspace}
\newlength\savewidth\newcommand\shline{\noalign{\global\savewidth\arrayrulewidth
  \global\arrayrulewidth 1pt}\hline\noalign{\global\arrayrulewidth\savewidth}}
\newcommand{\tablestyle}[2]{\setlength{\tabcolsep}{#1}\renewcommand{\arraystretch}{#2}\centering\footnotesize}
\newcolumntype{C}{>{\centering\let\newline\\\arraybackslash\hspace{0pt}}m{2cm}}
% https://tex.stackexchange.com/a/12712/156344


%%% 
% \newcommand{\smaller}[1]{\fontsize{7pt}{1em}\selectfont{#1}}
\newcommand{\error}[1]{\epsilon_{\gD_{#1}}}
\newcommand{\erroremp}[1]{\hat{\epsilon}_{\gD_{#1}}}

\newcommand{\blah}{
{\textcolor{white}{blah blah blah blah blah blah blah blah blah blah blah blah blah blah blah blah blah blah blah blah blah blah}}
}
\newcommand{\BLAH}{
\blah\blah\blah\blah\blah\blah
}

\newcommand{\cmark}{\ding{51}}%
\newcommand{\xmark}{\ding{55}}%
\newcommand{\club}{\ding{168}}

\definecolor{gray1}{HTML}{f2f2f2}
\definecolor{gray2}{HTML}{e5e5e5}
\definecolor{gray3}{HTML}{cccccc}
\definecolor{cream}{HTML}{fef9d2}
\definecolor{green1}{HTML}{e2fee2}
\definecolor{blue1}{HTML}{d7e0f9}
% \definecolor{orange}{HTML}{FF7F00}
% \definecolor{gray3}{HTML}{b2b2b2}


% For floating text box
\usepackage[pscoord]{eso-pic}% The zero point of the coordinate systemis the lower left corner of the page (the default).

\newcommand{\placetextbox}[3]{% \placetextbox{<horizontal pos>}{<vertical pos>}{<stuff>}
  \setbox0=\hbox{#3}% Put <stuff> in a box
  \AddToShipoutPictureFG*{% Add <stuff> to current page foreground
    \put(\LenToUnit{#1\paperwidth},\LenToUnit{#2\paperheight}){\vtop{{\null}\makebox[0pt][c]{#3}}}%
  }%
}%


% \input{defs.tex}
\input{math.tex}


%%
%% end of the preamble, start of the body of the document source.
\begin{document}

%%
%% The "title" command has an optional parameter,
%% allowing the author to define a "short title" to be used in page headers.
\title{Continual Learning of Large Language Models: A Comprehensive Survey}

%%
%% The "author" command and its associated commands are used to define
%% the authors and their affiliations.
%% Of note is the shared affiliation of the first two authors, and the
%% "authornote" and "authornotemark" commands
%% used to denote shared contribution to the research.
\author{Haizhou Shi}
\authornote{Correspondence to: Haizhou Shi <haizhou.shi@rutgers.edu> and Hao Wang <hw488@cs.rutgers.edu>.}
\email{haizhou.shi@rutgers.edu}
\author{Zihao Xu}
% \email{zihao.xu@rutgers.edu}
\author{Hengyi Wang}
% \email{hengyi.wang@rutgers.edu}
\author{Weiyi Qin}
% \email{weiyi.qin@rutgers.edu}
% \author{Hao Wang}
% \authornotemark[1]
% \email{hw488@cs.rutgers.edu}
\author{Wenyuan Wang}
\authornote{Work done as visiting students at Rutgers Machine Learning Lab.}

\author{Yibin Wang}
\authornotemark[2]

\affiliation{%
    \institution{Rutgers University}
    % \city{Piscataway}
    % \state{New Jersey}
    \country{USA}
}


% \affiliation{%
%     \institution{Wuhan University}
%     % \city{Wuhan}
%     \country{China}
% }

% \affiliation{%
%     \institution{Huazhong University of Science and Technology}
%     % \city{Wuhan}
%     \country{China}
% }

\author{Zifeng Wang}
% \email{zifengw@google.com}
\author{Sayna Ebrahimi}
% \email{saynae@google.com}
\affiliation{%
    \institution{Google Cloud AI Research}
    \city{Mountain View}
    % \state{California}
    \country{USA}
}

\author{Hao Wang}
\authornotemark[1]
\email{hw488@cs.rutgers.edu}
\affiliation{%
    \institution{Rutgers University}
    % \city{Piscataway}
    % \state{New Jersey}
    \country{USA}
}

% \author{Huifen Chan}
% \affiliation{%
%   \institution{Tsinghua University}
%   \city{Haidian Qu}
%   \state{Beijing Shi}
%   \country{China}}

% \author{Charles Palmer}
% \affiliation{%
%   \institution{Palmer Research Laboratories}
%   \city{San Antonio}
%   \state{Texas}
%   \country{USA}}
% \email{cpalmer@prl.com}

% \author{John Smith}
% \affiliation{%
%   \institution{The Th{\o}rv{\"a}ld Group}
%   \city{Hekla}
%   \country{Iceland}}
% \email{jsmith@affiliation.org}

% \author{Julius P. Kumquat}
% \affiliation{%
%   \institution{The Kumquat Consortium}
%   \city{New York}
%   \country{USA}}
% \email{jpkumquat@consortium.net}

%%
%% By default, the full list of authors will be used in the page
%% headers. Often, this list is too long, and will overlap
%% other information printed in the page headers. This command allows
%% the author to define a more concise list
%% of authors' names for this purpose.
\renewcommand{\shortauthors}{Shi et al.}
\authorsaddresses{} % suppressing the footnote.

%%
%% The abstract is a short summary of the work to be presented in the
%% article.
\begin{abstract}
    The challenge of effectively and efficiently adapting statically pre-trained Large Language Models~(LLMs) to ever-evolving data distributions remains predominant. When tailored for specific needs, pre-trained LLMs often suffer from significant performance degradation in previous knowledge domains -- a phenomenon known as \emph{``catastrophic forgetting''}. While extensively studied in the Continual Learning~(CL) community, this problem presents new challenges in the context of LLMs. In this survey, we provide a comprehensive overview and detailed discussion of the current research progress on LLMs within the context of CL. Besides the introduction of the preliminary knowledge, this survey is structured into four main sections: we first describe an overview of continually learning LLMs, consisting of two directions of continuity: \emph{vertical continuity (or vertical continual learning)}, i.e., continual adaptation from general to specific capabilities, and \emph{horizontal continuity  (or horizontal continual learning)}, i.e., continual adaptation across time and domains~(\Secref{sec:overview}). Following vertical continuity, we summarize three stages of learning LLMs in the context of modern CL: Continual Pre-Training~(CPT), Domain-Adaptive Pre-training~(DAP), and Continual Fine-Tuning~(CFT)~(\Secref{sec:stages}). We then provide an overview of evaluation protocols for continual learning with LLMs, along with currently available data sources~(\Secref{sec:eval-and-data}). Finally, we discuss intriguing questions related to continual learning for LLMs~(\Secref{sec:discussion}). This survey sheds light on the relatively understudied domain of continually pre-training, adapting, and fine-tuning large language models, suggesting the necessity for greater attention from the community.  Key areas requiring immediate focus include the development of practical and accessible evaluation benchmarks, along with methodologies specifically designed to counter forgetting and enable knowledge transfer within the evolving landscape of LLM learning paradigms. The full list of papers examined in this survey is available at \href{https://github.com/Wang-ML-Lab/llm-continual-learning-survey}{https://github.com/Wang-ML-Lab/llm-continual-learning-survey}.
\end{abstract}

%%
%% The code below is generated by the tool at http://dl.acm.org/ccs.cfm.
%% Please copy and paste the code instead of the example below.
%%
% \begin{CCSXML}
% <ccs2012>
%  <concept>
%   <concept_id>00000000.0000000.0000000</concept_id>
%   <concept_desc>Do Not Use This Code, Generate the Correct Terms for Your Paper</concept_desc>
%   <concept_significance>500</concept_significance>
%  </concept>
%  <concept>
%   <concept_id>00000000.00000000.00000000</concept_id>
%   <concept_desc>Do Not Use This Code, Generate the Correct Terms for Your Paper</concept_desc>
%   <concept_significance>300</concept_significance>
%  </concept>
%  <concept>
%   <concept_id>00000000.00000000.00000000</concept_id>
%   <concept_desc>Do Not Use This Code, Generate the Correct Terms for Your Paper</concept_desc>
%   <concept_significance>100</concept_significance>
%  </concept>
%  <concept>
%   <concept_id>00000000.00000000.00000000</concept_id>
%   <concept_desc>Do Not Use This Code, Generate the Correct Terms for Your Paper</concept_desc>
%   <concept_significance>100</concept_significance>
%  </concept>
% </ccs2012>
% \end{CCSXML}

\begin{CCSXML}
<ccs2012>
   <concept>
       <concept_id>10010147.10010257.10010258.10010262.10010278</concept_id>
       <concept_desc>Computing methodologies~Lifelong machine learning</concept_desc>
       <concept_significance>500</concept_significance>
       </concept>
   <concept>
       <concept_id>10010147.10010178.10010179</concept_id>
       <concept_desc>Computing methodologies~Natural language processing</concept_desc>
       <concept_significance>500</concept_significance>
       </concept>
   <concept>
       <concept_id>10010147.10010257.10010293.10010294</concept_id>
       <concept_desc>Computing methodologies~Neural networks</concept_desc>
       <concept_significance>500</concept_significance>
       </concept>
 </ccs2012>
\end{CCSXML}

\ccsdesc[500]{Computing methodologies~Lifelong machine learning}
\ccsdesc[500]{Computing methodologies~Natural language processing}
\ccsdesc[500]{Computing methodologies~Neural networks}



% \ccsdesc[500]{Do Not Use This Code~Generate the Correct Terms for Your Paper}
% \ccsdesc[300]{Do Not Use This Code~Generate the Correct Terms for Your Paper}
% \ccsdesc{Do Not Use This Code~Generate the Correct Terms for Your Paper}
% \ccsdesc[100]{Do Not Use This Code~Generate the Correct Terms for Your Paper}

%%
%% Keywords. The author(s) should pick words that accurately describe
%% the work being presented. Separate the keywords with commas.
\keywords{Large Language Models, Continual Learning.}

% \received{20 February 2007}
% \received[revised]{12 March 2009}
% \received[accepted]{5 June 2009}

%%
%% This command processes the author and affiliation and title
%% information and builds the first part of the formatted document.
\maketitle


\section{Introduction}
\label{sec:intro}
% general background of large language models: 
    % 1. general succes of large language models;
    % 2. challenge of large language models adapting to some other domains
        % a. (horizontal continual learning) across semantic domains, temporal shift of data;
        % b. (vertical continual learning) different downstream tasks;
% continual learning in general solves the forgetting issue;
    % the overview of modern continual learning llms: horizontal + vertical.
    % special properties of CL in LLMs
    % drawing connection between traditional CL methods and LLMs, how the focus of them should be changed.
% organization of the rest of the paper.
Recent advances in large language models~(LLMs) have demonstrated considerable potential for achieving artificial general intelligence (AGI)~\cite{radford2019language,brown2020language,achiam2022chatgpt,achiam2023gpt,chowdhery2023palm,anil2023palm,touvron2023llama,touvron2023llama2}.
Researchers have observed that complex abilities such as multi-step reasoning, few-shot in-context learning, and instruction following improve as the scale of parameter size increases~\cite{wei2022chain,wei2022emergent,yao2024tree,wei2021finetuned,min2022rethinking}.
The development of LLMs is impactful and revolutionary, prompting machine learning practitioners to reconsider traditional computational paradigms for once-challenging human-level tasks.
However, LLMs are typically trained on static, pre-collected datasets encompassing general domains, leading to gradual performance degradation over time~\cite{loureiro2022timelms,jang2022towards,jin2022lifelong,jang2022temporalwiki,amba2021dynamic,dhingra2022time} and across different content domains~\cite{gupta2023continual,jin2022lifelong,ke2022continual-train,sun2020ernie,cossu2022continual,gururangan2022demix,qin2023recyclable,chen2023lifelong,qin2022elle}.
Additionally, a single pre-trained large model cannot meet every user need and requires further fine-tuning~\cite{weyssow2023usage,winata2023overcoming,zheng2023learn,winata2023overcoming,biderman2023pythia,zheng2023learn,bai2023enhancing,ke2021achieve,wei2022circle,qin2021lfpt5,chen2024parameterizing}.
While one potential solution is re-collecting pre-training data and re-training models with additional specific needs, this approach is prohibitively expensive and impractical in real-world scenarios. 

% general introduction of continual learning
To efficiently adapt LLMs to downstream tasks while minimizing performance degradation on previous knowledge domains, researchers employ the methodology of Continual Learning~(CL), also known as \emph{lifelong learning} or \emph{incremental learning}~\cite{pentina2016theoretical,chen2018lifelong,van2022three,wang2024comprehensive}. 
Inspired by the incremental learning pattern observed in human brains~\cite{mcclelland1995there,kandel2000principles,pallier2003brain,mccaffary2021towards}, CL trains machine learning models sequentially on a series of tasks with the expectation of maintaining performance across all tasks~\cite{kirkpatrick2017overcoming,li2017learning,ebrahimi2020adversarial,ebrahimi2019uncertainty}. Throughout training, models have limited or no access to previous data, posing a challenge in retaining past knowledge as optimization constraints from unseen previous data are absent during current-task learning~\cite{li2017learning,lomonaco2020rehearsalfree,shi2024unified}. This challenge, known as \emph{catastrophic forgetting}~\cite{mccloskey1989catastrophic}, has been a central focus in continual learning research since its inception.
Over the years, researchers have explored various techniques to mitigate forgetting. These include replay-based methods~\cite{chaudhry2019tiny,schwarz2018progress,shi2024unified}, parameter regularization~\cite{kirkpatrick2017overcoming,ritter2018online,aljundi2018memory}, and model architecture expansion~\cite{ramesh2021model,wang2022coscl}. Together, these techniques have significantly advanced the goal of achieving zero forgetting in continual learning across diverse tasks, model architectures, and learning paradigms.

% vertical and horizontal continual learning for large language models
In the context of training and adapting LLMs sequentially, the significance of CL is undergoing semantic shifts of its own as well. 
To highlight this ongoing shift, in this paper, we provide a comprehensive overview and detailed discussion of the current research progress on continual LLMs. 
For the general picture of continual LLMs, we for the first time divide it into two directions of continuity that need to be addressed by practitioners~(details in \Secref{sec:overview}):
\begin{itemize}
    \item \textbf{Vertical continuity (or vertical continual learning)}, which refers to the ongoing adaptation of LLMs as they transition from large-scale general domains to smaller-scale specific domains, involving shifts in learning objectives and entities of execution. 
    For example, healthcare institutions may develop LLMs tailored to the medical domain while retaining their general reasoning and question answering capabilities for users.
    \item \textbf{Horizontal continuity (or horizontal continual learning)}, which refers to continual adaptation across time and domains, often entails multiple training stages and increased vulnerability to forgetting. 
    For example, social media platforms continuously update LLMs to reflect recent trends, ensuring accurate targeting of downstream services like advertising and recommendations without compromised experience for existing users.
    % E.g., social media platforms continually updating LLMs for recent trends of topics, so that the downstream services such as advertisement and recommendation can be accurately targeted at the user needs, without compromising the experience of existing loyal users.
\end{itemize}
% Importantly, the explicit separation of vertical and horizontal CL extends beyond a trivial modification of existing CL paradigms, such as domain-incremental learning, which might be considered analogous to horizontal continuity; it offers a robust conceptual framework for analyzing and describing complex learning paradigms in continual LLMs. For example, Recyclable Tuning aims to preserve both vertical and horizontal continuity simultaneously~\cite{qin2023recyclable}, and future designs could include zigzag CL, which alternates between horizontal and vertical CL.
Importantly, separating vertical and horizontal CL transcends mere modification of existing paradigms, like domain-incremental learning, which aligns with horizontal continuity. This distinction offers a robust framework for analyzing complex CL paradigms in language models. For instance, Recyclable Tuning preserves both vertical and horizontal continuity simultaneously~\cite{qin2023recyclable}, and future designs might include zigzagging between horizontal and vertical CL.

In \Figref{fig:overview}, following \emph{vertical continuity}, we delineate three key stages of LLM learning within modern CL: Continual Pre-Training~(CPT), Domain-Adaptive Pre-training~(DAP), and Continual Fine-Tuning~(CFT)~(details in \Secref{sec:stages}). 
In CPT, existing research primarily investigates three types of distributional shifts: temporal, content-level, and language-level. Each presents distinct focuses and challenges.
In DAP, CL evaluation and techniques are frequently utilized. However, there is a noticeable lack of diversity in these techniques, considering the maturity of the conventional CL community.
In CFT, our focus is on the emerging field of learning LLMs, covering topics such as Continual Instruction Tuning (CIT), Continual Model Refinement (CMR), Continual Model Alignment (CMA), and Continual Multimodal LLMs (CMLLMs).
Next, we present a compilation of publicly available evaluation protocols and benchmarks~(details in \Secref{sec:eval-and-data}).
We conclude our survey with a discussion covering emergent properties of continual LLMs, changes in the roles of conventional CL types and memory constraints within the context of continual LLMs, and prospective research directions for this subject~(details in \Secref{sec:discussion}).

In summary, this survey provides a comprehensive review of existing continual learning studies for LLMs, which significantly distinguishes itself from existing literature on related topics~\cite{biesialska2020continual,ke2023continual,wang2024comprehensive,wu2024continual,yang2024recent}.
Our survey highlights the underexplored research area of continually developing LLMs, especially in the field of CPT and DAP. We emphasize the needs for increased attention from the community, including the development of practical, accessible, and widely acknowledged evaluation benchmarks. Additionally, methodologies need to be tailored to address forgetting in emerging LLM learning paradigms.
We hope this survey can provide a systematic and novel perspective of continual learning in the rapidly-changing field of LLMs and can help the continual learning community contribute to the challenging goals of developing LLMs in a more efficient, reliable, and sustainable manner~\cite{jang2022temporalwiki,su2023efficient,xie2023efficient,Cao2023InstructMol,attanasio2023worth}.


% \textbf{Organization.}\quad 
% The rest of this paper is organized as follows. We will first start by introducing the background and preliminaries of large language models and continual learning in \Secref{app:preliminary}. 
% Then we present the overview of continual learning in the modern era of large language models in \Secref{sec:overview}. Vertically, it can be roughly divided into three stages of continual training LLMs, and we will present a one-by-one survey of each stage in \Secref{sec:stages}.
% In \Secref{sec:cft}, the unique aspects of continual fine-tuning LLMs will be introduced, including continual instruction tuning~(\Secref{sec:cft-cit}), continual model refinement~(\Secref{sec:cft-cmr}), continual model alignment~(\Secref{sec:cft-cma}), and continual multimodal large language models~(\Secref{sec:cft-cmllm}).
% In \Secref{sec:eval-and-data}, we give an inclusive introduction to the evaluation protocols and benchmarks of continual learning for LLMs that are publicly available.
% Finally, in \Secref{sec:discussion}, we present a series of discussion of the role of continual learning in the era of large language models, including 
% emergent abilities in large-scale continual LLMs~(\Secref{sec:discussion-emergent}), 
% three types of continual learning~(\Secref{sec:discussion-xil}),
% roles of memory in continual learning of LLMs~(\Secref{sec:discussion-mem}), and prospective future directions~(\Secref{sec:discussion-future}). 

\section{Background and Related Work}
\label{sec:background}
% \textbf{Notation.}\quad
% We denote scalars with lowercase letters, vectors with lowercase boldface letters, and matrices with uppercase boldface letters.
% The $l_2$-norm of vectors and the Frobenius norm of a matrix are represented by $\|\cdot\|_2$. For a vector $\vv = [v_1, v_2, \cdots, v_n]^\top$, $\|\vv\|_2 = (\sum_{i=1}^{n} v_i^2)^{\nicefrac{1}{2}}$; for a matrix $\mA\in \mathbb{R}^{m\times n}$, $\|\mA\|_2 = (\sum_{ij} A_{ij}^2)^{\nicefrac{1}{2}}$.
% We use $\epsilon_{\gD}$, $\gL_{\gD}$ to denote the error function, and loss function that is deployed for training, respectively, where the subscript is used to denote the error/loss measured by taking the expectation on the data distribution $\gD$. 
% We further use $\hat{\gL}_{S}$ to represent the empirical evaluation of the loss function $\gL$ over the set of examples $S$.
% Probability and expectation are denoted by $P$ and $\E$, respectively. 
% We use $[m]$ to denote the set of positive integers up to $m$, $\{1, \cdots, m\}$.



\subsection{Large Language Models}
\label{sec:background-llm}
Primarily built on the transformer architecture, pre-trained language models~(PLMs) have established a universal hidden embedding space through extensive pre-training on large-scale unlabeled text corpora~\cite{devlin2018bert, liu2019roberta, raffel2020exploring}.
By scaling parameters to billions or even hundreds of billions and training on massive text datasets~\cite{kaplan2020scaling, hoffmann2022training}, PLMs not only demonstrate superior language understanding and generation capabilities but also manifest emergent abilities such as in-context learning, instruction following, and multi-step reasoning~\cite{wei2022chain,wei2022emergent,yao2024tree,wei2021finetuned,min2022rethinking}. 
These larger models are commonly referred to as Large Language Models~(LLMs). For more detailed introduction, please refer to \appref{app:preliminary-llm}.

\subsubsection{Pre-training of LLMs}
\label{sec:background-llm-pretraining}
There are two popular pre-training paradigms for LLMs. (1) \emph{Decoder-only models} typically employ auto-regressive language modeling (LM) tasks during pre-training, including the GPT family~\cite{radford2019language,brown2020language,achiam2022chatgpt,achiam2023gpt}, Gemini family~\cite{team2023gemini,reid2024gemini}, and the open-source Llama family~\cite{touvron2023llama,touvron2023llama2}.
Specifically, given a sequence of tokens $\vx = [x_1, x_2, \cdots, x_N ]$, LM predicts the next token $x_t$ autoregressively based on all preceding tokens $\vx_{<t} = [x_1, x_2, \cdots, x_{t-1}]$, and trains the entire network by minimizing the negative log-likelihood $-\sum^N_{t=1} \log P( x_t | \vx_{<t} )$,
% \begin{align}\label{lm}
%     \mathcal{L}_{{\rm LM}}(\vx) &\triangleq -\sum^N_{t=1} \log P( x_t | \vx_{<t} ), 
% \end{align}
where $P(x_1|\vx_{<1})\triangleq P(x_1)$ is the unconditional probability estimation of the first token.
(2) \emph{Encoder-only models}, e.g., BERT~\cite{devlin2018bert,liu2019roberta}, use masked language modeling (MLM) as a common pre-training objective. 
In MLM, for the input sequence $\vx$, a subset of input tokens $m(\vx)$ are masked and replaced with the special [MASK] token. The pre-training goal is to utilize the unmasked parts $\vx_{\backslash m(\vx)}$ to predict the masked portions $m(\vx)$. 
In summary, the overarching goal of MLM is to minimize the negative log-likelihood $-\sum_{\hat{x} \in m(\vx)}{\rm log} \, P( \hat{x}|\vx_{\backslash m(\vx)} )$.
% \begin{align}\label{mlm}
%     \mathcal{L}_{{\rm MLM}}(\vx) &\triangleq .
% \end{align}

\subsubsection{Adaptation of LLMs}
\label{sec:background-llm-adaptation}
LLMs are primarily trained to generate linguistically coherent text. However, this training may not align with human values, preferences, or practical needs. Furthermore, the pre-training data can be outdated, leading to knowledge cutoffs or inaccuracies. To address these issues, various computational paradigms such as Instruction Tuning~(IT)~\cite{zhang2024instruction}, Model Refinement~(MR)~\cite{de2021editing}, and Model Alignment~(MA)~\cite{ouyang2022rlhf,rafailov2024dpo} have been proposed. These approaches adapt LLMs to better meet diverse downstream tasks and user requirements.
% LLMs primarily focus on generating linguistically coherent text during pre-training; as a result, their performance may not align with the actual needs of human users or conform to human values, preferences, and principles. Additionally, due to potentially outdated pre-training data, LLMs may also encounter knowledge cutoff or fallacy issues. 
% Therefore, different computational paradigms are proposed, such as Instruction Tuning~(IT)~\cite{zhang2024instruction}, Model Refinement~(MR)~\cite{de2021editing}, and Model Alignment~(MA)~\cite{ouyang2022rlhf, rafailov2024dpo}, to help adapting the LLMs to various downstream tasks for various needs.

Numerous studies show that \textbf{Instruction Tuning~(IT)} can notably improve LLMs' ability to follow textual instructions~\cite{zhang2024instruction,wei2021finetuned,jiang2024instructiontuned,sanh2022multitask,ouyang2022rlhf}, leveraging the pre-existing knowledge within LLMs to bridge the gap between general and task-specific performance~\cite{wei2022finetuned}. Recent works like WizardLM~\cite{xu2023wizardlm} and CodecLM~\cite{wang2024codeclm} further tailor synthetic data to steer LLMs' behavior through IT.  Additionally, IT enhances the interaction between humans and LLMs, providing a more natural interface and aligning LLM outputs more closely with human expectations and preferences~\cite{luo2023empirical}.
%
LLMs make mistakes, such as inaccurate translations or outdated information~\cite{de2021editing}. 
% To keep LLMs updated with the evolving factual knowledge, 
Directly fine-tuning the model to correct these mistakes may disrupt its performance on previously learned tasks. To overcome these challenges, \textbf{Model Refinement~(MR)} is proposed to rectify the model's errors while preserving its performance on other inputs, with only moderate computing resources~\cite{sinitsin2020editable,de2021editing,fast_edit,hase2021language,huang2023transformer,mitchell2022memory,hartvigsen2023aging}.
%
\textbf{Model Alignment~(MA)} ensures AI systems' actions and outputs align with human values, ethics, and preferences~\cite{ouyang2022rlhf,rafailov2024dpo}.
% It can be defined as the process of adjusting the objectives and functioning of an AI system to achieve such goals, involving a combination of mathematical models, algorithmic adjustments, and iterative feedback to refine AI behavior.
MA can be broadly categorized into two types: Reinforcement Learning-based (RL-based) and Supervised Learning-based (SL-based). RL-based approaches~\cite{ouyang2022rlhf,schulman2017proximal} are trained to make decisions reinforced by human feedback, using a reward system to guide them towards desirable outcomes. In contrast, SL-based approaches~\cite{hendrycks2023aligning, rafailov2024dpo,ji2024ai} directly train models on datasets of human preferences, aligning their output with demonstrated human values. 


\subsection{Continual Learning}
\label{sec:background-cl}
Humans can accumulate knowledge and skills across tasks without significant performance decline on previous tasks \cite{mcclelland1995there,kandel2000principles,pallier2003brain,mccaffary2021towards}. In contrast, machine learning models, which are typically data-centric, often experience performance degradation on old tasks when trained on new ones, a phenomenon known as \emph{``catastrophic forgetting.''} The challenge of adapting models to a sequence of tasks without forgetting, especially when little to no past data can be preserved, is extensively studied in the continual learning community \cite{pentina2016theoretical,chen2018lifelong,van2022three,wang2024comprehensive}. For formal definitions, a detailed introduction to the three CL scenarios and techniques, please refer to \appref{app:preliminary-cl}.
% Humans accumulate knowledge and skills across tasks without significant performance decline on previous tasks~\cite{mcclelland1995there,kandel2000principles,pallier2003brain,mccaffary2021towards}. In contrast, machine learning models are usually data-centric, minimizing the training loss on the subsequent tasks will cause the model fail on the old ones, which phenomenon is phrased as \emph{``catastrophic forgetting''}. The problem of adapting models to a sequence of tasks without forgetting, under the constraint that no (or only a handful of) past data can be preserved, is extensively studied in the continual learning community~\cite{pentina2016theoretical,chen2018lifelong,van2022three,wang2024comprehensive}. 
% For formal definitions and detailed introduction of the three CL scenarios, techniques, and evaluation metrics, please refer to \appref{app:preliminary-cl}.

\subsubsection{Types of Continual Learning}
\label{sec:background-cl-types}
To lay the groundwork for subsequent discussions (as illustrated in \Tabref{tab:cft} and \Secref{sec:discussion-xil}), we follow the conceptual framework proposed by \cite{van2022three,kim2022theoretical,wang2024comprehensive}. There are three primary types of continual learning scenarios: (i) Task-Incremental Learning~(TIL), where task indices are available to the model during inference~\cite{li2017learning,kirkpatrick2017overcoming}; (ii) Domain-Incremental Learning~(DIL), where the model learns a sequence of tasks with the same formulation but without task indices during inference~\cite{shi2024unified}; and (iii) Class-Incremental Learning~(CIL), where the model learns new classes of data during training~\cite{rebuffi2017icarl,kim2022theoretical}.


\subsubsection{Techniques of Continual Learning}
\label{sec:background-cl-techs}
Existing CL techniques can be roughly categorized into five groups~\cite{wang2024comprehensive}: (i)~replay-based, (ii)~regularization-based, (iii)~architecture-based, (iv)~optimization-based, and (v)~representation-based.
Here, we provide a concise yet comprehensive introduction to the first three categories of continual learning techniques, as they are extensively applied in continual LLMs.

\textbf{Replay-based methods} adopt the relaxed memory constraint by keeping a small buffer of observed data and retraining the model on it when learning new tasks. 
Although replay-based methods may theoretically lead to loose generalization bounds~\cite{shi2024unified}, they are valued for their simplicity, stability, and high performance, even with a small episodic memory~\cite{chaudhry2019tiny,riemer2018learning,buzzega2020dark,rebuffi2017icarl}.
\textbf{Regularization-based methods} adopt a regularization term $\lambda \left\| \vtheta - \vtheta_{t-1}\right\|_\mSigma$ that penalizes large deviation from the history model in the parameter space,
where $\|\vv\|_\mSigma = \vv^\top \mSigma \vv$ is the vector norm evaluated on a positive-semi-definite matrix $\mSigma$, and $\lambda$ is the regularization coefficient, a hyper-parameter introduced to balance the past knowledge retention and current knowledge learning. 
The matrix $\mSigma$ introduced is to measure the different level of importance of each parameters and their correlations in retaining the past knowledge. 
In practice, to reduce computational overhead, diagonal matrices are often designed to encode only the importance of each parameter~\cite{kirkpatrick2017overcoming,aljundi2018memory,rongali2021continual}.
% 
\textbf{Architecture-based methods}, especially expanding the network architecture dynamically to assimilate new knowledge, is considered the most efficient form of CL~\cite{wang2022learning,wang2022dualprompt}. This method primarily tackles adaptation challenges and can achieve zero-forgetting when task IDs are available during inference or can be correctly inferred~\cite{gururangan2022demix,wistuba2023}. 
However, due to the difficulty of task ID inference, architecture expansion is predominantly utilized in TIL but is scarcely explored in DIL or CIL.
% Progressive Neural Networks~(PNN)~\cite{rusu2016progressive} proposes learning laterally connected neurons as new tasks arise, ensuring non-forgetting and enabling transfer of previously learned neurons for future tasks. 
In conjunction with pre-trained backbone large models like ViT~\cite{dosovitskiy2020image}, CoLoR~\cite{wistuba2023} trains various low-rank adaptation (LoRA)~\cite{hu2021lora} modules for different tasks. It estimates and stores prototypes for each task and utilizes the natural clustering ability of the pre-trained model during testing to infer task IDs, selecting the corresponding LoRA component for prediction generation.
In the domain of continual LLMs, architecture expansion has resurged in popularity following the rise of parameter-efficient fine-tuning (PEFT)~\cite{shazeer2017outrageously,hu2021lora,dettmers2023qlora}, a topic we will delve into shortly~\cite{yang2024moral,wang2023orthogonal,li2024examining,jang2022towards,jin2022lifelong,paul2024ircoder,yan2023af,wu2024llama}.

\subsubsection{Evaluation Metrics of Continual Learning}
\label{sec:background-cl-eval}
There are four evaluation protocols primarily designed for continual learning. \textbf{Overall Performance~(OP)}~\cite{ke2021achieve,zhang2022continual,zhang2023copf} calculates the average performance up until the current training stage, measuring the overall ability of a model balancing the performance of each task. As noted in \cite{shi2024unified}, OP corresponds to the primary optimization objective of continual learning, and hence receives the most attention.
\textbf{Forgetting~(F)} represents the largest performance drop observed of each task throughout the training process, averaged over all training stages. 
It quantifies the negative impact of learning new tasks brought to previously acquired knowledge. Ideally, a robust continual learning framework should achieve \textbf{Backward Transfer~(BWT)}, where learning new tasks enhances performance on prior tasks. 
BWT is measured by negating the forgetting, and hence a negative forgetting indicates a an improvement in performance on earlier tasks. 
\textbf{Forward Transfer (FWT)} measures the generalization ability of the continual learning algorithms to unseen tasks. It is defined as the difference between the current model's performance evaluated on the future tasks and the randomly initialized model. 
Refer to \appref{app:eval} for more details. 

% \subsection{Other Related Topics}
% \subsubsection{(Continual) Domain Adaptation}
% \subsubsection{Parameter-Efficient Fine-Tuning}
\section{Continual Learning Meets Large Language Models: An Overview}
\label{sec:overview}
% large language models are large in every aspect: model size, dataset size, large scale of computational resource, long development cycle, project team size. 
    % E.g., ...


Large language models (LLMs) are extensive in various dimensions, including the size of model parameters, pre-training datasets, computational resources, project teams, and development cycles~\cite{radford2019language,brown2020language,achiam2022chatgpt,achiam2023gpt,chowdhery2023palm,anil2023palm,touvron2023llama,touvron2023llama2}. The substantial scale of LLMs presents notable challenges for development teams, particularly in keeping them updated amidst rapid environmental changes~\cite{amba2021dynamic,jin2022lifelong,dhingra2022time,jang2022towards,jang2022temporalwiki}. To illustrate, in 2023, the average daily influx of new tweets exceeds 500 million\footnote{Source: \href{https://www.omnicoreagency.com/twitter-statistics}{https://www.omnicoreagency.com/twitter-statistics} }, and training on even a subset of this large volume of data is unaffordable. 
% Efficiently and reliably adapting LLMs becomes more critical when considering their cascading impact on downstream applications. Downstream users often lack expertise in collecting and storing large-scale data, maintaining large-scale hardware systems, and training LLMs themselves.
% recyclable tuning
% Recyclable Tuning~\cite{qin2023recyclable} is the pioneering study that explicitly outlines the overall supplier-consumer structure of the modern LLM production pipeline. 
Recyclable Tuning~\cite{qin2023recyclable} is the first work to explicitly outline the supplier-consumer structure in the modern LLM production pipeline.
% This structure allows us to dissect the challenges of continual LLMs from the perspectives of various roles involved.
% Under this pipeline, we can dissect challenges of continual LLMs into specific aspects, from various perspectives of different roles involved. 
On the supplier side, the model is continually pre-trained over a sequence of large-scale unlabeled datasets. After every release of the pre-trained model, the consumer utilizes the stronger and more up-to-date upstream model for downstream tasks. 
Compared to the upstream supplier, downstream users often lack capacity of collecting and storing large-scale data, maintaining large-scale hardware systems, and training LLMs themselves. 
% Therefore Recyclable Tuning mainly focuses on efficiently adapting an updated pre-trained LLM to downstream tasks continuously. 
% To enhance the efficiency of fine-tuning for downstream consumers, they initially make several key observations about continually pre-trained LLMs, focusing on mode connectivity and functional similarity. Additionally, they propose reusing the outdated fine-tuned components after a major update of the upstream pre-trained LLM. 
In this survey, we extend this framework and further present a comprehensive modern production pipeline encompassing various studies on continual LLM pre-training, adaptation, and deployment~(\Figref{fig:overview}). What sets our framework apart from existing studies~\cite{wu2024continual} is the \emph{incorporation of two directions of continuity: \textbf{Vertical Continuity} and \textbf{Horizontal Continuity}}.

% recyclable tuning
% Recyclable Tuning~\tocite is the first study that explicitly depicts this \emph{supplier-consumer} structure of LLM production pipeline, in which the authors assume the existence of both temporal and domain shifts for LLM pre-training and fine-tuning at the same time. To improve the efficiency of fine-tuning for downstream consumers, they first make several key observations about continually pre-trained LLMs, in terms of mode connectivity and functional similarity. They then attempt to reuse the staled fine-tuned components after a major update of the upstream pre-trained LLM. Following the conceptual framework proposed by Recyclable Tuning~\tocite, here we provide a comprehensive framework of a modern production pipeline that encompasses various studies about continually pre-training, adapting, and deploying LLMs, as shown in Figure~\toref. The key feature that differentiates our proposed framework from the existing studies~\tocite is that \emph{in this survey, the concept of continuity is modeled at two directions: \textbf{vertical continuity} and \textbf{horizontal continuity}}. 

% \haizhou{description of the vertical and horizontal continually learning of LLMs}

\subsection{Vertical Continuity (Vertical Continual Learning)}
\textbf{Definition.}\quad 
Vertical continuity (or vertical continual learning) has long been studied, either implicitly or explicitly, in existing literature.
% it involves a sequence of adaptation from \emph{general} to \emph{specific} domains and tasks. 
Vertical continuity is characterized by a hierarchical structure encompassing data inclusiveness, task scope, and computational resources. Specifically, the training task transitions gradually from general pre-training to downstream tasks, typically undertaken by distinct entities within the production pipeline~\cite{qin2023recyclable,gururangan2022demix,rongali2021continual,guo2023continuous,yan2023af,xie2023efficient}. 
\Figref{fig:overview} shows a typical pipeline for vertical continuity in LLMs, i.e., ``pre-training'' $\rightarrow$ ``domain-adaptive training'' $\rightarrow$ ``downstream fine-tuning''~\cite{luo2023biomedgpt,li2023cfgpt,deng2023learning,han2021econet,zhou2020pre,guo2023continuous,gururangan2020dont,colombo2024saullm7b,wu2023pmc,wu2024llama,yan2023af,rongali2021continual,ma2023ecomgptct,huang2023lawyer}: % pipeline illustrates this concept.
\begin{itemize}
\item \textbf{Pre-training.} During the \emph{pre-training} stage, a substantial amount of data from diverse domains is required to develop a general-purpose LLM. This phase demands a sizable research and development team dedicated to training and benchmarking the model, along with considerable computational resources. 
\item \textbf{Domain-Adaptive Pre-training.} Subsequently, downstream institutions may opt for \emph{domain-adaptive pre-training} to tailor the model for specific tasks using domain-specific data unavailable to the upstream supplier. 
\item \textbf{Finetuning.} Finally, the LLM undergoes \emph{fine-tuning} on annotated data for downstream tasks before deployment.
\end{itemize}



%%%%%%%%%%%%%%%%%%%%%%%%%%%%%%%%%%%%%%%%%%%%%%%%%
%%% Cites in the Overview Figure.
% \input{tables/cites_in_figure_left}
% all the citations to appear in the figure. 
% CPT text boxes
% \placetextbox{0.377}{0.818}
\placetextbox{0.44}{0.818}{\tiny{\cite{loureiro2022timelms,jang2022towards,jin2022lifelong,jang2022temporalwiki,amba2021dynamic,dhingra2022time,yildiz2024investigating}}} % Temporal Shift
\placetextbox{0.596}{0.818}{\tiny{\cite{gupta2023continual,ke2022continual-train,jin2022lifelong,sun2020ernie,cossu2022continual,gururangan2022demix,qin2023recyclable,chen2023lifelong,qin2022elle,ibrahim2024simple}}} % Content Shift
\placetextbox{0.73}{0.818}{\tiny{\cite{gogoulou2024continual,li2024examining,ibrahim2024simple}}} % Language

% DAP text boxes
\placetextbox{0.55}{0.7893}{\tiny{\cite{Lu2023BBTFin,Nguyen2023AstroLLaMA,xie2023efficient,li2023starcoder,guo2024deepseekcoder,Xue2023WeaverBird,paul2024ircoder,cheng2024adapting,dou2024sailor,fujii2024continual,takahashi2024pretraining}}} % DAP
\placetextbox{0.56}{0.7694}{\tiny{\cite{li2023cfgpt,deng2023learning,han2021econet,zhou2020pre,guo2023continuous,gururangan2020dont,yan2023af,rongali2021continual,ma2023ecomgptct}}} % DAP -> SFT
\placetextbox{0.76}{0.7694}{\tiny{\cite{luo2023biomedgpt}}} % DAP -> MM-SFT
\placetextbox{0.852}{0.7694}{\tiny{\cite{wu2024llama}}} % DAP -> U-SFT

\placetextbox{0.535}{0.7486}{\tiny{\cite{Bi2023OCEANGPT,wu2023pmc,Yang2023PLLaMa,li2024blade,xie2024me,nakamura2024aurora}}} % DAP -> IT
\placetextbox{0.68}{0.7486}{\tiny{\cite{Zheng2023MarineGPT}}} % DAP -> MM-IT
\placetextbox{0.77}{0.7486}{\tiny{\cite{colombo2024saullm7b}}} % DAP -> U-IT


\placetextbox{0.535}{0.7284}{\tiny{\cite{song2024code,lin2023geogalactica,rozière2024code,Chen2023HuatuoGPTII,Zhang2023xuanyuan,huang2023lawyer,thulke2024climategpt,acikgoz2024hippocrates}}} % Others
\placetextbox{0.51}{0.7077}{\tiny{\cite{nijkamp2022codegen}}} % DAP -> DAP
\placetextbox{0.63}{0.7077}{\tiny{\cite{ke2022continual-pre}}} % (DAP)_n

% CFT text boxes
\placetextbox{0.476}{0.661}{\tiny{\cite{scialom2022fine,wang2023trace,mok2023large,zhang2023citb,huang2024mitigating}}} % CIT
\placetextbox{0.476}{0.668}{\tiny{\cite{he2024dont,yin2022contintin,wang2023orthogonal,zhao2024sapt}}} % CIT
\placetextbox{0.575}{0.661}{\tiny{\cite{lin2022continual,hartvigsen2023aging,hu2024wilke,wang2024wise}}} % CMR
\placetextbox{0.575}{0.668}{\tiny{\cite{das2024larimar,yu2023melo,li2023continual}}} % CMR
\placetextbox{0.647}{0.661}{\tiny{\cite{zhang2023copf,zhangcppo}}} % CMA
\placetextbox{0.647}{0.668}{\tiny{\cite{lin2024mitigating}}} % CMA
\placetextbox{0.713}{0.661}{\tiny{\cite{he2023continual,zheng2024antiforgetting,chen2024coin}}} % CMLLMs
\placetextbox{0.713}{0.668}{\tiny{\cite{zhu2024model,zhao2024reconstruct}}} % CMLLMs
%%%%%%%%%%%%%%%%%%%%%%%%%%%%%%%%%%%%%%%%%%%%%%%%%



Throughout the process, the unlabeled domain-specific dataset is smaller in scale than the upstream pre-training phase but larger than the final downstream task fine-tuning phase. This pattern extends to computational resources, team size, and other factors. It is important to note that vertical continuity can involve more than three stages~\cite{nijkamp2022codegen,lin2023geogalactica,rozière2024code,huang2023lawyer}. In real-world applications, during domain-adaptive pre-training, additional ``layers'' can be added to accommodate multiple entities, such as various departments with distinct objectives but operating within the same domain.  



\begin{figure*}[t]
	\begin{center}
		\includegraphics[width=1\textwidth]{figures/overview.v6.pdf}
	\end{center}
	\caption{
        A high-level overview of the modern pipeline for continually pre-training and fine-tuning LLMs, where two dimensions of continuity are described. 
        \textbf{Vertical Continuity~(or Vertical Continual Learning):} LLM training can be vertically divided into three stages: (i)~Continual Pre-Training~(CPT), (ii)~Domain-Adaptive Pre-training~(DAP), and (iii)~Continual Fine-Tuning~(CFT). 
        % Along the vertical axis, scale of data, scope of tasks, and computational resources, gradually decreases, while the specificity of the LLM is improved towards the final downstream task's solution. 
        The main focus is the retention of the LLM's general knowledge~(prevention of vertical forgetting).
        \textbf{Horizontal Continuity~(or Horizontal Continual Learning):} After the LLMs are deployed, the models are continually updated when a new set of data becomes available. 
        The primary goal is to prevent horizontal forgetting in a long sequence of tasks.
    }
	\label{fig:overview}
    \vspace{-1em}
\end{figure*}

\textbf{Vertical Forgetting.}\quad
We term the performance degradation (in terms of general knowledge) 
% on general knowledge of a model undergoing 
due to 
vertical continual learning \emph{``vertical forgetting''}. As shown in \Figref{fig:continuity}, for vertical continual learning, the data distribution of upstream tasks partially covers the downstream, meaning the model might start off at a decent initialization for the subsequent stage of training. 
Two significant challenges must be addressed to prevent vertical forgetting: 
% There are two significant challenges to be addressed to prevent vertical forgetting: 
\begin{itemize}
    \item \textbf{Task Heterogeneity.} Stemming from the inherent disparity between the formulation of upstream tasks and downstream tasks, \emph{task heterogeneity} can lead to differences in model structures and training schemes, which has long been recognized as a major hurdle~\cite{rebuffi2017icarl,li2017learning,wu2019large,ni2021revisiting,kim2022theoretical}. To mitigate this issue, practitioners often employ methodologies such as freezing shared parameters during downstream phases or reformulating downstream tasks to match the structure of pre-training tasks~\cite{yang2024moral,wang2023orthogonal,li2024examining,paul2024ircoder,yan2023af,wu2024llama}.
    \item \textbf{Inaccessible Upstream Data.} This challenge arises primarily from varying levels of confidentiality across entities undertaking vertical continual learning. Data collected and curated under different protocols may not be accessible to some downstream entities. This scenario is even more challenging than the strict memory constraint presented in conventional CL~(\Defref{def:memory}),
    as algorithms for latter case rely on access to previous data at specific points for parameter importance measurement~\cite{kirkpatrick2017overcoming,aljundi2018memory} or for replay~\cite{riemer2018learning,chaudhry2019tiny,buzzega2020dark,shi2024unified}.
    To address the challenge of \emph{inaccessible upstream data}, existing methods either use public datasets or generate pseudo-examples to create proxy pre-training datasets~\cite{qin2021lfpt5}.
\end{itemize}


% % properties of continuity
% \textbf{Challenges.}\quad 
% In scenarios involving vertical continuity, two significant challenges must be addressed to alleviate forgetting and preserve knowledge: (i) \emph{task heterogeneity} and (ii) \emph{inaccessible upstream data}.
% The challenge of \emph{task heterogeneity} stems from the inherent disparity between the formulation of pre-training tasks and downstream tasks. This disparity can lead to differences in model structures, such as the addition of extra components onto pre-trained LLMs to address downstream tasks. Updating these added components without compromising previously acquired knowledge has long been recognized as a major hurdle, especially for class-incremental learning~\cite{rebuffi2017icarl,li2017learning,wu2019large,ni2021revisiting,kim2022theoretical}. To mitigate this issue, practitioners often employ methodologies like freezing shared parameters during downstream phases or reformulating downstream tasks to match the structure of pre-training tasks~\cite{yang2024moral,wang2023orthogonal,li2024examining,paul2024ircoder,yan2023af,wu2024llama}.

% The challenge of \emph{inaccessible upstream data} arises primarily from varying levels of confidentiality across entities. Data collected and curated under different protocols may not be accessible to downstream entities, including interns and external users, due to confidentiality concerns.
% As most machine learning models rely heavily on data, losing access to upstream data jeopardizes the constraints essential for knowledge preservation. Popular continual learning algorithms depend on previous data, either at specific points of or throughout the entire training process. 
% For example, replay-based methods need the old data and mix them into mini-batches during the next phase of training to prevent forgetting~\cite{riemer2018learning,chaudhry2019tiny,buzzega2020dark,shi2024unified}. 
% Although regularization-based methods do not store the previous examples, importance evaluation of each parameter requires is still undertaken on the old dataset~\cite{kirkpatrick2017overcoming,aljundi2018memory}.
% To address the challenge of inaccessible data, existing methods either use public datasets or generate pseudo-examples to create proxy pre-training dataset for replay or parameter regularization~\cite{qin2021lfpt5}.



% 
\subsection{Horizontal Continuity (Horizontal Continual Learning)}
\textbf{Definition.}\quad 
Horizontal continuity (or horizontal continual learning) refers to continual adaptation across time and domains, a topic extensively explored within the continual learning community. The primary rationale for preserving horizontal continuity lies in the dynamic nature of data distribution over time. To stay updated with these content shifts, an LLM must incrementally learn newly-emerged data. Otherwise, the cost of re-training will become prohibitively expensive and impractical~\cite{chaudhry2019efficient,amba2021dynamic,su2023efficient,xie2023efficient}.
Empirical evidence has consistently shown that despite their impressive capabilities, LLMs struggle to generalize effectively to future unseen data, particularly in the face of temporal or domain shifts~\cite{amba2021dynamic,jang2022towards,jang2022temporalwiki,dhingra2022time}. Additionally, they struggle to retain complete knowledge of past experiences when adapting to new temporal domains, although they do demonstrate a higher level of robustness against catastrophic forgetting~\cite{tao2022can,luo2023investigating,zheng2023learn,mehta2023empirical}.
The necessity of employing complex CL algorithms to address challenges in LLMs remains an open question. For instance, during large-scale continual pre-training, large institutions can typically afford the storage costs of retaining all historical data, rendering memory constraints meaningless. %not suitable. 
Several studies have demonstrated that with full access to historical data, simple sparse replay techniques can effectively mitigate forgetting~\cite{tao2022can,scialom2022fine,prabhu2023online,garg2024tic}. In contrast, numerous continual learning studies have showcased superior performance compared to naive solutions, suggesting the importance of continual learning techniques in LLM training~\cite{jang2022temporalwiki,jin2022lifelong,qin2022elle,chen2023lifelong}. 

\begin{figure*}[t]
	\begin{center}
		\includegraphics[width=0.9\textwidth]{figures/continuity.v0.pdf}
	\end{center}
	\caption{
        A diagram showing two different directions of continual learning of LLMs. \textbf{(a) Vertical Continual Learning of LLMs:} in this case, the upstream data distribution usually partially covers the subsequent tasks' data distribution. \textbf{(b) Horizontal Continual Learning of LLMs:} No constraints on the data distributions are present on horizontal continual learning. The continual LLMs need to handle the challenge of abrupt distributional shifts and a longer sequence of training.
    }
	\label{fig:continuity}
    \vspace{-1em}
\end{figure*}




\textbf{Horizontal Forgetting.}\quad 
We informally define \emph{``horizontal forgetting''} as the performance degradation on the previous tasks when model is undergoing horizontal continual learning. As illustrated in \Figref{fig:continuity}, horizontal continual learning typically involves training stages of similar scales, with potential distributional overlap among their data. In summary, two main challenges need to be addressed for horizontal continual learning of LLMs:
\begin{itemize}
    \item \textbf{Long Task Sequences.}
    Horizontal continual learning ideally involves numerous incremental phases, particularly to accommodate temporal shifts in data distribution. A \emph{longer task sequence} entails more update steps of the model, leading to inevitable forgetting of previously learned tasks. To address this challenge, researchers employ established continual learning techniques with stronger constraints, such as continual model ensemble~\cite{ramesh2021model}.
    \item \textbf{Abrupt Distributional Shift.} 
    In contrast to vertical continuity, where distributional shifts are often predictable, horizontal continual learning does not impose constraints on task properties. Evidence suggests that abrupt changes in task distributions can result in significant horizontal forgetting of the model~\cite{sarfraz2023error}.
\end{itemize}

% \textbf{Challenges.}\quad
% Unlike vertical continuity, which usually involves a small number of incremental phases, e.g., a typical three-stage training pipeline of \emph{pre-training $\rightarrow$ domain-adaptive pre-training $\rightarrow$ fine-tuning}, horizontal continuity can ideally contain as many incremental phases as possible, especially when we consider the temporal shifts of data. Continually learning on more incremental phases poses huge challenges for existing frameworks as (i) they are typically evaluated on limited number of incremental learning phases (ranging from 2 to 20), and their reliability is largely compromised for deploying LLMs in the real world; (ii) more incremental phases will make the whole training process closer to online continual learning, where the sampling efficiency of the algorithm can be a huge concern~\cite{mai2022online,prabhu2023online}. 
% Other than these challenges, most of the existing frameworks are designed for preserving horizontal continuity, and hence many algorithms can be readily adapted for LLMs. 


% \subsection{Interpreting LLM Topics through the Lens of Vertical and Horizontal Continuity}
% The categorization of vertical and horizontal continuity can provide us a new viewpoint for interpreting the special topics of LLMs, including instruction tuning~\tocite, model alignment~\tocite, and model editing~\tocite. 
% \todo{instruction tuning and model alignment: could be both vertical and horizontal; model editing: horizontal.}
% We will have a more detailed discussion about these three topics of LLMs in \toref. 

% \haizhou{do we need a simple example that can help the readers to better understand this general production pipeline? }

% Here we describe the joint pipeline of the continual large language models, including three significant stages: (i)~continual pre-training, (ii)~domain-adaptive pre-training, (iii)~continual fine-tuning. We will give an introduction to the categorization of the lifelong large language models based on the techniques adopted. 

% \section{Modern Stages of Learning Large Language Models Continually}
\section{Learning Stages of Continual Large Language Models}
\label{sec:stages}
% \haizhou{
% After the three subsections are done.
% \begin{itemize}
%     \item General Intro
%     \item Significant results, general observation that applies to all three subsections
%     \item Brief intro to the structure of this section.
% \end{itemize}
% }
% General Intro
% Significant results
% Structure of this Chapter
\Figref{fig:overview} provides an overview of continually learning LLMs. Along the axis of vertical continuity, three main ``layers'' of modern continual learning emerge. 
The top layer, Continual Pre-Training~(CPT), involves continuous pre-training of LLMs by the supplier on newly-collected data alongside existing data (\Secref{sec:cpt}). 
% As data volume increases, the general capacity of LLMs naturally evolves. 
The middle layer, Domain-Adaptive Pre-training~(DAP), prepares LLMs for domain-specific applications through additional pre-training on domain-specific \emph{unlabeled} data (\Secref{sec:dap}). 
The bottom layer, Continual Fine-Tuning~(CFT), targets models for final downstream tasks on the consumer side (\Secref{sec:cft}), where the model needs to be updated after deployment for the specified task. 
% Within continual fine-tuning, we further cover topics including continual instruction tuning~(\Secref{sec:cft-cit}), model refinement~(\Secref{sec:cft-cmr}), model alignment~(\Secref{sec:cft-cma}), and multimodal LLMs~(\Secref{sec:cft-cmllm}).



\subsection{Continual Pre-Training~(CPT)}
\label{sec:cpt}
% Significance of CPT: some works, some arguments, on-going research

% The recent development of large language models has shattered the glass ceiling in achieving close-to-human levels of natural language understanding and generation. However, effectively adapting these models to the ever-evolving environment remains a fundamental challenge. In \Tabref{tab:cpt-small}, we outline the basic properties of existing CPT papers.

% Two problems that need to be considered due to the large amount of pre-training data and large scale of the LLM model: effectiveness and efficiency.
\subsubsection{CPT: Effectiveness and Efficiency}
Before delving into the details of continual pre-training (CPT), it is important to address two fundamental questions: Firstly, regarding \emph{effectiveness}, can CPT enhance performance on downstream tasks beyond that of the initial training on a wide range of data domains? Extensive studies
% , including ELLE~\cite{qin2022elle}, DEMix~\cite{gururangan2022demix}, CKL~\cite{jang2022towards}, TemporalWiki~\cite{jang2022temporalwiki}, LLPT~\cite{jin2022lifelong}, and Lifelong-MoE~\cite{chen2023lifelong}, 
have not only demonstrated the necessity of CPT for improved downstream performance~\cite{qin2022elle,gururangan2022demix,jang2022towards,jang2022temporalwiki,jin2022lifelong,chen2023lifelong}, but also shown that when distributional shifts are gradual~\cite{jang2022temporalwiki,yildiz2024investigating} or somewhat correlated~\cite{gururangan2022demix}, CPT can effectively help model generalize to unseen data.
The second question is about \emph{efficiency}: given the large size of an LLM' parameters and data, both old and new, can we achieve adaptation and knowledge retention in a computationally efficient way? Concerning efficiency, most studies focus on techniques for efficient knowledge retention~\cite{jin2022lifelong,jang2022towards,jang2022temporalwiki,li2024examining}, which significantly overlap with the CL literature addressing catastrophic forgetting~\cite{schwarz2018progress,riemer2018learning,buzzega2020dark,shi2024unified,rebuffi2017icarl,ritter2018online,aljundi2018memory,rusu2016progressive,ramesh2021model,wang2022coscl}. 
% As mentioned before, these techniques replay~\cite{schwarz2018progress,riemer2018learning,buzzega2020dark,shi2024unified}, parameter regularization~\cite{rebuffi2017icarl,ritter2018online,aljundi2018memory}, and architecture expansion~\cite{rusu2016progressive,ramesh2021model,wang2022coscl}.
In contrast to prior approaches that fully utilize emergent data, some studies recognize the impracticality of this approach in real production environments. Instead, they concentrate on further improving the efficiency of adaptation. For instance, ELLE~\cite{qin2022elle} employs a function-preserved model expansion to facilitate efficient knowledge growth; \cite{amba2021dynamic} and \cite{xie2023efficient} sub-sample training data based on novelty and diversity to enhance training efficiency, achieving superior performance compared to full-data training. Though currently underexplored, efficient adaptation in continual pre-training is poised to become significant, given recent findings emphasizing data quality over quantity for LLM generalization~\cite{xie2024data,soldaini2024dolma}.



% Table of CPT.
% \begin{table*}[t]
% \caption{
%     \textbf{Summary of the existing studies on Horizontal Continual Pre-training of LLMs,} where the papers are organized based on their type, where: (i)~no continual learning techniques are studied, (ii)~continual learning techniques are studied as solely baselines, and (iii)~new approaches are proposed, containing some of the continual learning techniques. 
%     In the table, \emph{Dist.~Shift} denotes what type(s) of distributional shifts this particular study considers and is dedicated to solve.
%     In the section of \textbf{Continual Learning Tech.}, we mainly categorize three types of continual learning techniques that are studied in the paper: rehearsal~(\emph{Rehearsal}), parameter regularization~(\emph{Param.~Reg.}), and architecture expansion~(\emph{Arch. Exp.}). We use ``\cmark'', ``\xmark'', and ``\club'' to denote ``deployed in the proposed method'', ``not studied in the paper'', and ``studied as a baseline method'', respectively; and use ``\cmark$^*$'' to represent the vocabulary expansion and replacement. 
%     It is noteworthy that we do not include naive sequential fine-tuning in this table, as it is universally studied as the important baseline method in all of the papers in the table. 
%     The papers with only ``\club''~\cite{jin2022lifelong,jang2022temporalwiki,jang2022towards} means that merely existing CL techniques are studied in them, and the papers with only ``\xmark''~\cite{gupta2023continual,gogoulou2024continual} means that no CL techniques but special aspects of fine-tuning are studied, e.g., model (re)warming via learning rate scheduling~\cite{gupta2023continual}.
%     For a more detailed summary, please refer to \Tabref{tab:cpt-big}.
% }
% \begin{center}
% \resizebox{1\linewidth}{!}{%
% \begin{tabular}{ccccc cCcc}
% 	\toprule[0.12em]
% 	\multirow{2}{*}{\textbf{Method}} & 
%     \multicolumn{2}{c}{\textbf{Scenario}} & 
%     \multicolumn{3}{c}{\textbf{{Continual Learning Tech.}}} & 
%     \multirow{2}{*}{\textbf{{LLM Arch.}}} & 
%     \multicolumn{2}{c}{\textbf{{Evaluation}}} \\
%     \cmidrule(lr){2-3}\cmidrule(lr){4-6}\cmidrule(lr){8-9}
%     & \emph{Dist. Shift} & \emph{\#Domains} & \emph{Rehearsal} & \emph{Param. Reg.} & \emph{Arch. Exp.} & & \emph{Pre-Training} & \emph{Downstream} \\
% 	%  &
%     \cmidrule[0.12em]{1-9}
%     TimeLMs~\cite{loureiro2022timelms} & Temporal & 8 & \xmark & \xmark & \xmark & RoBERTa & \cmark & \cmark \\
%     \midrule
%     \cite{gupta2023continual}  & Content & 1 & \xmark & \xmark & \xmark & Pythia & \cmark & \xmark \\
%     \midrule
%     \cite{gogoulou2024continual} & Language & 3 & \xmark & \xmark & \xmark & GPT & \cmark & \xmark \\
%     \midrule
%     \midrule
%     \rowcolor{gray2}
%     \cite{li2024examining} & Language & 1 & \xmark & \club & \club & Llama2 & \cmark & \cmark \\
%     \midrule
%     \rowcolor{gray2}
%     CKL~\cite{jang2022towards} & Temporal & 1 & \club & \club & \club & T5 & \xmark & \cmark \\
%     \midrule
%     \rowcolor{gray2}
%     & Temporal  & 4 & & & & & & \\
%     \cline{2-3}
%     \rowcolor{gray2}
%     \multirow{-2}{*}{LLPT~\cite{jin2022lifelong}} & Content & 8 & \multirow{-2}{*}{\club} & \multirow{-2}{*}{\club} & \multirow{-2}{*}{\club} & \multirow{-2}{*}{RoBERTa} & \multirow{-2}{*}{\cmark} & \multirow{-2}{*}{\cmark} \\
%     \midrule
%     \rowcolor{gray2}
%     TemporalWiki~\cite{jang2022temporalwiki} & Temporal & 5 & \club & \club & \club & GPT-2 & \cmark & \cmark\\
%     \midrule
%     \midrule
%     \rowcolor{gray3}
%     CPT~\cite{ke2022continual-train} & Content & 4 & \club & \cmark\club & \club & RoBERTa & \cmark & \xmark \\
%     \midrule
%     \rowcolor{gray3}
%     ERNIE~2.0~\cite{sun2020ernie} & Content & 4 & \cmark\club & \xmark & \xmark & ERNIE & \xmark & \cmark \\
%     \midrule
%     \rowcolor{gray3}
% 	\cite{amba2021dynamic} & Temporal & 7 & \xmark & \cmark & \cmark$^*$ & BERT & \xmark & \cmark \\
%     \midrule
%     \rowcolor{gray3}
%     \cite{cossu2022continual} & Content & 5 & \xmark & \xmark & \cmark & BERT RoBERTa & \xmark & \cmark\\
%     \midrule
%     \rowcolor{gray3}
%     DEMix~\cite{gururangan2022demix} & Content & 8 & \xmark & \xmark & \cmark & GPT-3 & \cmark & \cmark \\
%     \midrule
%     \rowcolor{gray3}
%     TempoT5~\cite{dhingra2022time} & Temporal & 1 & \cmark & \xmark & \cmark$^*$ & T5 & \xmark & \cmark \\
%     \midrule
%     \rowcolor{gray3}
%     RecTuning~\cite{qin2023recyclable} & Content & 4 & \cmark & \xmark & \cmark & RoBERTa  & \xmark & \cmark\\
%     \midrule
%     \rowcolor{gray3}
%     Lifelong-MoE~\cite{chen2023lifelong} & Content & 3 & \cmark & \cmark & \cmark & GLaM & \cmark & \cmark\\
%     \midrule
%     \rowcolor{gray3}
%     ELLE~\cite{qin2022elle} & Content & 5 & \cmark\club & \cmark\club & \cmark\club & BERT\quad \quad GPT & \cmark & \cmark \\
%     \midrule
% 	\bottomrule[0.12em]
% 	\end{tabular}
% 	}
% \end{center}
% \label{tab:cpt-small}
% \end{table*}


% Detailed version
\begin{table*}[t]
\caption{
    \textbf{Summary of existing studies on Continual Pre-training of LLMs.} The papers are organized based on their relation to CL: (i)~no CL techniques are studied, (ii)~CL techniques are studied as solely baselines, and (iii)~new CL approaches are proposed. 
    In the table, \emph{Dist.~Shift} denotes what type(s) of distributional shifts this particular study considers and is dedicated to solve.
    In the section of \textbf{Continual Learning Tech.}, we mainly categorize three types of continual learning techniques that are studied in the paper: rehearsal~(\emph{Rehearsal}), parameter regularization~(\emph{Param.~Reg.}), and architecture expansion~(\emph{Arch. Exp.}). We use ``\cmark'', ``\xmark'', and ``\club'' to denote ``deployed in the proposed method'', ``not studied in the paper'', and ``studied as a baseline method'', respectively. 
    Note that we do not include naive sequential fine-tuning in this table, as it is universally studied as the important baseline method in all of the papers in the table. 
    The papers with only ``\club''~\cite{jin2022lifelong,jang2022temporalwiki,jang2022towards} means that 
    only existing CL techniques are studied, without proposing new ones, 
    % no new but only existing CL techniques are studied in them, 
    and the papers with only ``\xmark''~\cite{gupta2023continual,gogoulou2024continual} means that 
    special aspects of fine-tuning are studied, without using CL techniques. 
    % no CL techniques but special aspects of fine-tuning are studied.
    % For a more detailed summary, please refer to \Tabref{tab:cpt-big}.
}
\vspace{-0.5em}
\begin{center}
\resizebox{1\linewidth}{!}{%
\setlength{\tabcolsep}{9pt}
\begin{tabular}{cCCCC CCcc}
	\toprule[0.12em]
	\multirow{2}{*}{\textbf{Method}} & 
    \multicolumn{2}{c}{\textbf{Scenario}} & 
    \multicolumn{3}{c}{\textbf{{Continual Learning Tech.}}} & 
    \multirow{2}{*}{\textbf{{LLM Arch.}}} & 
    \multicolumn{2}{c}{\textbf{{Evaluation}}} \\
    \cmidrule(lr){2-3}\cmidrule(lr){4-6}\cmidrule(lr){8-9}
    & \emph{Dist. Shift} & \emph{\#Domains} & \emph{Rehearsal} & \emph{Param. Reg.} & \emph{Arch. Exp.} & & \emph{Pre-Training} & \emph{Downstream} \\
	%  &
    \cmidrule[0.12em]{1-9}
    
    TimeLMs~\cite{loureiro2022timelms} & Temporal & 8 & \xmark & \xmark & \xmark & RoBERTa & \cmark & \cmark \\
    \hline

    \cite{yildiz2024investigating} 
    & Content & 159 
    & \xmark 
    & \xmark 
    & \xmark
    & RoBERTa GPT-2
    & \cmark 
    & \xmark \\
    \hline 

    \cite{gupta2023continual}  & Content & 1 & \xmark & \xmark & \xmark & Pythia & \cmark & \xmark \\
    \hline

    
    \cite{gogoulou2024continual} & Language & 3 & \xmark & \xmark & \xmark & GPT & \cmark & \xmark \\
    \hline

   
    RHO-1~\cite{lin2024rho} 
    & Other
    & 1
    & \xmark 
    & \xmark 
    & \xmark
    & TinyLlama Mistral
    & \cmark & \cmark
    \\

    \midrule
    \midrule


    
    % \rowcolor{gray2}
    \cite{li2024examining} 
    & Language & 1 
    & \cellcolor{gray2}\xmark 
    & \cellcolor{gray2}P-Freeze$^\text{\club}$ 
    & \cellcolor{gray2}Adapter$^\text{\club}$ LoRA$^\text{\club}$ 
    & Llama2 
    & \cmark & \cmark \\
    \hline

    
    % \rowcolor{gray2}
    CKL~\cite{jang2022towards} 
    & Temporal & 1 
    & \cellcolor{gray2} Mix-Review$^\text{\club}$ 
    & \cellcolor{gray2} P-Freeze$^\text{\club}$ RecAdam$^\text{\club}$  
    & \cellcolor{gray2} LoRA$^\text{\club}$ K-Adapter$^\text{\club}$ 
    & T5 
    & \xmark & \cmark \\
    \hline


    % \rowcolor{gray2}
    LLPT~\cite{jin2022lifelong} 
    & \makecell{Temporal \\ \rule[0.5ex]{1\linewidth}{0.4pt} \\ Content} 
    & \makecell{4 \\ \rule[0.5ex]{1\linewidth}{0.4pt} \\ 8} 
    & \cellcolor{gray2} ER$^\text{\club}$ Logit-KD$^\text{\club}$ Rep-KD$^\text{\club}$ Contrast-KD$^\text{\club}$ SEED-KD$^\text{\club}$ 
    & \cellcolor{gray2} oEWC$^\text{\club}$ 
    & \cellcolor{gray2} Adapter$^\text{\club}$ Layer~Exp.$^\text{\club}$ 
    & RoBERTa 
    & \cmark & \cmark \\
    \hline


    
    % \rowcolor{gray2}
    TemporalWiki~\cite{jang2022temporalwiki} 
    & Temporal & 5 
    & \cellcolor{gray2} Mix-Review$^\text{\club}$ 
    & \cellcolor{gray2} P-Freeze$^\text{\club}$ RecAdam$^\text{\club}$  
    & \cellcolor{gray2} LoRA$^\text{\club}$ K-Adapter$^\text{\club}$ 
    & GPT-2 
    & \cmark & \cmark\\
    \midrule
    \midrule

    
    % \rowcolor{gray3}
    CPT$^*$~\cite{ke2022continual-train} 
    & Content & 4 
    & \cellcolor{gray3} DER++$^\text{\club}$\qquad KD$^\text{\club}$ 
    & \cellcolor{gray3}CPT$^\text{\cmark}$\qquad EWC$^\text{\club}$\qquad HAT$^\text{\club}$ 
    & \cellcolor{gray3}Adapter$^\text{\club}$ DEMix$^\text{\club}$ 
    & RoBERTa 
    & \cmark & \xmark \\
    \hline

    
    % \rowcolor{gray3}
    ERNIE~2.0~\cite{sun2020ernie} 
    & Content & 4 
    & \cellcolor{gray3} ER$^\text{\cmark\club}$ 
    & \cellcolor{gray3} \xmark 
    & \cellcolor{gray3} \xmark 
    & ERNIE 
    & \xmark & \cmark \\
    \hline

    
    % \rowcolor{gray3}
	\cite{amba2021dynamic} 
    & Temporal & 7
    & \cellcolor{gray3} \xmark 
    & \cellcolor{gray3} P-Freeze$^\text{\cmark}$ 
    & \cellcolor{gray3} Vocab. Exp.$^\text{\cmark}$ 
    & BERT 
    & \xmark & \cmark \\
    \hline

    
    % \rowcolor{gray3}
    \cite{cossu2022continual} 
    & Content & 5 
    & \cellcolor{gray3} \xmark 
    & \cellcolor{gray3} \xmark 
    & \cellcolor{gray3} Vocab. Exp.$^\text{\cmark}$ 
    & BERT\qquad RoBERTa 
    & \xmark & \cmark \\
    \hline

    
    % \rowcolor{gray3}
    DEMix~\cite{gururangan2022demix} 
    & Content & 8 
    & \cellcolor{gray3} \xmark 
    & \cellcolor{gray3} \xmark 
    & \cellcolor{gray3} MoE$^\text{\cmark}$ 
    & GPT-3 
    & \cmark & \cmark \\
    \hline

    
    % \rowcolor{gray3}
    TempoT5~\cite{dhingra2022time} 
    & Temporal & 1 
    & \cellcolor{gray3} \xmark 
    & \cellcolor{gray3} \xmark 
    & \cellcolor{gray3} Vocab. Exp.$^\text{\cmark}$ Prompt$^\text{\cmark}$ 
    & T5 
    & \xmark & \cmark \\
    \hline

    
    % \rowcolor{gray3}
    RecTuning~\cite{qin2023recyclable} 
    & Content & 4 
    & \cellcolor{gray3} ER$^\text{\cmark}$\qquad \qquad KD$^\text{\cmark}$ 
    & \cellcolor{gray3} \xmark 
    & \cellcolor{gray3} Adapter$^\text{\cmark}$ 
    & RoBERTa  
    & \xmark & \cmark\\
    \hline

    
    % \rowcolor{gray3}
    Lifelong-MoE~\cite{chen2023lifelong} 
    & Content & 3 
    & \cellcolor{gray3} ER$^\text{\club}$\qquad \qquad KD$^\text{\cmark}$ 
    & \cellcolor{gray3} P-Freeze$^\text{\cmark}$\qquad L2$^\text{\club}$ 
    & \cellcolor{gray3} MoE$^\text{\cmark}$ 
    & GLaM 
    & \cmark & \cmark\\
    \hline

    
    % \rowcolor{gray3}
    ELLE~\cite{qin2022elle} 
    & Content & 5 
    & \cellcolor{gray3} ER$^\text{\cmark\club}$\qquad KD$^\text{\club}$ 
    & \cellcolor{gray3} P-Freeze$^\text{\cmark}$ 
    & \cellcolor{gray3} Prompt$^\text{\cmark}$\qquad Layer~Exp.$^\text{\cmark}$ Adapter$^\text{\club}$ 
    & BERT\qquad \qquad GPT 
    & \cmark & \cmark \\
    \hline


    \cite{ibrahim2024simple} 
    & Content Language & 2
    & \cellcolor{gray3} ER$^\text{\cmark}$ 
    & \cellcolor{gray3} \xmark
    & \cellcolor{gray3} \xmark
    & GPT-NeoX
    & \cmark & \cmark\\
    \hline


    CEM~\cite{zhao2024large} 
    & Other
    & 1 
    & \cellcolor{gray3} ER$^\text{\cmark}$
    & \cellcolor{gray3} \xmark
    & \cellcolor{gray3} \xmark
    & CuteGPT ChatGLM Qwen-Chat
    & \xmark
    & \cmark
    \\
    \hline
    
    IR-DRO~\cite{chen2024take}
    & Other
    & 1
    & \cellcolor{gray3} ER$^\text{\cmark}$
    & \cellcolor{gray3} \xmark
    & \cellcolor{gray3} \xmark
    & OPT
    & \xmark
    & \cmark 
    \\
    
	\bottomrule[0.12em]
	\end{tabular}
	}
\end{center}
\vspace{-1.5em}
\label{tab:cpt-small}
\end{table*}

% General trend shown in the table
\subsubsection{General Observations on CPT}
\Tabref{tab:cpt-small} summarizes the existing studies on continual pre-training~(CPT), and here are some key observations we make about CPT.
\begin{itemize}
    \item \textbf{OBS-1: The development of advanced techniques tailored specifically for CPT is at the starting stage and warrants further exploration.} 
    Only about half of the examined papers propose novel techniques for CPT~\cite{ke2022continual-train,sun2020ernie,amba2021dynamic,cossu2022continual,gururangan2022demix,dhingra2022time,qin2023recyclable,chen2023lifelong,qin2022elle}, while the remaining half either focus solely on the effects of pure adaptation without considering CL techniques~\cite{loureiro2022timelms,gupta2023continual,gogoulou2024continual}, or conduct empirical studies on the straightforward application of existing CL techniques~\cite{li2024examining,jin2022lifelong,jang2022towards,jang2022temporalwiki}.
    \item \textbf{OBS-2: The diversity of CL techniques incorporated in CPT remains limited.} 
    Most practical implementations of CL techniques for CPT primarily focus on architecture expansion of LLMs~\cite{amba2021dynamic,cossu2022continual,gururangan2022demix,dhingra2022time,qin2023recyclable,chen2023lifelong}, with only a few explicitly utilizing replay~\cite{qin2023recyclable,chen2023lifelong} and parameter regularization~\cite{amba2021dynamic,chen2023lifelong}.
    \item \textbf{OBS-3: There is an apparent gap between the existing studies and the real production environment of CPT.} 
    Except for the recent study~\cite{yildiz2024investigating} which conducts CPT over 159 domains, the longest sequence of pre-training stages explored is 8~\cite{jin2022lifelong,gururangan2022demix}. However, this falls short of real-world scenarios where continual pre-training occurs more frequently and persists for months or years. The efficacy of CPT methods in such prolonged scenarios remains uncertain. Additionally, investigating CPT in a task-boundary-free data stream setting is an important avenue for research to be explored in the future as well.
\end{itemize}

% \haizhou{DELETE: 
% % The analysis presented in \Tabref{tab:cpt-small} sheds light on the prevailing research trends in continual pre-training (CPT). 
% Firstly, it is evident that \emph{\textbf{the development of advanced techniques tailored specifically for CPT is still at the starting stage and warrants further exploration.}} This observation is underscored by the fact that only about half of the examined papers propose novel techniques (9 out of 16 papers, represented in the deep gray section of \Tabref{tab:cpt-small}), while the remaining half either focus solely on the effects of pure adaptation without considering continual learning techniques (3 out of 16 papers, represented in the white section), or conduct empirical studies on the straightforward application of existing continual learning techniques (4 out of 16 papers, represented in the light gray section).
% Secondly, while research extensively covers various continual learning techniques, such as rehearsal, parameter regularization, and architecture expansion (as indicated in the light gray section of \Tabref{tab:cpt-small}), \textbf{\emph{the practical incorporation of these techniques in systems remains relatively limited}}. Most practical implementations primarily focus on architecture expansion of LLMs~\cite{amba2021dynamic,cossu2022continual,gururangan2022demix,dhingra2022time,qin2023recyclable,chen2023lifelong}, with only a few explicitly utilizing replay~\cite{qin2023recyclable,chen2023lifelong} and parameter regularization~\cite{amba2021dynamic,chen2023lifelong} explicitly~(deep gray section of \Tabref{tab:cpt-small}).
% Thirdly, \textbf{\emph{there is a pressing need for exploration into longer sequences of incremental phases in continual pre-training.}} Currently, the longest sequence of domains explored is eight, with content-level distributional shifts~\cite{jin2022lifelong,gururangan2022demix}. However, this falls short of real-world scenarios where continual pre-training may occur more frequently and persist for months or years. The efficacy of continual learning techniques in such prolonged scenarios remains uncertain, as potential performance degradation with longer domain sequences is observed in techniques like EWC~\cite{kirkpatrick2017overcoming}. Additionally, investigating CPT in a task-boundary-free data stream setting is an important avenue for research as well.
% }

% Simple categorization of the three types of the distributional shifts.
\subsubsection{Distributional Shifts in CPT} 
This survey categorizes distributional shifts of CPT into three main types:
(i)~\emph{Language Shift}: LLMs sequentially learn different language corpora, e.g., English $\rightarrow$ Chinese~\cite{gogoulou2024continual,li2024examining}.
(ii)~\emph{Content Shift}: LLMs sequentially learn corpora from different fields, e.g., chemistry $\rightarrow$ biology~\cite{gururangan2022demix,cossu2022continual,jin2022lifelong,qin2023recyclable,chen2023lifelong,gupta2023continual}.
(iii)~\emph{Temporal Shift}: Distributional shifts occur over time, e.g., news in 2021 $\rightarrow$ news in 2022, with a major focus on timestamp-sensitive knowledge retention and update~\cite{amba2021dynamic,jin2022lifelong,dhingra2022time,jang2022towards,jang2022temporalwiki}.
% There is some work continually pre-training LLMs on datasets constructed by errors~\cite{zhao2024large}, re-weighting the samples~\cite{chen2024take} or tokens~\cite{lin2024rho}, and cannot be properly categorized, we use ``Other'' to represent them in \Tabref{tab:cpt-small}.


% Individual description about the papers listed in the table.
\textbf{Language Shift.}\quad 
% In contrast to the common approach of pre-training multilingual language models jointly on large corpora from multiple languages, 
\cite{gogoulou2024continual} focuses on assessing LLMs' natural ability to learn new languages sequentially. With no explicit CL techniques employed, the study observes consistent positive forward transfer of the knowledge, facilitating new language acquisition regardless of the learning order. Forgetting, on the other hand, emerges as a significant challenge that cannot be mitigated by the increasing size of LLMs.
In \cite{li2024examining}, the degree of forgetting of previously learned language when adapting LLMs to a new language is investigated. Various CL techniques, including parameter freezing, LoRA~\cite{hu2021lora}, and (IA)$^3$~\cite{liu2022few}, are evaluated across multiple dimensions. Preliminary experimental results highlight the non-trivial nature of addressing horizontal forgetting for CPT under the language shift as well.
% We argue that research on CPT under language shifts is in its preliminary stages for two main reasons: Firstly, the datasets' scale, including the number of languages and total token count, remains small. Secondly, specific methods targeting language shifts have yet to be proposed; only basic combinations of existing continual learning techniques have been evaluated.



\textbf{Content Shift.}\quad
\cite{yildiz2024investigating} explores the large-scale CPT over 159 content domains, and shows that CPT on various domains can effectively improve models' adaptation ability compared to DAP on single domain.
Similarly, \cite{gupta2023continual} continues the pre-training phase of Pythia~\cite{biderman2023pythia} with no complex CL techniques and discovers that learning rate re-warming consistently improves models trained from scratch. 
Built upon this simple observation, \cite{ibrahim2024simple} further shows that proper combination of learning rate re-warming and re-decay, and replay of the previous data is sufficient to achieve a comparable performance to full re-training. 
LLPT~\cite{jin2022lifelong} establishes a comprehensive training and evaluation protocol for a series of content-level distributional shifts. They assess multiple CL methods and, similar to \cite{gogoulou2024continual}, find consistent forward knowledge transfer, yet horizontal forgetting remains significant. Besides, contrary to the common understanding that experience replay~\cite{chaudhry2019tiny} is the most efficient approach to preventing forgetting, the authors find it ineffective in the case of CPT, due to the potential overfitting issue.
Recyclable Tuning~\cite{qin2023recyclable} shows that if the upstream supplier continually pre-trains LLMs, with or without replay, consumer-side efficiency can be boosted by recycling previously learned update components when proper CL techniques are applied. 

DEMix~\cite{gururangan2022demix} incrementally trains and integrates new experts (DEMix layer) for new domains during CPT. To ensure reasonable inference performance during testing when no domain information is available, it proposes a parameter-free probabilistic approach
% , distinct from the gating function in MoE~\cite{shazeer2017outrageously}, 
to dynamically estimate a weighted mixture of domains. 
% Introducing a new domain variable $D_t$ alongside each word $x_t$, the authors estimate the next word probability $p(x_t|\vx_{<t})$ by marginalizing over all experts\footnote{In the marginalization step of the following equation in the original paper, it is $p(D_t=j|\vx_{t})$ instead of $p(D_t=j|\vx_{<t})$, while we believe this is a minor typo, and hence here we use the updated version.}: 
% \begin{align}
% \nonumber p(x_t|\vx_{<t}) 
%     &= \sum_{j=1}^n p(x_t|\vx_{<t}, D_t=j) \cdot p(D_t=j|\vx_{<t}) 
%     = \sum_{j=1}^n p(x_t|\vx_{<t}, D_t=j) \cdot \left[ \frac{p(\vx_{<t}|D_t=j)\cdot p(D_t=j)}{\sum_{j^\prime=1}^n p(\vx_{<t}|D_t=j^\prime) \cdot p(D_t=j^\prime)} \right],
% \end{align}
% where the conditional probability terms $p(\cdot|\cdot, D_t)$ are calculated by using a specific domain expert.
% The authors develop a large-scale CPT evaluation benchmark comprising eight semantic domains for sequential training and another set of eight domains for assessing LLMs' generalization ability. 
DEMix's modularization has been shown to facilitate efficient domain-adaptive pre-training, promote relevant knowledge during inference, and allow for removable components.
Lifelong-MoE~\cite{chen2023lifelong}, similar to DEMix~\cite{gururangan2022demix}, incrementally trains domain experts for new domains. However, Lifelong-MoE differs from DEMix in utilizing a \emph{token-level gating function} to activate multiple experts for intermediate embedding calculation. During training, previous experts' parameters and gating functions remain frozen, and knowledge distillation loss is employed to regulate parameter updates, which thereby makes Lifelong-MoE robust against the issue of horizontal forgetting.
% Although the data distributions for evaluation are extremely large-scale~(3 domains, 686 billion tokens in total), the Lifelong-MoE is able to efficiently mitigate the issue of horizontal forgetting.

It is noteworthy that some papers draw almost opposite conclusions regarding the significance of CPT for content shifts. For instance, \cite{cossu2022continual} continually pre-trains BERT-based models~\cite{devlin2018bert,liu2019roberta} on five scientific domains and evaluates performance on downstream sentiment analysis. They observe that even the trivial sequential pre-training does not exhibit severe forgetting, prompting reasonable questions about the necessity of CPT. 

\textbf{Temporal Shift.}\quad
In the context of CPT amid content shifts, Multi-Task Learning~(MTL) is often regarded as the upper bound achievable~\cite{pentina2016theoretical, wang2024comprehensive, shi2024unified}. However, this belief does not fully hold when considering CL under temporal shifts~\cite{jang2022towards,jang2022temporalwiki,dhingra2022time}, as temporal shifts can introduce conflicting information, posing challenges for LLMs. For instance, the statement \emph{``Lionel Messi plays for team Barcelona''} remains accurate from 2004 to 2021 but becomes false by 2024, as \emph{``Lionel Messi plays for team Inter Miami''} becomes the correct statement.

Hence, as advocated by CKL~\cite{jang2022towards} and TemporalWiki~\cite{jang2022temporalwiki}, LLMs undergoing continual adaptation to temporal shifts must simultaneously achieve three objectives: (i)~retention of old knowledge, (ii)~acquisition of new knowledge, and (iii)~update of the outdated knowledge.
They evaluate the same set of continual learning baseline methods~\cite{chen2020recall,he2021analyzing,hu2022lora,wang2021kadapter}, 
% , including parameter regularization~(RecAdam~\cite{chen2020recall}), rehearsal~(Mix-review~\cite{he2021analyzing}), and parameter expansion~(LoRA~\cite{hu2022lora} and K-Adapter~\cite{wang2021kadapter}), 
each highlighting distinct aspects of their impact. CKL~\cite{jang2022towards} 
% introduces a unified metric, \textbf{FUAR}~(\textbf{F}orgetting / (\textbf{U}pdated + \textbf{A}cquired) \textbf{R}atio), to assess the three learning objectives collectively. They 
observes that parameter expansion consistently exhibits robust performance across all experimental conditions. In contrast, replay-based methods struggle to efficiently adapt to new knowledge acquisition and outdated knowledge update, leading to rapid forgetting of newly learned information during training.
TemporalWiki~\cite{jang2022temporalwiki} constructs a series of temporal corpora and their differential sets from sequential snapshots of Wikipedia, revealing that updating LLMs on these differential sets substantially enhances new knowledge acquisition and updates, requiring significantly less computational resources, and various CL techniques prove effective in mitigating horizontal forgetting during this process. 
LLPT~\cite{jin2022lifelong} introduces temporal generalization evaluation for LLMs pre-trained on sequential corpora. Through experiments on a large-scale chronologically-ordered Tweet Stream, the authors demonstrate the superiority of CPT combined with CL techniques to task-specific LMs, in terms of both knowledge acquisition and temporal generalization. Nonetheless, these preliminary experiments do not conclusively determine which specific CL method is more preferable than the others.

Another line of work, Temporal Language Models (TLMs), takes a different approach to address knowledge retention, acquisition, and update under temporal shifts by integrating temporal information into the model~\cite{rosin2022time,dhingra2022time,su2023efficient}. During training, they inject temporal information into training examples as prefixes of prompts, using special tokens~\cite{rosin2022time}, explicit year information~\cite{dhingra2022time}, or syntax-guided structural information~\cite{su2023efficient}. In sequential training experiments conducted by TempoT5~\cite{dhingra2022time}, comparison between continually and jointly pre-trained LMs demonstrates that CPT better balances adaptation and forgetting when the replay rate of past data is appropriately set.

\textbf{Others.}\quad
CPT as a technique to progressively attain novel knowledge, can be used to refine LLMs' behavior.
CEM~\cite{zhao2024large} collects examples where the model's response is incorrect and continually trains the model on these examples, along with a supplemental dataset.
RHO-1~\cite{lin2024rho} proposes Selective Language Modeling (SLM), which employs a reference model to evaluate the perplexity of each token in the training corpus, and continually pre-trains the model on high-perplexity tokens. 
Similarly, IR-DRO~\cite{chen2024take} re-trains the model on re-weighted examples from the original pre-training dataset, focusing more on higher-loss sequences.


The significance of addressing temporal shifts through CPT is underscored by several industrial studies. For instance, \cite{amba2021dynamic} employs a dynamic vocabulary expansion algorithm and an efficient sub-sampling procedure to conduct CPT on large-scale emerging tweet data. Conversely, \cite{loureiro2022timelms} adopts CPT without explicit measures to constrain model updates, releasing a series of BERT-based LMs incrementally trained on new tweet data every three months. Preliminary experimental results demonstrate substantial improvements of continually pre-trained LMs over the base BERT model across downstream tasks. While some studies question the necessity of continually adapting LLMs along the temporal axis for environmental reasons, such as reducing CO$_2$ emissions~\cite{attanasio2023worth}, the community commonly embraces CPT as a more efficient learning paradigm compared to the traditional ``combine-and-retrain'' approach. 




% List existing work

% Missing things and prospective research direction



% \begin{table*}[t]
%     \centering
%     \caption{
%     \textbf{Summary of the existing studies on Horizontal Continual Domain-Adaptive Pre-Training of LLMs,} where the papers are organized in the chronological order.
%     1. Domain(s): use ``/'' and ``$\rightarrow$'' to denote the DAP is performed in parallel and sequentially to different domains, respectively. 
%     }
%     \label{tab:pre-training-big}
%     \resizebox{1\linewidth}{!}{%
% \begin{tabular}{ccccc cccc}
% 	\toprule[0.15em]
% 	\multirow{2}{*}[-0.25em]{\textbf{Domain}} & 
%     \multirow{2}{*}[-0.25em]{\textbf{Method}} & 
%     \multirow{2}{*}[-0.25em]{\textbf{Train Proc.}} & 
%     \multirow{2}{*}[-0.25em]{\textbf{LLM Arch.}} & 
%     \multicolumn{3}{c}{\textbf{{Continual Learning Tech.}}} & 
%     \multicolumn{2}{c}{\textbf{{Continual Learning Eval.}}} \\
%     \cmidrule(lr){5-7}\cmidrule(lr){8-9}
%     & & & & \emph{Rehearsal} & \emph{Param. Reg.} & \emph{Arch. Exp.} & \emph{Backward Transfer} & \emph{Forward Transfer} \\
% 	%  &
%     \cmidrule[0.15em]{1-9}
%     \multirow{2}{*}[-0.7em]{\textbf{Legal}} & LayerLlama~\cite{huang2023lawyer} & \small{DAP $\rightarrow$ G-SFT $\rightarrow$ D-SFT} & Llama & Replay & \xmark & \xmark & Zero-Shot & Zero-Shot \\
%     \cmidrule{2-9}
%     & SaulLM~\cite{colombo2024saullm7b} & \small{DAP $\rightarrow$ IT} & Mistral & Replay & \xmark & \xmark & \xmark & \makecell{Perplexity \\ Zero-Shot} \\
%     \midrule
%     % Medical 
%     \multirow{6}{*}[-4em]{\makecell{\textbf{Medical}}} & PMC-Llama~\cite{wu2023pmc} & \small{DAP $\rightarrow$ IT} & Llama & Replay & \xmark & \xmark & \xmark & \makecell{Zero-Shot \\ Fine-Tune} \\
%     \cmidrule{2-9}
%     & AF Adapter~\cite{yan2023af} & \small{DAP $\rightarrow$ FT} & RoBERTa & \xmark & \xmark & \makecell{Layer Exp. \\ LoRA$^{\text{\club}}$} & Acc. & \makecell{Loss \\ Fine-Tune}\\
%     \cmidrule{2-9}
%     & \cite{rongali2021continual} & \small{DAP $\rightarrow$ FT} & \makecell{BERT \\ RoBERTa \\ DistilBERT} & \makecell{Replay$^{\text{\club}}$ \\ GEM$^{\text{\club}}$} & \makecell{L2 Reg.$^{\text{\club}}$ \\ EWC$^{\text{\club}}$} & \xmark & \makecell{Loss \\ Fine-Tune} & \makecell{Loss \\ Fine-Tune}\\
%     \cmidrule{2-9}
%     & \cite{guo2023continuous} & \small{DAP $\rightarrow$ FT} & Llama2 & \xmark & \xmark & \xmark & \makecell{Few-Shot \\ Fine-Tune} & \makecell{Few-Shot \\ Fine-Tune}\\
%     \cmidrule{2-9}
%     & BioMedGPT~\cite{luo2023biomedgpt} & \small{DAP $\rightarrow$ MM-SFT} & Llama2 & \xmark & \xmark & \xmark & \xmark & Fine-Tune \\
%     \cmidrule{2-9}
%     & HuatuoGPT-II~\cite{Chen2023HuatuoGPTII} & \small{DAP $\rightarrow$ U-SFT} & Baichuan2 & Replay & \xmark & \xmark & Zero-Shot & \makecell{Zero-Shot \\ Human Eval.}\\
%     \midrule
%     % Financial
%     \multirow{5}{*}[-1.5em]{\textbf{Financial}} & WeaverBird~\cite{Xue2023WeaverBird} & \small{DAP} & GLM2 & \xmark & \xmark & LoRA & \xmark & Human Eval. \\
%     \cmidrule{2-9}
%     & \cite{xie2023efficient} & \small{DAP} & Pythia & \xmark & \xmark & \xmark & \makecell{Loss \\ Few-Shot} & \makecell{Loss \\ Few-Shot} \\
%     \cmidrule{2-9}
%     & BBT-Fin~\cite{Lu2023BBTFin} & \small{DAP} & T5 & \xmark & \xmark & \xmark & \xmark & Fine-Tune \\
%     \cmidrule{2-9}
%     & CFGPT~\cite{li2023cfgpt} & \small{DAP $\rightarrow$ SFT} & InternLM & \xmark & \xmark & Q-LoRA$_{\text{(SFT)}}$ & \xmark & Human Eval.$^1$ \\
%     \cmidrule{2-9}
%     & XuanYuan-2.0~\cite{Zhang2023xuanyuan} & \small{DAP + SFT} & BLOOM & Replay & \xmark & \xmark & Human Eval. & Human Eval.\\
%     \midrule
%     %%%%%%%% Scientific
%     \multirow{7}{*}[-5em]{\makecell{\textbf{Scientific}}} & Llema~\cite{Azerbayev2023LLEMMA} & \small{DAP} & ChatGLM & Replay & \xmark & \xmark & \xmark & \makecell{Perplexity \\ Few-Shot} \\
%     \cmidrule{2-9}
%     & AstroLlama~\cite{Nguyen2023AstroLLaMA} & \small{DAP} & LlaVa & \xmark & \xmark & \xmark & \xmark & Perplexity \\
%     \cmidrule{2-9}
%     & PLlama~\cite{Yang2023PLLaMa} & \small{DAP $\rightarrow$ IT} & GAL & Replay & \xmark & \xmark & Loss & \makecell{Loss \\ Zero-Shot} \\
%     \cmidrule{2-9}
%     & OceanGPT~\cite{Bi2023OCEANGPT} & \small{DAP $\rightarrow$ IT} & \makecell{Vicuna \\ Llama2-chat \\ ChatGLM2} & \xmark & \xmark & LoRA$_{\text{(IT)}}$ & \xmark & Human Eval.\\
%     \cmidrule{2-9}
%     & K2~\cite{deng2023learning} & \small{DAP $\rightarrow$ SFT} & Llama & \xmark & \xmark & LoRA$_{\text{(SFT)}}$ & \xmark & \makecell{Perplexity \\ Zero-Shot \\ LLM Eval.} \\
%     \cmidrule{2-9}
%     & GeoGalactica~\cite{lin2023geogalactica} & \small{DAP $\rightarrow$ G-SFT $\rightarrow$ D-SFT} & GAL & \xmark & \xmark & \xmark & Zero-Shot & \makecell{Perplexity \\ Zero-Shot \\ LLM Eval.} \\
%     \cmidrule{2-9}
%     & MarineGPT~\cite{Zheng2023MarineGPT} & \small{MM-DAP $\rightarrow$ MM-IT} & Llama & \xmark & \xmark & \xmark & \xmark & Human Eval. \\
%     \midrule
%     % Code
%     \multirow{7}{*}[-5.5em]{\textbf{Code}} & CodeGen~\cite{nijkamp2022codegen} & \small{DAP $\rightarrow$ DAP} & CodeGen & \xmark & \xmark & \xmark & \xmark & \makecell{Perplexity \\ Zero-Shot}\\
%     \cmidrule{2-9} 
%     & CodeLlama~\cite{rozière2024code} & \makecell{\small{DAP $\rightarrow$ LC-FT $\rightarrow$ IT} \\ \small{DAP $\rightarrow$ DAP $\rightarrow$ LC-FT}} & Llama2 & \xmark & \xmark & \xmark & \xmark & \makecell{Perplexity \\ Zero-Shot} \\
%     \cmidrule{2-9} 
%     & StarCoder~\cite{li2023starcoder} & DAP & StarCoder & \xmark & \xmark & \xmark & \makecell{Perplexity \\ Zero-Shot \\ Few-Shot} & \makecell{Perplexity \\ Zero-Shot \\ Few-Shot} \\
%     \cmidrule{2-9} 
%     & DeepSeek-Coder~\cite{guo2024deepseekcoder} & \small{DAP} & DeepSeek-LLM & \xmark & \xmark & \xmark & \makecell{Zero-Shot \\ Few-Shot} & \makecell{Zero-Shot} \\
%     \cmidrule{2-9} 
%     & IRCoder~\cite{paul2024ircoder} & \small{DAP} & \makecell{StarCoder \\ DeepSeekCoder \\ CodeLlama} & \xmark & \xmark & LoRA & \xmark & Zero-Shot \\
%     \cmidrule{2-9} 
%     & Struct-Rep~\cite{agarwal2024structured} & \small{DAP $\rightarrow$ FT} & \makecell{CodeT5 \\ CodeGen} & \xmark & \xmark & \xmark & \xmark & \makecell{Zero-Shot \\ Few-Shot}\\
%     \cmidrule{2-9} 
%     & Comment-Aug~\cite{song2024code} & \small{DAP} & \makecell{Llama2 \\ Code Llama}  & \xmark & \xmark & \xmark & \xmark & Zero-Shot \\
%     \midrule
%     %%%%%%%%% Others
%     \multirow{5}{*}[-2.4em]{\textbf{Others}} & \cite{gururangan2020dont} & \small{DAP $\rightarrow$ FT} & RoBERTa & \xmark & \xmark & \xmark & Loss & \makecell{Loss \\ Fine-Tune} \\
%     \cmidrule{2-9}
%     & EcoNet~\cite{han2021econet}$^1$ & \small{DAP $\rightarrow$ FT} & \makecell{BERT \\ RoBERTa} & \xmark & \xmark & \xmark & \xmark & Fine-Tune \\
%     \cmidrule{2-9}
%     & CALM~\cite{zhou2020pre} & \small{DAP $\rightarrow$ FT} & T5 & \xmark & \xmark & \xmark & \xmark & Fine-Tune \\
%     \cmidrule{2-9}
%     & QUERT~\cite{xie2023quert} & \small{DAP $\rightarrow$ FT} & \makecell{BERT \\ ERNIE} & \xmark & \xmark & \xmark & \xmark & Fine-Tune \\
%     \cmidrule{2-9}
%     & EcomGPT-CT~\cite{ma2023ecomgptct} & \small{DAP $\rightarrow$ SFT} & BLOOM & Replay & \xmark & \xmark & \makecell{Zero-Shot \\ Few-Shot} & \makecell{Zero-Shot \\ Few-Shot} \\
% 	\bottomrule[0.15em]
% 	\end{tabular}
% 	}
% \end{table*}
% \footnotetext[1]{In this paper, only qualitative demonstration has been shown.}



\subsection{Domain-Adaptive Pre-training~(DAP)}
\label{sec:dap}
% General background of DAP.
\textbf{Background of DAP.}\quad
Institutions, regardless of size, often possess significant amounts of unlabeled, domain-specific data. This data bridges the gap between general-purpose LLMs trained on diverse corpora and fine-tuned LLMs designed for specific downstream tasks. Leveraging this data as a preparatory stage can facilitate effective adaptation of LLMs to downstream tasks.
Such process of ``continued/continual/continuous pre-training''~\cite{yan2023af,guo2023continuous,ma2023ecomgptct,han2021econet,xie2023efficient,xie2023quert,huang2023lawyer,Lu2023BBTFin,Xie2023PIXIU,Azerbayev2023LLEMMA,yue2023mammoth,colombo2024saullm7b,Zhang2024SciGLM,shen2024tag}, 
``further pre-training''~\cite{song2024code,lin2023geogalactica,deng2023learning,Rubungo2023LLM-Prop,agarwal2024structured}, 
``domain tuning''~\cite{rongali2021continual},
``knowledge enhancement pre-training''~\cite{Lu2023BBTFin}, 
and ``knowledge injection training''~\cite{wu2023pmc} 
is unified and termed ``\textbf{\emph{Domain Adaptive Pre-training~(DAP)}}''~\cite{gururangan2020dont} for clarity and consistency throughout this survey.
In the pioneering work of domain-adaptive pre-training~(DAPT)~\cite{gururangan2020dont}, the authors continuously pre-train the language models on a larger domain-specific dataset before fine-tuning them to the downstream tasks, resulting in universally improved performance aross various tasks. 
As the observation above has been validated on multiple domains in parallel, including BioMed, CS, News, and Reviews~\cite{gururangan2020dont}, practitioners commonly accept that employing DAP on additional unlabeled domain-specific data benefits downstream tasks. Consequently, this technique has become widely deployed in many modern LLMs. 

% General intro to the table.
\textbf{Summary of LLMs with DAP.}\quad
We provide a summary of the existing 41 studies utilizing DAP for LLMs in \Tabref{tab:dap}. Each entry is characterized by three main features: (i)~training process specifications, encompassing the vertical domain for which LLMs are trained, the training pipeline preceding release, and the LLM architecture employed; (ii)~adopted continual learning techniques, including rehearsal, parameter regularization, and architecture expansion; and (iii) evaluation metrics for CL, such as backward transfer (forgetting) and forward transfer (adaptation to downstream data). 
% Following this overview, we will present general observations about DAP in \Secref{sec:dap-obs}, followed by a detailed introduction to LLMs developed in vertical domains in \Secref{sec:dap-domains}.




\subsubsection{General Observation on DAP}\label{sec:dap-obs}
Several key observations emerge regarding the research landscape of DAP~(\Tabref{tab:dap}). 
\begin{itemize}
    \item \textbf{OBS-1: DAP predominantly occurs in a single stage.}
    Continual DAP which involves more than one stage is seldom explored: among all papers listed in \Tabref{tab:dap}, only one employs two stages of DAP~(``PT $\rightarrow$ DAP $\rightarrow$ DAP $\rightarrow$ FT'' in Code Llama \cite{rozière2024code}). 
    % In Code Llama~\cite{rozière2024code}, aimed at developing a language model tailored to Python programming, the authors initialize the model from the pre-trained Llama~2 checkpoint. They then conduct the first stage of DAP across multiple programming languages (500 billion tokens) before proceeding to the second stage, focusing solely on Python code (100 billion tokens). Finally, they perform long context fine-tuning (20 billion tokens) to enhance the model's capability in challenging long-context scenarios of code generation. This PT~$\rightarrow$~DAP~$\rightarrow$~DAP~$\rightarrow$~FT pipeline represents the sole example found thus far that strictly adheres to the definition and hierarchical structure of vertical continuity in pre-training and adapting LLMs for final end-use.
    It is arguably reasonable to categorize studies that conduct only one stage of DAP and nothing more~\cite{Lu2023BBTFin,Nguyen2023AstroLLaMA,song2024code,xie2023efficient,li2023starcoder,guo2024deepseekcoder,Xue2023WeaverBird,paul2024ircoder,Azerbayev2023LLEMMA,cheng2024adapting} into CPT rather than DAP. Nevertheless, considering that they aim to adapt a general-purpose LLM to a specific domain, we include them in this section.
    \item \textbf{OBS-2: The notion of interpreting DAP through the lens of CL, whether intentional or not, is widely embraced.}
    As shown in~\Tabref{tab:dap}, except for the first section~(white, 13/41), where papers overlook any potential side effects of DAP leading to vertical forgetting, the remaining sections~(all gray, 28/41) either evaluate the potential negative impacts of DAP or proactively employ CL techniques to mitigate the risk of vertical forgetting.
    \item \textbf{OBS-3: Further research of more sophisticated CL techniques for not just DAP, but general vertical continual learning is much needed.} It is supported by the widespread adoption of CL techniques (22/41) for training domain-specific LLMs. However, the diversity of these techniques is limited, with only replay~\cite{colombo2024saullm7b,wu2023pmc,Azerbayev2023LLEMMA,rongali2021continual,Chen2023HuatuoGPTII,Zhang2023xuanyuan,Yang2023PLLaMa,ma2023ecomgptct,huang2023lawyer,cheng2024adapting} and parameter expansion (LoRA~\cite{Xue2023WeaverBird,paul2024ircoder,wu2024llama,yan2023af}) or Layer/Block expansion~\cite{wu2024llama,yan2023af} utilized. 
    In fact, it appears that individuals may not explicitly recognize that DAP should be viewed from the perspective of vertical continuity, as they often employ CL techniques unknowingly, e.g., studies deploying replay terming the technique as ``data combination''~\cite{wu2023pmc} or ``data mixing/mixture''~\cite{Azerbayev2023LLEMMA,Yang2023PLLaMa,ma2023ecomgptct,cheng2024adapting}, without recognizing it as a typical CL solution to vertical continual learning. 
    % This deduction arises from two observations: 
    % (i) parameter expansion methods~\cite{hu2022lora,yan2023af,wu2024llama} embody implicit CL techniques, as they put constraint on the size of the tunable parameters;
    % For instance, in LoRA~\cite{hu2022lora}, the increment of weights $\Delta\mW=\mO$ preserves the original performance on previous data distributions, but once adaptation occurs ($\Delta\mW\neq\mO$), forgetting on the original data distribution follows. This analysis extends to other parameter expansion techniques such as layer expansion~\cite{yan2023af} and block expansion~\cite{wu2024llama}. Authors typically empirically demonstrate the effectiveness of these approaches, attributing forgetting mitigation to the low-rank property and parameter efficiency;
    % (ii) studies deploying replay often term the technique as ``data combination''~\cite{wu2023pmc} or ``data mixing/mixture''~\cite{Azerbayev2023LLEMMA,Yang2023PLLaMa,ma2023ecomgptct,cheng2024adapting}, without recognizing it as a typical CL solution to vertical continual learning.
\end{itemize}

% Firstly, \textbf{\emph{DAP predominantly occurs in a single stage.}} Horizontal Continual DAP which involves more than one stage is seldom explored: among the 34 papers listed, only one paper employs two stages of DAP~\cite{rozière2024code}. 
% In Code Llama~\cite{rozière2024code}, aimed at developing a language model tailored to Python programming, the authors initialize the model from the pre-trained Llama~2 checkpoint. They then conduct the first stage of DAP across multiple programming languages (500 billion tokens) before proceeding to the second stage, focusing solely on Python code (100 billion tokens). Finally, they perform long context fine-tuning (20 billion tokens) to enhance the model's capability in challenging long-context scenarios of code generation. This PT~$\rightarrow$~DAP~$\rightarrow$~DAP~$\rightarrow$~FT pipeline represents the sole example found thus far that strictly adheres to the definition and hierarchical structure of vertical continuity in pre-training and adapting LLMs for final end-use.
% Hence, categorizing the 10 studies that solely conduct one stage of DAP and nothing more~\cite{Lu2023BBTFin,Nguyen2023AstroLLaMA,song2024code,xie2023efficient,li2023starcoder,guo2024deepseekcoder,Xue2023WeaverBird,paul2024ircoder,Azerbayev2023LLEMMA,cheng2024adapting} proves challenging. One could also argue that they deploy an additional single stage of CPT rather than DAP. Nevertheless, considering that all these papers aim to adapt a general-purpose LLM to a specific domain, we include them in this section for discussion, aligning with the categorization we have established thus far.

% Secondly, \textbf{\emph{the notion of interpreting DAP through the lens of continual learning, whether intentional or not, is widely embraced.}} As demonstrated in \Tabref{tab:dap}, with the exception of the first section (white, 11/33), where papers overlook any potential side effects of DAP leading to vertical forgetting of previously learned general knowledge, the remaining sections (all gray, 22/33) either evaluate the potential negative impacts of DAP or proactively employ continual learning techniques to mitigate the risk of vertical forgetting from the outset.

% Thirdly, we observe widespread adoption of CL techniques (14/33) for training domain-specific LLMs. However, the diversity of these techniques is limited, with only replay~\cite{colombo2024saullm7b,wu2023pmc,Azerbayev2023LLEMMA,rongali2021continual,Chen2023HuatuoGPTII,Zhang2023xuanyuan,Yang2023PLLaMa,ma2023ecomgptct,huang2023lawyer,cheng2024adapting} and parameter expansion (LoRA~\cite{Xue2023WeaverBird,paul2024ircoder,wu2024llama,yan2023af}) or Layer/Block expansion~\cite{wu2024llama,yan2023af} being utilized. \textbf{\emph{This highlights the need for further research to investigate, incorporate, and design more sophisticated CL techniques for not just DAP, but vertical continual learning in general.}}
% In fact, it appears that individuals may not explicitly recognize that DAP should be viewed from the perspective of vertical continuity, as they often employ CL techniques unknowingly. This deduction arises from two observations: 
% (i) parameter expansion methods inherently embody implicit CL techniques. For instance, in LoRA~\cite{hu2022lora}, the increment of weights $\Delta\mW=\mO$ preserves the original performance on previous data distributions, but once adaptation occurs ($\Delta\mW\neq\mO$), forgetting on the original data distribution follows. This analysis extends to other parameter expansion techniques such as layer expansion~\cite{yan2023af} and block expansion~\cite{wu2024llama}. Authors typically empirically demonstrate the effectiveness of these approaches, attributing forgetting mitigation to the low-rank property and parameter efficiency;
% (ii) excluding parameter expansion methods, replay emerges as the only CL technique employed during DAP, except in cases where extensive empirical investigations of CL methods are conducted~\cite{rongali2021continual}. Furthermore, studies deploying replay often term the technique as ``data combination''~\cite{wu2023pmc} or ``data mixing/mixture''~\cite{Azerbayev2023LLEMMA,Yang2023PLLaMa,ma2023ecomgptct,cheng2024adapting}, without recognizing it as a vertical continual learning problem.


% In fact, we can even conclude that people are \emph{not} explicitly aware that DAP should be viewed from the CL perspective, and they are using the CL techniques without realizing it. This conclusion is deduced from the two facts: 
% (i)~parameter expansion themselves are implicit continual learning methods, as there will be no guarantee that the functionality on the previous data distribution will be preserved once the expanded parameters are trained on the current data. For example, in LoRA~\cite{hu2022lora}, when the increment of the weights $\Delta\mW=\mO$ are initialized to zero matrix, the original weights $\mW+\Delta\mW$ preserves the full performance on the original data distribution. Once the adaptation happens, causing $\Delta\mW\neq\mO$, the forgetting on the original data distribution ensues. This analysis as well applies to other parameter expansion techniques including layer expansion~\cite{yan2023af} and block expansion~\cite{wu2024llama}. In both of their experiments, the authors only empirically demonstrate the effectiveness of these parameter expansion approaches, and attributes the success of forgetting mitigation to the low-rank property and parameter-efficiency, which is originally the motivation of adopting such techniques. 
% (ii)~if we strictly exclude the parameter expansion methods, then replay remains the only CL method adopted during DAP, except the work doing extensive empirical investigation of CL methods~\cite{rongali2021continual}. In addition, in these studies that deploy replay, they often term the technique of replay as ``data combination''~\cite{wu2023pmc}, and ``data mixing/mixture''~\cite{Azerbayev2023LLEMMA,Yang2023PLLaMa,ma2023ecomgptct,cheng2024adapting}, without realizing it is a continual learning problem to be coped with.




\subsubsection{Different Domains of DAP}
\label{sec:dap-domains}
We include work aimed at establishing vertical LLMs across various domains, including legal, medical, financial, scientific, and code. Additionally, we cover other domains such as language and e-commerce.

\textbf{Legal Domain.}\quad 
% Given the legal industry's demand for managing ever-growing volumes of legal documents, there's a burgeoning need to harness LLMs to aid legal professionals in navigating, interpreting, and generating high-quality legal materials~\cite{xiao2021lawformer}. 
% While general-purpose LLMs may perform adequately on some legal benchmarks~\cite{martin2024better}, customizing LLMs with additional unlabeled resources specific to the legal domain can yield superior results. This is because the high-volume unlabeled legal corpus resembles the conditions under which general-purpose LLMs are pre-trained.
In Layer Llama~\cite{huang2023lawyer}, the authors gathered publicly available legal texts from China Courts websites, 
% including judgment documents, legal articles, judicial interpretations, court news, and law popularization articles, 
totaling approximately 10 billion tokens as noted in a GitHub issue. 
In SaulLM~\cite{colombo2024saullm7b}, the authors collected the DAP corpus from various jurisdictions in different countries, 
% such as the U.S., Europe, and Australia, 
resulting in a corpus of 30 billion tokens to cover diverse aspects of legal texts. When combined with previously available datasets, the total number of tokens used for legal-domain DAP reaches 94 billion. 
The substantial volume of DAP data, while offering valuable insights into specific domains, increases the risk of vertical forgetting of the general knowledge due to the large number of update steps involved. To mitigate this issue, SaulLM incorporates general data from Wikipedia, StackExchange, and GitHub into the DAP data, constituting about 2\% of the final dataset~\cite{colombo2024saullm7b}. 
% Following DAP, SaulLM then employs a combination of general-domain and legal-domain instructions to enhance the model's instruction-following ability (U-IT, see \Tabref{tab:dap}).
Similarly, Lawyer Llama incorporates replaying general-domain data during DAP, but the replay rate is not disclosed~\cite{huang2023lawyer}. 
\cite{takahashi2024pretraining} also replays of non-latest business documents during DAP when building a Japanese business-specific LLM. 
% After DAP, it undergoes two distinct phases of instruction tuning: first, general-domain instruction tuning (G-IT), followed by domain-specific downstream legal consultation application instruction tuning (D-IT). However, no explanation is provided for why two-stage IT is preferred over consolidation into one as in SaulLM~\cite{colombo2024saullm7b}, leaving this as an open question for future research.

% At the same time, the huge volume of the DAP data can easily make the model prone to catastrophic forgetting, as the number of update steps is now inevitably large. Hence to reduce the risk of general knowledge forgetting, SaulLM mixes the commonly available general data from Wikipedia, StackExchange, and GitHub into the DAP data, comprising about 2\% of the final data~\cite{colombo2024saullm7b}. After DAP, it uses a mix of general-domain instructions and legal-domain instructions to improve the instruction following ability of the model~(U-IT as shown in \Tabref{tab:dap}). 
% Lawyer~Llama~\cite{huang2023lawyer} as well replays the general-domain data during DAP with no disclosure of the replay rate. After DAP, it adopts two separate phases of instruction tuning: it first performs a general-domain instruction tuning~(G-IT) and then adapts to the domain-specific downstream legal consultation application with instruction tuning~(D-IT). However, no discussion is provided explaining why two-stage IT is performed rather than mixing them into one, as done in SaulLM~\cite{colombo2024saullm7b}, which makes it an open question for future work.





\textbf{Medical Domain.}\quad 
% The development of LLMs holds promise for revolutionary changes in the medical industry, offering potential improvements in efficiency and quality across medical communication, disease diagnosis, and decision-making for doctors~\cite{li2023chatdoctorjeblick2022chatgpt}. 
% While some instances of general-domain LLMs have shown success in providing useful advice and accelerating diagnosis progress for patients~\cite{}, direct deployment poses risks. These risks include the potential for sub-optimal solutions, such as imprecise medical advice, and the possibility of harm, such as incorrect drug recommendations and the propagation of medical misinformation~\cite{}.
Efforts have been made to develop medical specialists by either training an LLM from scratch~\cite{gu2021domain,luo2022biogpt} or fine-tuning publicly-available LLMs to meet specific medical needs~\cite{luo2023biomedgpt,wu2023pmc,Chen2023HuatuoGPTII}. Among these approaches, DAP techniques have been extensively utilized to preserve the communication and instruction-following abilities of a general LLM, preparing it for subsequent medical applications~\cite{luo2023biomedgpt,wu2023pmc,Chen2023HuatuoGPTII}.
BioMedGPT~\cite{luo2023biomedgpt} is a multi-modal biomedical language model that integrates representations of human language and the language of life (molecules, proteins, cells, genes, etc.). Prior to final multi-modal supervised fine-tuning, the authors initialize the model from Llama2-Chat~\cite{touvron2023llama2} and conduct DAP using extensive biomedical documents from S2ORC~\cite{lo2020s2orc}, without considering any CL techniques or evaluations.
In \cite{guo2023continuous}, DAP is performed using Chinese medical encyclopedias and online expert articles, 
% yielding over 364,000 question-answer pairs, 
with next-token prediction as the training objective. During DAP, the performance gradually deteriorates on general-domain datasets as the training step increases, but improves on the downstream medical examination tasks~\cite{hendryckstest2021}.
% To ensure the model possesses sufficient fundamental medical knowledge before the instruction tuning phase, 
PMC-LLama~\cite{wu2023pmc} gathers biomedical papers from S2ORC~\cite{lo2020s2orc} and medical textbooks for ``knowledge injection training.'' During this phase, a general language corpus from RedPajama-Data~\cite{together2023redpajama} is replayed at a 5\% rate within a training batch. However, the paper does not analyze the effectiveness of this operation of mixing in general-domain data for DAP.


To mitigate vertical forgetting, AF Adapter~\cite{yan2023af} proposes an adapter structure extending the width of Attention layers and FFNs for acquiring domain knowledge and only the adapters are tuned during DAP. 
% To gauge the extent of vertical forgetting, the authors collect a small subset of general-domain samples from WuDao~\cite{sha2021wudao} and calculate the accuracy of masked word prediction and find minimal performance degradation.
Similarly, Hippocrates~\cite{acikgoz2024hippocrates} deploys LoRA during DAP to both have medical-specific knowledge injected and general ability preserved. 
Me-Llama~\cite{xie2024me} mixes in about 25\% of the general-domain data for DAP on the clinical notes and biomedical articles, which achieves even positive backward transfer on MMLU~\cite{hendryckstest2021}. 
HuatuoGPT-II~\cite{Chen2023HuatuoGPTII} proposes to fuse the DAP into the final SFT, unifying the two stages into one single process. The challenge of such process mainly comes from the data heterogeneity of DAP's unlabeled corpus. The authors address this challenge by reformulating paragraphs of data into \emph{(instruction, output)} format using existing large language models. They further employ a priority sampling strategy to avoid compromising downstream ability, a pitfall observed in the fixed-rate data mixing strategy~\cite{touvron2023llama2}. This paper empirically demonstrates the superiority of unified one-stage SFT over two-stage training, questioning the reasonability of the current DAP.
On medical-domain data, \cite{rongali2021continual}
% , the authors investigate the effectiveness of CL techniques for mitigating vertical forgetting in DAP on medical-domain corpus. They 
finds that LMs constrained by CL techniques on source domains exhibit greater robustness to future domain shifts. Specifically, they identify that parameter regularization techniques like EWC~\cite{kirkpatrick2017overcoming}, despite slightly higher cost, can facilitate positive forward and backward transfer.


%%%%%%%% Second Version, sorted in the continual learning significance
\begin{table*}[htbp]
    \centering
    \caption{
    \textbf{Summary of the existing studies that leverage Domain-Adaptive Pre-training of LLMs,} 
    where the papers are organized in four main categories based on whether they (i) adopt the \emph{continual learning techniques} and (ii) perform the evaluation for \emph{backward transfer~(forgetting)}.
    In the column of \textbf{Train Proc.}~(Training Process), we omit the phase of general Pre-Training. DAP represents Domain-Adaptive Pre-Training; SFT represents Supervised Fine-Tuning; IT represents Instruction Tuning. The prefix G- and D- represent General and Domain-Specific training process~\cite{lin2023geogalactica,huang2023lawyer}, and the prefix U- represents them unified~\cite{wu2024llama,Chen2023HuatuoGPTII}. The prefix MM- and LC- represents Multi-Modal and Long-Context training phases~\cite{luo2023biomedgpt,Zheng2023MarineGPT,rozière2024code}.
    In the column of \textbf{Continual Learning Eval}., we consider two criteria: (i)~\emph{Backward Transfer}, i.e., performance degradation on the previous tasks, which is also known as catastrophic forgetting, (ii)~\emph{Forward Transfer}, i.e., the performance gained by DAP while transferring the LLMs to the downstream tasks. We use L and Perp. to denote Loss and Perplexity, FT to denote Fine-Tuning, ZS and FS to denote Zero-Shot and Few-Shot Accuracy, HE and LLM to denote the Human Evaluation and LLM Evaluation for generative tasks.
    % Among 33 papers presented in this table that adopt DAP during the development, nearly 65\% (22/33) of them explicitly study the influence of DAP from a continual learning perspective: they either evaluate the degree of forgetting, or adopt the continual learning techniques to prevent forgetting of the general knowledge. 
    % However, there is a significant lack of diversity of the continual learning techniques adopted in these works (only Replay and LoRA), which advocates the further study on the efficacy of vertical continual learning in the realm of LLMs. 
    }
    \label{tab:dap}
    \resizebox{1\linewidth}{!}{%
    \setlength{\tabcolsep}{5pt}
    \begin{tabular}{ccccC CCcc}
    	\toprule[0.15em]
    	\multirow{2}{*}[-0.25em]{\textbf{Domain}} & 
        \multirow{2}{*}[-0.25em]{\textbf{Method}} & 
        \multirow{2}{*}[-0.25em]{\textbf{Train Proc.}} & 
        \multirow{2}{*}[-0.25em]{\textbf{LLM Arch.}} & 
        \multicolumn{3}{c}{\textbf{{Continual Learning Tech.}}} & 
        \multicolumn{2}{c}{\textbf{{Continual Learning Eval.}}} \\
        \cmidrule(lr){5-7}\cmidrule(lr){8-9}
        & & & & \emph{Rehearsal} & \emph{Param. Reg.} & \emph{Arch. Exp.} & \emph{Backward Transfer} & \emph{Forward Transfer} \\
    	%  &
        \midrule
        \midrule

        %%%%%%%%%%%%%%%%%%%%%%%%%%%%%%%%%%%
        %%%% No CL Tech.; No CL Eval.
        %%%% 13 papers in total.
        %%%%%%%%%%%%%%%%%%%%%%%%%%%%%%%%%%%
        Medical & BioMedGPT~\cite{luo2023biomedgpt} 
        & \small{DAP $\rightarrow$ MM-SFT} & Llama2 
        & \xmark & \xmark & \xmark 
        & \xmark & FT \\
        \hline

        
        Financial & BBT-Fin~\cite{Lu2023BBTFin} 
        & \small{DAP} & T5 
        & \xmark & \xmark & \xmark 
        & \xmark & FT \\
        \hline

        
        Financial & CFGPT~\cite{li2023cfgpt} 
        & \small{DAP $\rightarrow$ SFT} & InternLM 
        & \xmark & \xmark & Q-LoRA$_{\text{(SFT)}}$ 
        & \xmark & HE$^1$ \\
        \hline

        
        Scientific & AstroLlama~\cite{Nguyen2023AstroLLaMA} 
        & \small{DAP} & LlaVa 
        & \xmark & \xmark & \xmark 
        & \xmark & Perp. \\
        \hline

        
        Scientific & OceanGPT~\cite{Bi2023OCEANGPT} 
        & \small{DAP $\rightarrow$ IT} & \makecell{Vicuna \\ Llama2-chat \\ ChatGLM2} 
        & \xmark & \xmark & LoRA$_{\text{(IT)}}$ 
        & \xmark & HE\\
        \hline

        
        Scientific & K2~\cite{deng2023learning} 
        & \small{DAP $\rightarrow$ SFT} & Llama 
        & \xmark & \xmark & LoRA$_{\text{(SFT)}}$ 
        & \xmark & Perp. | ZS | LLM \\
        \hline

        
        Scientific & MarineGPT~\cite{Zheng2023MarineGPT} 
        & \small{MM-DAP $\rightarrow$ MM-IT} & Llama 
        & \xmark & \xmark & \xmark 
        & \xmark & HE \\
        \hline

        
        Code & CodeGen~\cite{nijkamp2022codegen} 
        & \small{DAP $\rightarrow$ DAP} & CodeGen 
        & \xmark & \xmark & \xmark 
        & \xmark & \makecell{Perp. | ZS}\\
        % \midrule
        % Code & Struct-Rep~\cite{agarwal2024structured} & \small{DAP $\rightarrow$ FT} & CodeT5 CodeGen & \xmark & \xmark & \xmark & \xmark & ZS | FS\\
        \hline


        Code & Comment-Aug~\cite{song2024code} 
        & \small{IT $\rightarrow$ DAP} & \makecell{Llama2 \\ Code~Llama \\ InternLM2}
        & \xmark & \xmark & \xmark 
        & \xmark & ZS \\
        \hline

        
        EventTemporal & EcoNet~\cite{han2021econet}$^1$ 
        & \small{DAP $\rightarrow$ FT} & \makecell{BERT \\ RoBERTa}
        & \xmark & \xmark & \xmark 
        & \xmark & FT \\
        \hline

        
        CommonSense & CALM~\cite{zhou2020pre} 
        & \small{DAP $\rightarrow$ FT} & T5 
        & \xmark & \xmark & \xmark 
        & \xmark & FT \\
        \hline
        
        Multi-Domain & BLADE~\cite{li2024blade} 
        & \small{DAP $\rightarrow$ IT} & BLOOMZ 
        & \xmark & \xmark & \xmark
        & \xmark & ZS 
        \\
        \hline

        Scientific & ClimateGPT~\cite{thulke2024climategpt} 
        & \small{DAP $\rightarrow$ IT $\rightarrow$ RAG} & Llama2
        & \xmark & \xmark & \xmark 
        & \xmark & FS | Ret.
        \\
        \midrule
        \midrule

        %%%%%%%%%%%%%%%%%%%%%%%%%%%%%%%%%%%
        %%% No CL Tech.; CL Eval.
        %%% 6 papers in total.
        %%%%%%%%%%%%%%%%%%%%%%%%%%%%%%%%%%%
        Medical & \cite{guo2023continuous} 
        & \small{DAP $\rightarrow$ FT} & Llama2 
        & \xmark & \xmark & \xmark 
        & \cellcolor{gray1} FS | FT & FS | FT\\
        \hline

        
        % \rowcolor{gray1}
        Financial & \cite{xie2023efficient} 
        & \small{DAP} & Pythia 
        & \xmark & \xmark & \xmark 
        & \cellcolor{gray1} L | FS & L | FS \\
        \hline

        
        % \rowcolor{gray1}
        Scientific & GeoGalactica~\cite{lin2023geogalactica} 
        & \small{DAP $\rightarrow$ G-SFT $\rightarrow$ D-SFT} & GAL 
        & \xmark & \xmark & \xmark 
        & \cellcolor{gray1} ZS & Perp. | ZS | LLM \\
        \hline

        
        % \rowcolor{gray1}
        Code & StarCoder~\cite{li2023starcoder} 
        & \small{DAP} & StarCoder 
        & \xmark & \xmark & \xmark 
        & \cellcolor{gray1} Perp. | ZS | FS &  Perp. | ZS | FS \\
        \hline

        
        % \rowcolor{gray1}
        Code & DeepSeek-Coder~\cite{guo2024deepseekcoder} 
        & \small{DAP} & DeepSeek-LLM 
        & \xmark & \xmark & \xmark 
        & \cellcolor{gray1} ZS | FS & ZS \\
        \hline

        
        % \rowcolor{gray1}
        Multi-Domain & DAPT~\cite{gururangan2020dont} 
        & \small{DAP $\rightarrow$ FT} & RoBERTa 
        & \xmark & \xmark & \xmark 
        & \cellcolor{gray1} Loss & L | FT \\
        \hline
        \hline


        %%%%%%%%%%%%%%%%%%%%%%%%%%%%%%%%%%%
        %%% CL Tech.; No CL Eval.
        %%% 10 papers in total.
        %%%%%%%%%%%%%%%%%%%%%%%%%%%%%%%%%%%
        Financial & WeaverBird~\cite{Xue2023WeaverBird} 
        & \small{DAP} & GLM2 
        & \cellcolor{gray2} \xmark & \cellcolor{gray2} \xmark & \cellcolor{gray2} LoRA 
        & \xmark & HE \\
        \hline

        
        % \rowcolor{gray2}
        Code & IRCoder~\cite{paul2024ircoder} 
        & \small{DAP} & \makecell{StarCoder \\ DeepSeek-Coder \\ Code~Llama}
        & \cellcolor{gray2} \xmark & \cellcolor{gray2} \xmark & \cellcolor{gray2} LoRA 
        & \xmark & ZS \\
        \hline

        
        % \rowcolor{gray2}
        Code & Code~Llama~\cite{rozière2024code} 
        & \makecell{\small{DAP $\rightarrow$ LC-FT $\rightarrow$ IT} \\ \small{DAP $\rightarrow$ DAP $\rightarrow$ LC-FT}} & Llama2
        & \cellcolor{gray2} Replay & \cellcolor{gray2} \xmark & \cellcolor{gray2} \xmark 
        & \xmark & Perp. | ZS \\
        \hline


        % \rowcolor{gray2}
        Legal & SaulLM~\cite{colombo2024saullm7b} 
        & \small{DAP $\rightarrow$ U-IT} & Mistral 
        & \cellcolor{gray2} Replay & \cellcolor{gray2} \xmark & \cellcolor{gray2} \xmark 
        & \xmark & Perp. | ZS \\
        \hline

        
        % \rowcolor{gray2}
        Medical & PMC-Llama~\cite{wu2023pmc} 
        & \small{DAP $\rightarrow$ IT} & Llama 
        & \cellcolor{gray2} Replay & \cellcolor{gray2} \xmark & \cellcolor{gray2} \xmark 
        & \xmark & ZS | FT \\
        \hline

        
        % \rowcolor{gray2}
        Scientific & Llema~\cite{Azerbayev2023LLEMMA} 
        & \small{DAP} & Code~Llama 
        & \cellcolor{gray2} Replay & \cellcolor{gray2} \xmark & \cellcolor{gray2} \xmark 
        & \xmark & Perp. | FS \\
        \hline

        
        % \rowcolor{gray2}
        Multi-Domain & DAS~\cite{ke2022continual-pre} 
        & \small{[DAP]$_n$} & RoBERTa 
        & \cellcolor{gray2} DER++$^\text{\club}$ & \cellcolor{gray2} EWC$^\text{\club}$\qquad HAT$^\text{\club}$\qquad Soft-Masking & \cellcolor{gray2} Adapter$^\text{\club}$\qquad DEMix$^\text{\club}$ 
        & \xmark & FT \\
        \hline

        Medical & Hippocrates~\cite{acikgoz2024hippocrates} 
        & \small{DAP $\rightarrow$ IT $\rightarrow$ MA} & \makecell{Llama2 \\ Mistral}
        & \cellcolor{gray2} \xmark & \cellcolor{gray2} \xmark & \cellcolor{gray2} LoRA
        & \xmark & ZS | FS
        \\
        \hline

        Language & Sailor~\cite{dou2024sailor} 
        & \small{DAP} & Qwen1.5 
        & \cellcolor{gray2} Replay & \cellcolor{gray2} \xmark & \cellcolor{gray2} \xmark
        & \xmark & ZS  
        \\

        
        \midrule
        \midrule

        
        %%%%%%%%%%%%%%%%%%%%%%%%%%%%%%%%%%%
        %%% CL Tech.; CL Eval.
        %%% 12 papers in total.
        %%%%%%%%%%%%%%%%%%%%%%%%%%%%%%%%%%%
        Code \& Math & Llama Pro~\cite{wu2024llama} 
        & \small{DAP $\rightarrow$ U-SFT} & Llama2 
        & \cellcolor{gray3} \xmark & \cellcolor{gray3} \xmark & \cellcolor{gray3} Block~Exp. LoRA$^{\text{\club}}$ 
        & \cellcolor{gray3} ZS | FS & Perp. | ZS | FS \\
        \hline

        
        % \rowcolor{gray3}
        Medical & AF Adapter~\cite{yan2023af} 
        & \small{DAP $\rightarrow$ FT} & RoBERTa 
        & \cellcolor{gray3} \xmark & \cellcolor{gray3} \xmark & \cellcolor{gray3} Layer~Exp. LoRA$^{\text{\club}}$ 
        & \cellcolor{gray3} Acc. & L | FT\\
        \hline

        
        % \rowcolor{gray3}
        Medical & \cite{rongali2021continual} 
        & \small{DAP $\rightarrow$ FT} & \makecell{BERT \\ RoBERTa \\ DistilBERT}
        & \cellcolor{gray3} Replay$^{\text{\club}}$ GEM$^{\text{\club}}$ & \cellcolor{gray3} L2~Reg.$^{\text{\club}}$ EWC$^{\text{\club}}$ & \cellcolor{gray3} \xmark 
        & \cellcolor{gray3} L | FT & L | FT\\
        \hline

        
        % \rowcolor{gray3}
        Medical & HuatuoGPT-II~\cite{Chen2023HuatuoGPTII} 
        & \small{DAP + U-SFT} & Baichuan2 
        & \cellcolor{gray3} Replay & \cellcolor{gray3} \xmark & \cellcolor{gray3} \xmark 
        & \cellcolor{gray3} ZS & ZS | HE\\
        \hline

        
        % \rowcolor{gray3}
        Financial & XuanYuan 2.0~\cite{Zhang2023xuanyuan} 
        & \small{DAP + SFT} & BLOOM 
        & \cellcolor{gray3} Replay & \cellcolor{gray3} \xmark & \cellcolor{gray3} \xmark 
        & \cellcolor{gray3} HE & HE\\
        \hline

        
        % \rowcolor{gray3}
        Scientific & PLlama~\cite{Yang2023PLLaMa} 
        & \small{DAP $\rightarrow$ IT} & GAL 
        & \cellcolor{gray3} Replay & \cellcolor{gray3} \xmark & \cellcolor{gray3} \xmark 
        & \cellcolor{gray3} L &  L | ZS \\
        \hline

        
        % \rowcolor{gray3}
        E-Commerce & EcomGPT-CT~\cite{ma2023ecomgptct} 
        & \small{DAP $\rightarrow$ SFT} & BLOOM 
        & \cellcolor{gray3} Replay & \cellcolor{gray3} \xmark & \cellcolor{gray3} \xmark 
        & \cellcolor{gray3} ZS | FS & ZS | FS \\
        \hline

        
        % \rowcolor{gray3}
        Legal & Layer~Llama~\cite{huang2023lawyer} 
        & \small{DAP $\rightarrow$ G-IT $\rightarrow$ D-IT} & Llama 
        & \cellcolor{gray3} Replay & \cellcolor{gray3} \xmark & \cellcolor{gray3} \xmark 
        & \cellcolor{gray3} ZS & ZS\\
        \hline

        
        % \rowcolor{gray3}
        Multi-Domain & AdaptLLM~\cite{cheng2024adapting} 
        & \small{DAP} & Llama 
        & \cellcolor{gray3} Replay & \cellcolor{gray3} \xmark & \cellcolor{gray3} \xmark 
        & \cellcolor{gray3} ZS & ZS | FT\\
        \hline

        Language & Swallow~\cite{fujii2024continual} 
        & \small{DAP} & Llama2
        & \cellcolor{gray3} Replay & \cellcolor{gray3} \xmark  & \cellcolor{gray3} \xmark 
        & \cellcolor{gray3} FS & FS 
        \\
        \hline

        Financial & \cite{takahashi2024pretraining} 
        & \small{DAP} & Llama2 
        & \cellcolor{gray3} Replay & \cellcolor{gray3} \xmark & \cellcolor{gray3} \xmark
        & \cellcolor{gray3} Loss | ZS & Loss | ZS | FS | RAG
        \\
        \hline

        Medical & Me-Llama~\cite{xie2024me} 
        & \small{DAP $\rightarrow$ IT} & Llama2
        & \cellcolor{gray3} Replay & \cellcolor{gray3} \xmark & \cellcolor{gray3} \xmark
        & \cellcolor{gray3} ZS | FS & ZS | FS | FT
        \\
        \hline

        Language & Aurora-M~\cite{nakamura2024aurora} 
        & \small{DAP $\rightarrow$ IT} & StarCoder 
        & \cellcolor{gray3} Replay & \cellcolor{gray3} \xmark  & \cellcolor{gray3} \xmark
        & \cellcolor{gray3} ZS & ZS | FS | HE
        \\
        \hline
        
    	\bottomrule[0.15em]
	   \end{tabular}
	}
\end{table*}
% \footnotetext[1]{In this paper, only qualitative demonstration has been shown.}
% \footnotetext[2]{In DAPT~\cite{gururangan2020dont} and AdaptLLM~\cite{cheng2024adapting}, there are 4 domains considered in parallel, not sequentially in this work.}

\textbf{Financial Domain.}\quad
% Similar to the medical domain, LLMs hold immense potential for enhancing financial communication, decision-making processes, and risk assessment for both traders and ordinary individuals~\cite{shah2023zero,Yang2023InvestLM,Wang2023FinGPT,li2023cfgpt}. 
% However, the financial domain entails high stakes, with even minor errors bearing significant consequences. Thus, integrating LLMs into financial workflows requires heightened caution to avoid inaccuracies or misunderstandings that could lead to substantial financial losses or the dissemination of misleading information. 
A gap persists between general-purpose LLMs and existing domain-specific smaller-scale LLMs~\cite{araci2019finbert,Wu2023BloombergGPT}, underscoring the urgent need for more powerful financial-domain experts through the integration of LLMs. Notably, DAP techniques have emerged as crucial tools for tailoring LLMs to the intricacies of the financial domain while mitigating the negative effects of abrupt domain shifts from general to finance~\cite{Lu2023BBTFin,li2023cfgpt,xie2023efficient,Xue2023WeaverBird,Zhang2023xuanyuan}.

BBT-Fin~\cite{Lu2023BBTFin} collects a Chinese financial DAP dataset comprising 80 billion tokens sourced from corporate reports, analyst reports, social media, and financial news. In addition to the conventional masked language modeling (MLM) training objective, BBT-Fin further incorporates triplet masking and span masking techniques during DAP. 
% This knowledge enhancement pre-training entails: (i)~actively selecting sentences containing specific knowledge triplets \emph{(head entity, relation, tail entity)}, masking one of them, and~(ii) simultaneously masking 15\% of a random-length span.
CFGPT~\cite{li2023cfgpt} creates CFData, a financial dataset for DAP and SFT, comprising 141 billion tokens. 
% The dataset includes corporate prospectuses, announcements, research reports, social media content, financial news, and Wikipedia articles. 
During DAP, CFGPT does not employ CL techniques but utilizes QLoRA~\cite{dettmers2023qlora} for preventing overfitting to downstream data and balancing general response ability and domain-specific ability during SFT. These two methods are typical domain-specific LLMs focusing solely on adaptation to target domains without explicit CL measures or evaluation of vertical forgetting.


In \cite{xie2023efficient}, the authors aim to enhance the data efficiency of DAP. 
% They propose two data selection techniques: (i)~efficient task-similar~(ETS) and (ii)~efficient task-agnostic~(ETA) domain-adaptive pre-training. 
When the downstream tasks' data distribution $\gT$ are known, based on the generalization bound~\cite{ben2010theory,ganin2016domain,shi2024unified}, the authors propose to sample the subset of DAP data whose distribution $\gD$ is similar to the downstream task's data, i.e., $d_{\gH\Delta \gH}(\gD, \gT)$ is low. 
% where $\gD$ is the distribution of the sampled DAP dataset, and $d_{\gH\Delta \gH}$ is the $\gH\Delta \gH$ divergence that measures the distributional discrepancy between two distributions based on a hypothesis set $\gH$. 
When the downstream data distribution is unknown, the authors suggest ensuring \emph{novelty} and \emph{diversity} in the sampled corpus for DAP. 
% Here, high perplexity of an LLM on a sentence indicates high novelty, while high entropy of part-of-speech (POS) tags on a sentence indicates high diversity~\cite{bengio2009curriculum,zewdu2022part}.
This approach significantly enhances DAP efficiency: it utilizes only 10\% of the originally collected data yet outperforms models trained on the entire DAP dataset, underscoring the importance of data quality over quantity. 
% Suppose the source domain data distribution is $\gD$ and the task data distribution is $\gT$. The generalization error $\epsilon_{\gT}(h)$ of the hypothesis $h$ on the target task distribution $\gT$ can be given as
% \begin{align}
%     \epsilon_{\gT}(h) &\leq \epsilon_{\gD}(h) + \frac{1}{2}d_{\gH\Delta \gH}(\gD, \gT) + \gC,
% \end{align}
% where $\epsilon$ is the 0-1 error function for binary classification, $h$ represents the model, $d_{\gH\Delta \gH}$ is the $\gH\Delta \gH$ divergence that measures the distributional discrepancy between two distributions based on a hypothesis set $\gH$ used for discriminating different data distributions~\cite{ben2010theory}, and $\gC$ is a constant.
% Therefore, based on the theory, finding a set of examples from the DAP corpus that are similar to the downstream task's data, i.e., $d_{\gH\Delta \gH}(\gD, \gT)$ is low, should help the final-stage adaptation to the tasks. 
% In the context of large language models (LLMs), \cite{xie2023efficient} suggests ensuring \emph{novelty} and \emph{diversity} in the sampled corpus for DAP. Here, high perplexity of an LLM on a sentence indicates high novelty, while high entropy of part-of-speech (POS) tags on a sentence indicates high diversity~\cite{bengio2009curriculum,zewdu2022part}. This approach significantly enhances DAP efficiency: it utilizes only 10\% of the originally collected data yet outperforms models trained on the entire dataset, underscoring the importance of data quality over quantity in DAP. Furthermore, with the reduced number of examples, the authors do not observe any signs of forgetting regarding the model's capacity in open-domain scenarios.
WeaverBird~\cite{Xue2023WeaverBird} introduces an intelligent finance dialogue system, where the encoder is trained on Chinese and English financial documents, alongside expert-annotated financial query-response pairs, using LoRA~\cite{hu2022lora}. Xuanyuan 2.0~\cite{Zhang2023xuanyuan}, akin to HuatuoGPT-II~\cite{Chen2023HuatuoGPTII}, proposes the technique of hybrid-tuning, which fuses the stages of DAP and SFT into one, general-domain data and financial-domain data into one. Notably, the distribution of data in hybrid-tuning is unconventional: financial DAP data comprises only a small portion of 13\%. This prompts a pertinent question in line with the investigation on efficient DAP in \cite{xie2023efficient}: Is a large DAP dataset necessary for developing a domain-specific LLM?

% WeaverBird~\cite{Xue2023WeaverBird} proposes an intelligent dialogue system for finance, in which the encoder is trained on the Chinese and English financial documents as well as expert-annotated financial query-response pairs, via LoRA~\cite{hu2022lora}. Xuanyuan~2.0~\cite{Zhang2023xuanyuan}, similar to HuatuoGPT-II~\cite{Chen2023HuatuoGPTII}, proposes the technique of hybrid-tuning, which fuses the stage of DAP and SFT into one, general-domain data and financial-domain data into one. The proportions of the data in the hybrid-tuning is non-conventional, as the DAP data used to embed the fundamental domain-specific knowledge in the model, is comparatively small-scale: financial pre-training data 13\%, general pre-training data 20\%, and instruction tuning data 67\%. At the same time, in the instruction tuning data, financial domain only takes up about 30\%. This leaves us an intriguing question that resonates with the motivation of the efficient DAP as studied in \cite{xie2023efficient}: is the large volume of the DAP data necessary when developing a domain-specific LLM?



\textbf{Scientific Domain.}\quad
Vertical scientific LLMs span many subjects~\cite{Zhang2024SciGLM,Nguyen2023AstroLLaMA,Azerbayev2023LLEMMA,luo2023wizardmath,lin2023geogalactica,Bi2023OCEANGPT,Zheng2023MarineGPT}.
However, among all the studies listed above, only a small fraction of them adopt the technique of DAP. 
OceanGPT~\cite{Bi2023OCEANGPT} is the first LLM tailored specifically for the ocean domain. It performs DAP on a raw corpus of ocean science literature, prioritizing recent research and historically significant works.
K2~\cite{deng2023learning} pioneers the development of a foundational language model tailored specifically for geoscience. It aggregates geoscience open access literature and Earth science-related Wikipedia pages for DAP. Following this, it undergoes multi-task instruction tuning utilizing LoRA~\cite{hu2022lora} on both a general instruction tuning dataset and the GeoSignal benchmark introduced within the K2 framework.
AstroLlama~\cite{Nguyen2023AstroLLaMA} gathers abstracts solely from astronomy papers on arXiv and proceeds pre-training. It observes an improved perplexity on the domain of scholarly astronomy, without providing more quantitative evaluation. 
MarineGPT~\cite{Zheng2023MarineGPT} is a multi-modal LLM designed specifically for the marine domain. 
% It utilizes ViT~\cite{dosovitskiy2020image} as the visual encoder and Llama~\cite{touvron2023llama} as the decoder for response generation. 
During DAP, MarineGPT incorporates 5 million marine image-text pairs to imbue domain knowledge. This involves training a Q-Former~\cite{li2023blip2} between the frozen visual and text decoder~\cite{dosovitskiy2020image,touvron2023llama}. 
% Consequently, it achieves enhanced visual-text alignment within the marine domain.
% The four methods introduced above lack further validation regarding the necessity of DAP deployment. They seem to adopt DAP merely out of convention, likely for the convenience of utilizing domain-specific corpora as pre-training data.

Another branch of methods proactively integrate in the replay of the general-domain data to mitigate vertical forgetting.
GeoGalactica~\cite{lin2023geogalactica} introduces a series of LLMs tailored for geoscience. In the DAP phase, besides the 52-billion-token geoscience corpus, Arxiv papers and Codedata are incorporated, with a mixing ratio of 8:1:1. The authors believe that the inclusion of the Codedata during the model's pre-training can significantly boost the reasoning ability of the LLMs. 
Although GeoGalactica pinpoints challenges of DAP, including overfitting, catastrophic forgetting,  maintaining the training stability, and convergence speed, it does not further provide empirical evidence supporting the inclusion of the Codedata, or deploying specific measures to address the challenges proposed above. 
Llemma~\cite{Azerbayev2023LLEMMA} focuses on mathematics, initialized from Code~Llama~\cite{rozière2024code}, and undergoes DAP on a blend of the 55-billion-token mathematical pre-training dataset and general domain data at the ratio of 19:1. 
In contrast, PLlama~\cite{Yang2023PLLaMa}, designed for plant science, mixes domain-specific and general-domain data at the ratio of 9:1.






\textbf{Code Domain.}\quad
The development of LLMs for automatic code filling, debugging, and generation holds significant practical importance~\cite{moradidakhel2023github,sun2024survey}. These advancements cover various frameworks, including encoder-only~\cite{moradidakhel2023github}, encoder-decoder~\cite{wang2021codet5,wang2023codet5plus}, and decoder-only~\cite{nijkamp2022codegen,lozhkov2024starcoder,guo2024deepseekcoder}.
There is a growing trend towards decoder-only architectures~\cite{sun2024survey}, leveraging models pre-trained on general natural language like Llama~\cite{touvron2023llama,touvron2023llama2}. Consequently, there is a shift in the training objective from utilizing code structures to simpler tasks like next token prediction and infilling.


From the perspective of CL, the code domain presents unique advantages and challenges for DAP, compared to other domains. On one hand, its hierarchical structure (\emph{general domain corpus $\rightarrow$ multi-language code $\rightarrow$ specific programming language}) provides an ideal training pipeline for DAPs~\cite{rozière2024code}, offering potential for more efficient training strategies. On the other hand, programming languages adhere to strict grammars, unlike the fuzzy and context-dependent natural language. Consequently, language models should ideally leverage these structures through tailored designs, and adopting the same training objectives as for natural languages may yield sub-optimal results. Therefore, many existing studies omit DAP~\cite{wang2021codet5,wang2023codet5plus,luo2023wizardcoder}.
In the following section, we will introduce existing code LLMs that employ DAP before the final downstream tasks, discussing both their common attributes and unique characteristics.

Representing a series of notable works that focus solely on adaptation to target domains, 
CodeGen~\cite{nijkamp2022codegen} comprises a suite of LLMs designed for natural language~(CodeGen-NL), multi-lingual programming languages~(CodeGen-Multi), and mono-lingual programming languages~(CodeGen-Mono). These models are trained sequentially, with each subsequent model initialized from the previous one trained on more general-domain data. 
% Specifically, CodeGen-NL is trained on Pile~\cite{gao2020pile}, while CodeGen-Multi further trains on a subset of the BigQuery dataset\footnote{Link to BigQuery: \href{https://cloud.google.com/bigquery/public-data}{https://cloud.google.com/bigquery/public-data}.}, which includes source code from six popular programming languages (C, C++, Go, Java, JavaScript, and Python). Subsequently, CodeGen-Mono is trained on BigPython, a large-scale corpus of Python code. Throughout the training process, consistent pre-training objectives are employed, focusing on next token prediction.
Comment-Aug~\cite{song2024code} addresses the challenge of aligning programming languages with natural languages (PL-NL alignment) by performing DAP on the code augmented with generated additional comments. 
% Recognizing that denser comments in existing code could enhance the model's generation capabilities, Comment-Aug proposes a self-augmentation strategy. Initially, it enhances LLMs with comment generation ability through instruction tuning. Then it augments sparsely commented code by generating additional comments, followed by further pre-training on the enriched code data.
StarCoder~\cite{li2023starcoder} introduces two models: StarCoderBase and StarCoder. StarCoderBase is initially trained on a mixed dataset comprising various programming languages without significant reweighting on the data. Subsequently, StarCoderBase undergoes further fine-tuning on additional 35 billion tokens of Python code, resulting in the development of StarCoder.
DeepSeek-Coder-v1.5~\cite{guo2024deepseekcoder} originates from DeepSeek-LLM~\cite{deepseekai2024deepseek} and undergoes pre-training on 2 trillion tokens, comprising 87\% source code, 10\% English code-related natural language, and 3\% Chinese natural language corpus. 
% Unlike Deepseek-Coder, which employs both next token prediction and fill-in-the-middle objectives, DeepSeek-Coder-v1.5 focuses solely on next token prediction during the stage of DAP.
% Despite this change, i
Initialization from a general-domain LLM results in improved performance across various tasks, including natural language and mathematical reasoning, with minimal performance degradation on coding tasks, which underscores the efficacy of DAP.


As the only work that utilizes the general data replay to mitigate vertical forgetting in the code domain,
Code~Llama~\cite{rozière2024code} introduces a sophisticated training framework tailored for various coding tasks and model sizes. Initialized from Llama~2 weights, these models undergo DAP on a dataset composed of deduplicated public code, discussions about code, and a subset of natural language data. This mix of natural language data serves as a form of pseudo-replay to maintain the models' proficiency in understanding natural language. 
% During DAP, Code~Llama optimizes two objectives: autoregressive next token prediction and code infilling prediction (except for the 34B model, which excludes infilling). Subsequently, Code~Llama undergoes further refinement through long-context fine-tuning (LC-FT) to enhance its repository-level understanding. Building upon Code~Llama, Code~Llama-Instruct undergoes additional instruction-tuning, while a Python-specific variant of Code~Llama undergoes additional domain-adaptive pre-training on Python language datasets before LC-FT.
Besides replay, architecture expansion has proven effective in acquiring robust coding abilities and preventing vertical forgetting simultaneously.
IRCoder~\cite{paul2024ircoder} utilizes compiler intermediate representations to enhance the multilingual transferability of Code LLMs. By conducting DAP on code grounded in intermediate representations with LoRA~\cite{hu2021lora}, IRCoder achieves superior multilingual programming instruction following, enhanced multilingual code understanding, and increased robustness to prompt perturbations.
Llama~Pro~\cite{wu2024llama} undergoes DAP on a combination of code and math data. It expands the original Llama2 architecture by dynamically adding multiple identity copies of the transformer blocks. These added blocks initially preserves the original functionality, and will be tuned for DAP. The proposed expansion method is shown to be more resilient against vertical forgetting compared to other parameter-efficient tuning methods like LoRA.

The three aforementioned studies highlight the importance of DAP for code LLMs. However, it is crucial to note that the problem definition and conventional architectures of existing Code LLMs may present challenges of compatibility for DAP deployment, and need to be addressed in the future.


\textbf{Other Domains.}\quad
ECONET~\cite{han2021econet} enhances the model's ability to reason about event temporal relations through a dedicated DAP phase. Temporal and event indicators are masked out, and a contrastive loss is applied to the recovered masked tokens. Results demonstrate that incorporating this DAP stage significantly improves performance on final tasks compared to direct fine-tuning.
Concept-Aware Language Model~(CALM)~\cite{zhou2020pre} introduces a data-efficient DAP approach for enhancing the concept-centric commonsense reasoning ability of LLMs. It incorporates both generative and discriminative commonsense reasoning tasks specifically tailored for concept-centric reasoning tasks. Consequently, even a small number of data examples for DAP can lead to notable improvements for downstream tasks.

% EcomGPT-CT~\cite{ma2023ecomgptct} is tailored for the E-commerce domain.
% Given that E-commerce data often exhibits a semi-structured format, stored in tables or databases, 
Aurora-M~\cite{nakamura2024aurora} and Swallow~\cite{fujii2024continual} adopt the simple replay strategy that mixes in a small portion of general data during DAP for their multi-lingual ability.
Furthermore, Sailor~\cite{dou2024sailor} studies the optimal strategy of data mixing for DAP, balancing the general knowledge and capacity of different languages. 
EcomGPT-CT~\cite{ma2023ecomgptct} employs a data mixing strategy for DAP which transforms semi-structured E-commerce data into a set of nodes and edges, samples a cluster of nodes, and then extracts and concatenates them into a training example. It combines the general-domain corpus with E-commerce data at a ratio of 2:1, which is significantly lower than the common setting adopted by other works.


% Notably, AdaptLLM~\cite{cheng2024adapting} challenges the data efficiency of conventional DAP by adopting a novel approach inspired by human learning patterns in reading comprehension. 
Notably, there are some papers studying other effective ways of DAP. 
AdaptLLM~\cite{cheng2024adapting} transforms raw corpora into \emph{(raw text, question, answer)} format, creating intrinsic reading comprehension tasks. 
% The model is trained on reading tasks (next token prediction on the raw text) and comprehension tasks (question-answering based on the corpora). To ensure instruction diversity beyond predefined templates, it integrates substantial amounts of general instruction tuning data at ratios ranging from 1:1 to 1:2 across various domains. Compared to traditional DAP approaches that utilize raw domain-specific corpora as-is, 
AdaptLLM demonstrates superior domain-specific knowledge adaptation and minimal vertical forgetting, thereby challenging the data efficiency of conventional DAP. 
Tag-LLM~\cite{shen2024tag} re-purposes the general-domain LLM into domain-specific one by multi-stage training of domain tags and function tags, without modifying the base LLM's weights and thereby mitigates forgetting.




\subsection{Continual Fine-Tuning~(CFT)}
\label{sec:cft}
% General Background of CFT, from a traditional view of continual learning on natural language processing. 
% \haizhou{
% Structure of the chapter of CFT.
% \begin{itemize}
%     \item Intro to CFT
%     \begin{itemize}
%         \item what CFT means: application-level
%         \item traditional continual learning on natural language processing can be seen as CFT
%         \item four specific topics of CFT in the context of large language models: including CIT, CMR, CMA, CFT for MLLMs.
%     \end{itemize}
%     \item General Observations about CFT
%     \begin{itemize}
%         \item Compared to CPT and DAP, continual learning techniques are more commonly deployed in CFT, as this is an explicit topic studied across the community. 
%         \item In the realm of LLMs, focus of the continual learning community has shifted from class-incremental learning to domain-incremental and task-incremental learning.
%     \end{itemize}
% \end{itemize}
% }

\textbf{Background of Continual Fine-Tuning~(CFT).}\quad
Continual Fine-Tuning~(CFT) lies at the bottom layer of the vertical continuity, where models are trained on successive homogeneous tasks drawn from an evolving data distribution. As the service-oriented layer of LLM, it does not require consideration of further adaptation to another downstream tasks, simplifying optimization objectives to a great extent: better adaptation and less forgetting\footnote{We direct interested readers to additional survey literature on the topic of general CFT~\cite{biesialska2020continual,ke2023continual}.}. 
% In essence, the goal is to improve the overall performance. The challenge of CFT has been extensively explored in the continual learning community. Essentially, all existing continual learning literature can be seen as a variant of CFT: models are either randomly initialized or initialized from pre-trained weights and undergo CFT thereafter. 
% In the realm where CL intersects with NLP, we will only briefly outline the most notable works in this domain in \ref{sec:cft-general}, and direct interested readers to additional survey literature on this topic~\cite{biesialska2020continual,ke2023continual}.
In the era of LLMs, new computational paradigms in CFT have emerged and attracted significant attention within the research community. 
These topics include (i)~Continual Instruction Tuning~(CIT)~\cite{zhang2023citb}, (ii)~Continual Model Refinement~(CMR)~\cite{hartvigsen2023aging}, (iii)~Continual Model Alignment~(CMA)~\cite{lin2024mitigating,zhangcppo}, and (iv)~Continual Learning for Multimodal Language Models~(CMLLMs)~\cite{he2023continual,ni2023continual}.
% \begin{itemize}
%     \item \textbf{Continual Instruction Tuning (CIT)}, where models must generalize to new tasks encoded in instructions, requiring semantic understanding~\cite{zhang2023citb} (\Secref{sec:cft-cit}). 
%     \item \textbf{Continual Model Refinement (CMR)}, where fine-grained, possibly example-level solutions are required, differing from task-level approaches~\cite{hartvigsen2023aging} (\Secref{sec:cft-cmr}).
%     \item \textbf{Continual Model Alignment (CMA)}, which aligns models with evolving human preferences, challenging due to subjective nature and lack of clear task boundaries~\cite{lin2024mitigating,zhangcppo} (\Secref{sec:cft-cma}).
%     \item \textbf{Continual Learning for Multimodal Language Models (CMLLMs)}, where addressing the composite architectural design and preventing catastrophic forgetting are key challenges~\cite{he2023continual,ni2023continual} (\Secref{sec:cft-cmllm}).
% \end{itemize}
% ,  , and .
% While all these fall under the umbrella of CFT, each presents distinct features and challenges. In CIT, . CMR demands fine-grained, possibly example-level, operations for model refinement, differing from task-based approaches~\cite{hartvigsen2023aging}. CMA aligns models with evolving human preferences, challenging due to subjective nature and lack of clear task boundaries~\cite{lin2024mitigating,zhangcppo}. In CMLLMs,  Detailed exploration of these sub-categories follows in subsequent chapters.
% General intro to the table.
We summarize existing studies on CFT in \Tabref{tab:cft}, categorizing studies into sub-categories as listed above. The table includes details on incremental learning types (X-IL), LLM architecture, and employed CL techniques and evaluation metrics. After discussing general observations on CFT in \Secref{sec:cft-obs}, we will delve into each sub-category in detail.



% Table for Continual Fine-Tuning (V1)
% \begin{table*}[htbp]
%     \centering
%     \caption{
%     \textbf{Summary of the existing studies on Continual Fine-Tuning LLMs,} where the papers are organized in four main categories based on what downstream tasks they are designed to tackle: (i)~general text classification, generation, and labeling; (ii)~Instruction Tuning; (iii)~Model Refinement; (iv)~Model Alignment; (v)~Multi-Modal LLMs.
%     }
%     \label{tab:pre-training-big}
%     \resizebox{1\linewidth}{!}{%
% \begin{tabular}{ccCCC CCccc c}
% 	\toprule[0.15em]
%     \multirow{2}{*}[-0.25em]{\textbf{Method}} & 
%     \multirow{2}{*}[-0.25em]{\textbf{Tasks}} & 
%     \multirow{2}{*}[-0.25em]{\textbf{X-IL}} & 
%     \multirow{2}{*}[-0.25em]{\textbf{LLM Arch.}} & 
%     \multicolumn{4}{c}{\textbf{{Continual Learning Tech.}}} & 
%     \multicolumn{3}{c}{\textbf{{Continual Learning Eval.}}} \\
%     \cmidrule(lr){5-8}\cmidrule(lr){9-11}
%     & & & & \emph{Rehearsal} & \emph{Param. Reg.} & \emph{Arch. Exp.} & \emph{Others} & \emph{Avg. Acc.} & \emph{Bwd. Trans.} & \emph{Fwd. Trans.} \\
% 	%  &
%     \midrule
%     \midrule
%     CTR~\cite{ke2021achieve} & CLS & DIL \& CIL & BERT & \xmark & \xmark & Adapter & \xmark & \cmark & \cmark & \cmark \\
%     \midrule
%     \cite{tao2022can} & CLS | GEN & TIL & BERT & S-Replay & \xmark & \xmark & \xmark & \club & \club & \club \\
%     \midrule
%     CIRCLE~\cite{wei2022circle} & GEN & DIL & T5 & Replay & EWC & Prompt & \xmark & \cmark & \cmark & \cmark \\
%     \midrule
%     ConPET~\cite{song2023conpet} & ??? & DIL & Llama & Replay & \xmark & LoRA & \xmark & \cmark & \cmark & \cmark \\
%     \midrule
%     \cite{bai2023enhancing} & CLS & DIL \& CIL & BERT & \xmark & \xmark & \xmark & G-Prompt & \cmark & \cmark & \xmark \\
%     \midrule
%     \cite{luo2023investigating} & CLS & TIL & DistilBERT ALBERT RoBERTa & ER \& DER \&~LwF & \xmark & \xmark & \xmark & \club & \club & \xmark \\
%     \midrule
%     SEQ$^*$~\cite{zheng2023learn} & CLS & TIL \& CIL & BERT\quad Pythia\quad GPT2 & \xmark & P-Freeze & \xmark & Tricks for Classfiers & \xmark & \cmark & \xmark \\
%     \midrule
%     LFPT5~\cite{qin2021lfpt5} & LAB | CLS | GEN & DIL & T5 & P-Replay & \xmark & \xmark & \xmark & \cmark & \cmark & \xmark\\
%     \midrule
%     \cite{weyssow2023usage} & GEN & DIL & RoBERTa GPT2 & Replay & EWC \& SI \& RWalk & \xmark & \xmark & \cmark & \cmark & \xmark \\
%     \midrule
%     \cite{winata2023overcoming} & LAB | CLS & DIL & XLM-R & \xmark & \xmark & \xmark & LR Scheduling & \cmark & \cmark & \cmark \\
%     \midrule
%     C3~\cite{chen2024parameterizing} & GEN & TIL & T5 & KD & \xmark & Prompt Tuning & \xmark &\cmark & \cmark & \xmark \\
%     \midrule
%     \midrule
%     \rowcolor{gray2}
%     CMR~\cite{lin2022continual} & CMR & DIL & BART & ER \& MIR \& MLR & L2 \& EWC & \xmark & \xmark & \cmark & \cmark & \cmark \\
%     \midrule
%     \rowcolor{gray2}
%     GRACE~\cite{hartvigsen2023aging} & CMR & DIL & T5\quad \quad~BERT\quad \quad~GPT2 & \xmark & \xmark & Adapter & \xmark & \cmark & \cmark & \xmark \\
%     \midrule
%     \rowcolor{gray2}
%     WilKE~\cite{hu2024wilke} & CMR & DIL & GPT2\quad GPT-J & \xmark & \xmark & Adaptor & \xmark & \cmark & \cmark & \cmark \\
%     \midrule
%     \rowcolor{gray2}
%     Larimar~\cite{das2024larimar} & CMR & DIL & BERT\quad GPT-J & \xmark & \xmark & \xmark & Kanerva Memory & \cmark & \cmark & \cmark \\
%     \midrule
%     \rowcolor{gray2}
%     MELO~\cite{yu2023melo} & CMR & DIL & BERT\quad GPT2\quad \qquad~T5 & \xmark & \xmark & LoRA & \xmark & \cmark & \cmark & \cmark\\
%     \midrule
%     \rowcolor{gray2}
%     CME~\cite{li2023continual} & CMR & DIL & BERT & Replay & \xmark & \xmark & Inner-Prod. Reg. & \cmark & \cmark & \cmark \\
%     \midrule
%     \midrule
%     \rowcolor{cream}
%     CT0~\cite{scialom2022fine} & CIT & D(T)IL & T0 & S-Replay & \xmark & \xmark & \xmark & \cmark & \cmark & \cmark \\
%     \midrule
%     \rowcolor{cream}
%     \cite{wang2023trace} & CIT & TIL & LLaMA Vicuna Baichuan & Replay & \xmark & \xmark & \xmark & \cmark & \cmark & \cmark \\
%     \midrule
%     \rowcolor{cream}
%     \cite{mok2023large} & CIT & TIL & BART & Replay & \xmark & \xmark & \xmark & \cmark & \cmark & \cmark \\
%     \midrule
%     \rowcolor{cream}
%     \cite{zhang2023citb} & CIT & TIL & T5 & Replay \& AGEM &  L2 \& EWC & AdapterCL & Init \& FT-init \& FT-no-init \& Multi & \cmark & \cmark & \cmark \\
%     \midrule
%     \rowcolor{cream}
% %    \cite{chen2024coin} & CIT & TIL & LLaVA & \xmark & \xmark & \xmark & \xmark & \cmark & \cmark & \cmark \\
% %    \midrule
%  %   \rowcolor{cream}
%     \cite{huang2024mitigating} & CIT & TIL & LLaMA Alpaca & RandSel \& KMeansSel & \xmark & \xmark & Multi-task Learning \& Non-rehearsal & \cmark & \cmark & \cmark \\
%     \midrule
%     \rowcolor{cream}
%     \cite{he2024dont} & CIT & DTIL & LLaMA Baichuan & DYNAINST \& PCLL \& DCL & L2 \& EWC & DARE \& LM-Cocktail & KPIG & \cmark & \cmark & \cmark \\
%     \midrule
%     \rowcolor{cream}
%     % \cite{scialom2022fine} & CIT & TIL & T0 & Replay & \xmark & \xmark & \xmark & \cmark & \cmark & \cmark \\
%     % \midrule
%     \rowcolor{cream}
%     \cite{yin2022contintin} & CIT & TIL & BART & Replay & \xmark & \xmark & InstructionSpeak & \cmark & \cmark & \cmark \\
%     \midrule
%     \rowcolor{cream}
%     \cite{wang2023orthogonal} & CIT & TIL & LLaMA Alpaca & \xmark  & \xmark & \xmark & O-LoRA & \cmark & \cmark & \cmark \\
%     \midrule
%     \rowcolor{cream}
%     \cite{zhao2024sapt} & CIT & TIL & T5 LLaMA & \xmark  & \xmark & \xmark & SAPT & \cmark & \cmark & \cmark \\
%     %%%% Continual Model Alignment
%    % \cite{zhao2024sapt} & CIT & TIL & T0 & S-Replay & \xmark & \xmark & \xmark & \cmark & \cmark & \cmark \\
%    %%%MLLM
%     \midrule
%     \midrule
%     \rowcolor{green1}    
%     Fwd-Prompt~\cite{zheng2024antiforgetting} & MM-CIT & TIL & InstructBLIP/BLIP2 & \xmark & \xmark & learns new projection layers & Fwd-Prompt & \cmark & \cmark & \cmark\\
%     \midrule
%     \rowcolor{green1}   
%     MoELoRA~\cite{chen2024coin} & MM-CIT & TIL & LLaVA 
%     & \xmark & \xmark & \xmark & MoELoRA & \cmark & \xmark & \cmark
%     \\
%     \midrule
%     %CKT-MHA~\cite{qi2024interactive}\\ 
%     %\midrule
%     \rowcolor{green1}   
%     \cite{he2023continual}& MM-CIT & TIL & InstrutBLIP
%     & \xmark & task-similarity-informed regularization(TIR) & expand the projection layer & EProj & \cmark & \xmark & \cmark\\
%     \midrule
%     \rowcolor{green1}   
%     \cite{zhu2024model} & MM-CIT & TIL & InstrutBLIP and LLaVA-1.5
%     & \xmark & \xmark & \xmark & model tailor & \cmark & \cmark & \cmark \\
%     \midrule
%     \rowcolor{green1}   
%     RebQ~\cite{zhao2024reconstruct}& CMML & TIL & ViLT
%     & \xmark & \xmark & \xmark & prompt tuning & \cmark & \xmark & \cmark \\
%     \midrule   
% 	\bottomrule[0.15em]
% 	\end{tabular}
% 	}
% \end{table*}
% \footnotetext[1]{In this paper, only qualitative demonstration has been shown.}
% \footnotetext[2]{In \cite{gururangan2020dont}, there are 4 domains considered in parallel, not sequentially in this work.}





%%%%% Second Version of CFT table.
%%%%
\setlength{\aboverulesep}{0pt}
\setlength{\belowrulesep}{0pt}
\begin{table*}[htbp]
    \centering
    \caption{
    \textbf{Summary of the existing studies on Continual Fine-Tuning LLMs,} where the papers are organized in five main categories based on what downstream tasks they are designed to tackle, including (i)~General Continual Fine-Tuning~(CFT); (ii)~Continual Instruction Tuning~(CIT); (iii)~Continual Model Refinement~(CMR); (iv)~Continual Model Alignment~(CMA); (v)~Continual Multimodal LLMs~(CMLLMs), which is shown in the column of \textbf{CFT Type}.
    The column of \textbf{X-IL} shows what continual learning paradigm the study includes~\cite{van2022three}, where 
    \emph{TIL} represents task-incremental learning, meaning task ID/information is provided during inference; 
    \emph{DIL} represents domain-incremental learning, meaning the tasks are defined in the same format, and no task ID/information is available during inference; 
    \emph{CIL} represents class-incremental learning, meaning the task ID needs to be further inferred when testing. 
    % Among 34 papers shown in the table, 100\%~(34/34) of them explicitly deploy the continual techniques to address the challenge of CFT. 
    % Furthermore, 30\%~(10/34) of them develop their own new techniques that cannot be easily categorized into the three mainstream sets of continual learning algorithms.
    }
    \label{tab:cft}
    \resizebox{1\linewidth}{!}{%
    \setlength{\tabcolsep}{2pt}
\begin{tabular}{ccccc ccccc c}
	\toprule[0.15em]
    \multirow{2}{*}[-0.6em]{\textbf{CFT Type}} & 
    \multirow{2}{*}[-0.6em]{\textbf{Method}} & 
    \multirow{2}{*}[-0.6em]{\textbf{X-IL}} & 
    \multirow{2}{*}[-0.6em]{\textbf{LLM Arch.}} & 
    \multicolumn{4}{c}{\textbf{{Continual Learning Tech.}}} & 
    \multicolumn{3}{c}{\textbf{{Continual Learning Eval.}}} \\
    \cmidrule(lr){5-8}\cmidrule(lr){9-11}
    & & & & \emph{Rehearsal} & \emph{Param. Reg.} & \emph{Arch. Exp.} & \emph{Others} & \emph{\makecell{Avg. \\Acc.}} & \emph{\makecell{Bwd. \\Trans.}} & \emph{\makecell{Fwd. \\Trans.}} \\
	%  &
 %% General Continual Fine-Tuning
    \midrule
    \midrule
    
    \multirow{11}{*}[0em]{\makecell{General}} & CTR~\cite{ke2021achieve} & DIL | CIL & BERT 
    & \xmark & \xmark & Adapter & \xmark 
    & \cmark & \cmark & \cmark \\
    \cmidrule{2-11}

    
    & \cite{tao2022can} & TIL & BERT 
    & S-Replay & \xmark & \xmark & \xmark 
    & \club & \club & \club \\
    \cmidrule{2-11}

    
    & CIRCLE~\cite{wei2022circle} & DIL & T5
    & Replay & EWC & Prompt & \xmark 
    & \cmark & \cmark & \cmark \\
    \cmidrule{2-11}

    
    & ConPET~\cite{song2023conpet} & DIL & Llama 
    & Replay & \xmark & LoRA & \xmark 
    & \cmark & \cmark & \cmark \\
    \cmidrule{2-11}

    
    & \cite{bai2023enhancing} & DIL | CIL & BERT 
    & \xmark & \xmark & \xmark & G-Prompt 
    & \cmark & \cmark & \xmark \\
    \cmidrule{2-11}

    
    & \cite{luo2023investigating} & TIL & \makecell{DistilBERT \\ ALBERT | RoBERTa} 
    & \makecell{ER | DER | LwF} & \xmark & \xmark & \xmark 
    & \club & \club & \xmark \\
    \cmidrule{2-11}

    
    & SEQ$^*$~\cite{zheng2023learn} & TIL | CIL & \makecell{Pythia | BERT | GPT2} & \xmark & P-Freeze & \xmark & Tricks for Classifiers 
    & \xmark & \cmark & \xmark \\
    \cmidrule{2-11}

    
    & LFPT5~\cite{qin2021lfpt5} & DIL & T5 
    & P-Replay & \xmark & \xmark & \xmark 
    & \cmark & \cmark & \xmark\\
    \cmidrule{2-11}

    
    & \cite{weyssow2023usage} & DIL & \makecell{RoBERTa | GPT2} 
    & Replay & \makecell{EWC | SI | RWalk} & \xmark & \xmark 
    & \cmark & \cmark & \xmark \\
    \cmidrule{2-11}

    
    & LR~ADJUST~\cite{winata2023overcoming} & DIL & XLM-R 
    & \xmark & \xmark & \xmark & LR Scheduling 
    & \cmark & \cmark & \cmark \\
    \cmidrule{2-11}

    
    & C3~\cite{chen2024parameterizing} & TIL & T5 
    & KD & \xmark & Prompt Tuning & \xmark 
    &\cmark & \cmark & \xmark \\


    
    %% Continual Instruction Tuning
    \midrule
    \midrule

    
    & CT0~\cite{scialom2022fine} & TIL & T0 
    & S-Replay & \xmark & \xmark & \xmark 
    & \cmark & \cmark & \cmark \\
    \cmidrule{2-11}

    
    & RCL~\cite{wang2023trace} & TIL & \makecell{LLaMA \\ Vicuna | Baichuan} & Replay & \xmark & \xmark & \xmark 
    & \cmark & \cmark & \cmark \\
    \cmidrule{2-11}

    
    % \rowcolor{cream}
    & DynaInst~\cite{mok2023large} & TIL & BART 
    & Replay & \xmark & \xmark & \xmark 
    & \cmark & \cmark & \cmark \\
    \cmidrule{2-11}

    
    % \rowcolor{cream}
    & CITB~\cite{zhang2023citb} & TIL & T5 
    & \makecell{Replay | AGEM} & \makecell{L2 | EWC} & AdapterCL & \xmark 
    & \cmark & \cmark & \cmark \\
    \cmidrule{2-11}
    
%    \cite{chen2024coin} & CIT & TIL & LLaVA & \xmark & \xmark & \xmark & \xmark & \cmark & \cmark & \cmark \\
%    \midrule
 %   \rowcolor{cream}
    & SSR~\cite{huang2024mitigating} & TIL & \makecell{LLaMA | Alpaca} 
    & \makecell{RandSel | KMeansSel} & \xmark & \xmark & \xmark 
    & \cmark & \cmark & \cmark \\
    \cmidrule{2-11}

    
    % \rowcolor{cream}
    & KPIG~\cite{he2024dont} & DIL | TIL & \makecell{LLaMA | Baichuan} 
    & \makecell{DynaInst | PCLL | DCL} & \makecell{L2 \\ EWC} & \makecell{DARE \\ LM-Cocktail} & KPIG 
    & \cmark & \cmark & \cmark \\
    \cmidrule{2-11}

    
    % \rowcolor{cream}
    % \rowcolor{cream}
    & ConTinTin~\cite{yin2022contintin} & TIL & BART 
    & Replay & \xmark & \xmark & InstructionSpeak 
    & \cmark & \cmark & \cmark \\
    \cmidrule{2-11}

    
    & O-LoRA~\cite{wang2023orthogonal} & TIL & \makecell{LLaMA | Alpaca} & \xmark  & \xmark &  O-LoRA & \xmark 
    & \cmark & \cmark & \cmark \\
    \cmidrule{2-11}

    
    \multirow{-9}{*}[1em]{{\makecell{CIT}}} & SAPT~\cite{zhao2024sapt} & TIL & \makecell{T5 | LLaMA} 
    & \xmark  & \xmark & \xmark & SAPT 
    & \cmark & \cmark & \cmark \\
    \cmidrule{2-11}

    
    % \rowcolor{cream}
    & InsCL~\cite{wang2024inscl} & TIL & LLaMA 
    & Replay & \xmark & \xmark & InsCL 
    & \cmark & \cmark & \cmark \\
    \midrule
    \midrule

    
    %%%%%%%%%%%%%%%%%%%%%%%%%%%%%%%%%%%
    %% Continual Model Refinement
    %%%%%%%%%%%%%%%%%%%%%%%%%%%%%%%%%%%
    & CMR~\cite{lin2022continual} & DIL & BART 
     & \makecell{ER | MIR | MLR} & \makecell{L2 | EWC} & \xmark & \xmark & \cmark & \cmark & \cmark \\
    \cmidrule{2-11}

    
    % \rowcolor{gray2}
    & GRACE~\cite{hartvigsen2023aging} & DIL & \makecell{T5 | BERT | GPT2} 
    & \xmark & \xmark & Adapter & \xmark 
    & \cmark & \cmark & \xmark \\
    \cmidrule{2-11}

    
    % \rowcolor{gray2}
    & WilKE~\cite{hu2024wilke} & DIL & \makecell{GPT2 | GPT-J} 
    & \xmark & \xmark & Adaptor & \xmark 
    & \cmark & \cmark & \cmark \\
    \cmidrule{2-11}

    
    % \rowcolor{gray2}
    & Larimar~\cite{das2024larimar} & DIL & \makecell{BERT | GPT-J} 
    & \xmark & \xmark & \xmark & Kanerva Memory 
    & \cmark & \cmark & \cmark \\
    \cmidrule{2-11}

    
    % \rowcolor{gray2}
    & MELO~\cite{yu2023melo} & DIL & \makecell{BERT | GPT2 | T5} 
    & \xmark & \xmark & LoRA & \xmark 
    & \cmark & \cmark & \cmark \\
    \cmidrule{2-11}

    & CME~\cite{li2023continual} & DIL & BERT 
    & Replay & \xmark & \xmark & Inner-Prod. Reg. 
    & \cmark & \cmark & \cmark \\
    \cmidrule{2-11}

    
    \multirow{-7}{*}[0em]{{\makecell{CMR}}} & WISE~\cite{wang2024wise} & DIL & \makecell{GPT-J | Llama2 | Mistral} 
    & \xmark & \xmark & \xmark & \makecell{Side Memory}
    & \cmark & \cmark & \cmark \\
    %%%% Continual Model Alignment
    \midrule
    \midrule

    
    % \rowcolor{blue1}
    & COPF~\cite{zhang2023copf} & TIL | DIL & Llama 
    & Replay & Function Reg. & Prompt & \xmark 
    & \checkmark & \xmark & \checkmark \\
    \cmidrule{2-11}

    
    % \rowcolor{blue1}
    & AMA~\cite{lin2024mitigating} & DIL & \makecell{OpenLLaMA | Mistral} & Replay & L1 | L2 & LoRA & Adaptive Model Avg.
    &\club &\club &\club \\
    \cmidrule{2-11} 

    
    \multirow{-3}{*}[0em]{{\makecell{CMA}}} & CPPO~\cite{zhangcppo} & TIL & GPT2 
    & \xmark & Weighting & Prompt & \xmark 
    & \checkmark & \checkmark & \checkmark \\ 
    %%%MLLM
    \midrule
    \midrule
    
    % \rowcolor{green1}   
    & EProj~\cite{he2023continual} & TIL & InstructBLIP
    & \xmark & TSIR & Projector Exp. & \xmark 
    & \cmark & \xmark & \cmark\\
    \cmidrule{2-11}
       
    & Fwd-Prompt~\cite{zheng2024antiforgetting} & TIL & \makecell{InstructBLIP | BLIP2} & \xmark & \xmark & Projector Exp. & \xmark 
    & \cmark & \cmark & \cmark\\
    \cmidrule{2-11}

    
    % \rowcolor{green1}   
    & CoIN~\cite{chen2024coin} & TIL & LLaVA 
    & \xmark & \xmark & \makecell{MoE | LoRA} & \xmark 
    & \cmark & \xmark & \cmark
    \\
    \cmidrule{2-11}
    %CKT-MHA~\cite{qi2024interactive}\\ 
    %\midrule
    % \rowcolor{green1}   

    
    & Model~Tailor~\cite{zhu2024model} & TIL & 
    \makecell{InstructBLIP | LLaVA}
    & \xmark & Model~Tailor & \xmark & \xmark 
    & \cmark & \cmark & \cmark \\
    \cmidrule{2-11}

    
    \multirow{-5}{*}[0em]{{\makecell{CMLLMs}}} &  RebQ~\cite{zhao2024reconstruct} & TIL & ViLT
    & \xmark & \xmark &  Prompt Tuning & \xmark 
    & \cmark & \xmark & \cmark \\
	\bottomrule[0.15em]
	\end{tabular}
	}
\end{table*}




\subsubsection{General Observations on CFT}
\label{sec:cft-obs}
% \begin{itemize}
%     \item In the realm of LLMs, focus of the continual learning community has shifted from class-incremental learning to domain-incremental and task-incremental learning.
%     \item Compared to CPT and DAP, continual learning techniques are more commonly deployed in CFT, as this is an explicit topic studied across the community. 
%     \item More domain-specific and advanced algorithms, beyond trivial adoption of the existing continual learning techniques, are designed for different sub-categories of CFT, showing promising future of the current research trend.
% \end{itemize}

Examining the landscape of continual learning in the context of LLMs, and combined with the results shown in \Tabref{tab:cft}, we make several key observations about CFT. 
\begin{itemize}
    \item \textbf{OBS-1: There has been a noticeable transition in focus from CIL to TIL and DIL.}
    It has been a longstanding common sense in the CL community that CIL, as it requires the model to predict the context label and within-context label at the same time~\cite{van2022three,wang2024comprehensive,kim2022theoretical}, is the most challenging CL scenario and hence receives most of the attention from the community. 
    However, among all 35 papers presented in \Tabref{tab:cft}, only 3 papers study CFT of CIL.
    The transition of the research focus demonstrates the importance of TIL and DIL in the real-world applications of continual LLMs. More detailed discussion of this transition is included in \Secref{sec:discussion-xil}. 
    \item \textbf{OBS-2: In CFT, CL techniques enjoy broader adoption and explicit exploration compared to CPT and DAP.}
    In \Tabref{tab:cft}, all 35 papers explicitly deploy the CL techniques, 50\% of which develop new techniques that cannot be easily interpreted as trivial combination of existing classic CL techniques, e.g., shared attentive learning framework in SAPT~\cite{zhao2024sapt}, external memory deployed in Larimar~\cite{das2024larimar}, and adaptive model averaging method to achieve Pareto-optimal in AMA~\cite{lin2024mitigating}, etc.
    This underscores the recognition of continual learning as a pivotal component in the development of resilient and adaptive LLMs.
\end{itemize}
\setlength{\aboverulesep}{2pt}
\setlength{\belowrulesep}{2pt}

% Firstly, \textbf{\emph{there has been a noticeable transition in focus from class-incremental learning to domain-incremental and task-incremental learning paradigms}}. For example, in 12 papers of general continual fine-tuning~(the first white section), only 3 papers study continually fine-tuning language models in the setting of class-incremental learning. In the remaining four topics, CIT completely belongs to the field of task-incremental learning, as the instruction provided to the models can be seen as a soft encoding of the task information; CMR completely belongs to the field of domain-incremental learning, as different editing examples follows the same problem definition, and no task information is provided during inference. In CMA and CMLLMs, no examples of class-incremental learning has been reported. 

% It has been a longstanding common sense in the continual learning community that class-incremental learning, as it requires the model to predict the context label and within-context label at the same time~\cite{van2022three,wang2024comprehensive,kim2022theoretical}, is the most challenging continual learning scenario and hence receives most of the attention from the community. 
% However, the evolution observed so far suggests a recognition of the broader spectrum of challenges faced in real-world applications of continually learning LLMs. This growing awareness of the importance of task-incremental and domain-incremental learning underscores the necessity for more comprehensive and specific approaches to continual LLMs, with precedent continual learning literature provided as reference. 

% Furthermore, \emph{\textbf{in continual fine-tuning~(CFT), continual learning techniques enjoy broader adoption and explicit exploration compared to CPT and DAP}}. 
% In \Tabref{tab:cft}, all 34 papers explicitly deploy the continual techniques, 50\% of which develop their own new techniques that cannot be easily categorized into the three mainstream sets of continual learning algorithms. 
% For example, 
% % O-LoRA proposes to use different LoRA modules to learn different orthogonal subspaces to minimize the mutual interference during inference~\cite{wang2023orthogonal}; 
% in instruction tuning, SAPT designs a shared attentive learning framework to enable the catastrophic forgetting mitigation and knowledge transfer at the same time~\cite{zhao2024sapt}; 
% in model refinement, Larimar proposes to adopt an external memory system, Kanerva Memory, that supports operations of read, write, and generate~\cite{das2024larimar}; 
% in model alignment, AMA proposes adaptive model averaging method to achieve Pareto-optimal when averaging different models for a reward-tax trade-off~(alignment-performance trade-off)~\cite{lin2024mitigating};
% in continual learning multimodal LLMs, \cite{zhao2024reconstruct} devises a novel method named Reconstruct before Query (RebQ), harnessing the multi-modal knowledge from a pre-trained model to reconstruct the absent information for the missing modality.

% Hence we can further conclude that, beyond mere replication of existing techniques, \textbf{\emph{researchers are actively developing tailored solutions for various scenarios where an LLM needs to be continually updated directly for downstream tasks.}} 
% This emphasis underscores the recognition of continual learning as a pivotal component in the development of resilient and adaptive language models, and further signals a maturation of the field towards more specialized and effective continual learning methodologies for large language models.


\subsubsection{General Continual Fine-Tuning~(General CFT)}
\label{sec:cft-general}
% \haizhou{Structure of the general CFT: 
%     \begin{itemize}
%         \item investigations of the natural robustness against forgetting of pre-trained large language models.
%         \item list of other works
%     \end{itemize}
% }

Researchers have long investigated the phenomenon of forgetting resilience in pre-trained LLMs when fine-tuned for downstream tasks \cite{ke2021achieve,tao2022can,luo2023investigating,zheng2023learn,mehta2023empirical}, despite some discover the opposite~\cite{luo2023investigating}. 
Although the pre-trained weights initially position the model in a flat-loss basin, aiding adaptation to future tasks without severely impacting previous ones \cite{mehta2023empirical}, zero or near-zero forgetting is only observed at the representation level. This implies that while the model retains its ability to distinguish between task-specific representations, it may still forget specific task details \cite{wu2021pretrained,tao2022can,luo2023investigating,zheng2023learn}. Therefore, additional measures are necessary when deploying these models in real-world applications \cite{ke2021achieve,wei2022circle,bai2023enhancing,qin2021lfpt5,weyssow2023usage,chen2024parameterizing}. 
% For instance, \cite{weyssow2023usage} demonstrate that relying solely on the intrinsic anti-forgetting and generalization abilities of LLMs can be risky in scenarios with changing code and API distributions, and simple baseline continual learning techniques can significantly enhance performance.

% naive fine-tuning with simple tricks
Many studies advance beyond naive sequential fine-tuning, leveraging the inherent anti-forgetting nature of LLMs while avoiding the adoption of overly complex CL techniques~\cite{winata2023overcoming,zheng2023learn}. For instance, LR~ADJUST~\cite{winata2023overcoming} proposes a straightforward yet effective method of dynamically adjusting the learning rate to mitigate the overwriting of knowledge from new languages onto old ones. Building on the innate anti-forgetting ability of large language models like Pythia \cite{biderman2023pythia}, SEQ$^*$ \cite{zheng2023learn} introduces several strategies for fine-tuning LLMs on a sequence of downstream classification tasks, such as freezing the LLM and old classifier's parameters after warm-up, and pre-allocating future classifiers, etc. 
% : (i) freezing the LLMs and old classifiers after the warm-up phase and learning new tasks, respectively; (ii) employing (cosine) linear classifiers when old data is (not) available; and (iii) pre-allocating future classifiers.


% representation-level constraints
Given the minimal forgetting observed at the representation level in CL, some studies aim to tackle the misalignment between the representation space and the decision-making layers by introducing representation-level constraints during CFT. NeiAttn~\cite{bai2023enhancing} exemplifies this approach by formulating classification tasks as masked language modeling and proposing a neighboring attention mechanism to counteract negative representation drift. 
% This method seamlessly complements existing continual learning techniques \cite{chaudhry2019tiny,chaudhry2019efficient,sprechmann2018memory}, as its focus is independent of mainstream continual learning methods.


% architectural design of different adapters embedded in the transformers
Another line of approaches refines the input/output format and network architectures of pre-trained LLMs to be better suited for CFT. 
For instance, CTR~\cite{ke2021achieve} incorporates two CL-plugin modules, i.e., a task-specific module~(TSM) for acquiring task-specific knowledge and a knowledge-sharing module~(KSM) for selectively transferring previously learned similar knowledge. 
% CTR demonstrates the effectiveness of this architectural design in class-incremental and domain-incremental settings. 
% prompts
CIRCLE~\cite{wei2022circle} manually designs diverse prompt templates for various types of buggy code, unifying them as the cloze task and employs difficulty-based replay to enhance continual program repair. 
LFPT5~\cite{qin2021lfpt5} addresses lifelong few-shot language learning by consolidating sequence labeling, text classification, and text generation into a text-to-text generation task. It undergoes prompt tuning on generated pseudo-examples from previous domains when adapting to new tasks. 
% in-context learning + PET techniques
In \cite{zhang2022continual}, the authors propose a method for adaptively adding compositional adapters during continual sequence generation tasks. Before training on new domains, a decision stage determines which trained module can be reused. During training, this module also regenerates examples of the past for replay.
C3~\cite{chen2024parameterizing} merges PEFT and in-context learning (ICL) in a teacher-student framework. The teacher model undergoes in-context tuning focused solely on the current domain, while the student model, together with tunable prompts, minimizes the KL-divergence between the output distribution and the ground truth and teacher model simultaneously.



% As we point out before, we do not have a full list of the continual fine-tuning language models included in this subsection. 
% For a more comprehensive survey that is specifically to the point of continual natural languages models, please refer to \cite{biesialska2020continual} and \cite{ke2023continual}.
% Our main focus of the discussion will be centered on the following subsections.



\subsubsection{Continual Instruction Tuning~(CIT)}
\label{sec:cft-cit}
% \haizhou{Structure of the CIT: 
%     \begin{itemize}
%         \item General Introduction to Instruction Tuning. 
%         \begin{itemize}
%             \item Highlighting the difficulties in changing preferences.
%             \item Formal definition of IT (can be found in multiple papers).
%         \end{itemize}  
%         \item Definition of Continual IT (CIT). 
%         \item Taxonomy of existing CIT methods.
%         \begin{itemize}
%             \item List all the methods, describe what they do.
%         \end{itemize}
%         \item What is missing, and special aspects that are different from traditional CL paradigms.
%     \end{itemize}
% }

% Instruction Tuning~(IT) is a technique used to refine the instruction-following capabilities of LLMs~\cite{zhang2024instruction}. 
% While LLMs are typically pre-trained on extensive and diverse corpora, they may struggle with specific tasks such as instruction following despite their general knowledge. Numerous studies have shown that Instruction Tuning~(IT) can notably improve LLMs' ability to follow textual instructions~\cite{zhang2024instruction,wei2021finetuned,jiang2024instructiontuned,sanh2022multitask,ouyang2022rlhf}, leveraging the pre-existing knowledge within LLMs to bridge the gap between general and task-specific performance~\cite{wei2022finetuned}. Recent works like WizardLM~\cite{xu2023wizardlm} and CodecLM~\cite{wang2024codeclm} further tailor synthetic data to steer LLMs' behavior through IT.  Additionally, IT enhances the interaction between humans and LLMs, providing a more natural interface and aligning LLM outputs more closely with human expectations and preferences~\cite{luo2023empirical}.
% \todo{Some observations and conclusions about the CIT, about its continual learning techniques deployed and evaluation.} 
% The replay-based method is a prevalent technique in Instruction Tuning for addressing catastrophic forgetting, favored for its simplicity and minimal computational and memory demands. This method typically encompasses both domain and task-incremental learning. In terms of evaluation, besides accuracy, metrics such as Relative Gain and the forgetting rate are crucial for assessing how effectively the model acquires new knowledge while retaining previous information.

% \textbf{Continual Instruction Tuning~(CIT).}\quad 
When the instruction tuning data comes in as a stream, 
forgetting of the previously learned instructions should be addressed. 
% this series of the fine-tuning tasks need to address the challenge of catastrophic forgetting of the general knowledge. 
% \todo{Can we organize them better? E.g., their common techniques, common focus of the CIT.}
CT0~\cite{scialom2022fine} represents the inaugural study on Continual Instruction Tuning~(CIT) of LLMs, applying the replay method on the base T0 model throughout the process. 
% This model successfully learns 8 new tasks while maintaining robust performance on previously learned tasks. 
Many subsequent studies focus on enhancing the replay method used during CIT. For instance, \cite{he2024dont} improve replay efficiency by computing Key-Part Information Gain~(KPIG) on masked parts to dynamically select replay data, addressing the ``half-listening'' issue in instruction following. Similarly, SSR~\cite{huang2024mitigating} uses the LLM to generate synthetic instances for replay, achieving superior or comparable performance to traditional methods at a lower cost.

Other approaches introduce multiple CL techniques during CIT. DynaInst~\cite{mok2023large} merges parameter regularization with dynamic replay, selectively storing and replaying instances and tasks to enhance outcomes. InstructionSpeak~\cite{yin2022contintin} employs negative training and replay instructions to improve both forward transfer and backward transfer. Some methods incorporate PEFT. Orthogonal Low-Rank Adaptation~(O-LoRA) learns new tasks within an orthogonal subspace while preserving LoRA parameters for previous tasks~\cite{wang2023orthogonal} to minimize the interference among different tasks. Shared Attention Framework~(SAPT) combines a PET block with a selection module via a Shared Attentive Learning \& Selection module, tackling catastrophic forgetting and knowledge transfer concurrently~\cite{zhao2024sapt}.
While regularization-based and architectural-based methods require additional parameter storage and GPU memory, together with replay-based methods they remain for CIT due to the simplicity and effectiveness~\cite{wang2024inscl}. 
% These methods are particularly favored in traditional continual learning scenarios for tuning LLMs.

%%%% FULL
% \textbf{CIT vs Conventional CL.}\quad 
% Both CIT and conventional CL aim to enable LLMs to acquire new tasks or information over time while retaining previously learned knowledge. However, CIT specifically utilizes rich natural language instructions to enhance LLMs' ability to follow human instructions~\cite{zhang2023citb}. The natural-language encoding of task information enables positive forward transfer when semantically similar tasks are encountered in the data stream. This scenario is typically challenging to engineer manually in conventional continual learning setups.
% In contrast, conventional CL focuses on broader knowledge acquisition and is not limited to instruction-based learning. While CIT is predominantly applied within LLMs, conventional CL is employed across various fields, including vision, multimodal models and robotics. Regarding challenges, both approaches contend with catastrophic forgetting; however, CIT additionally concentrates on refining instruction-following capabilities to better interact with human needs. Conversely, conventional CL emphasizes building robust generalizability to manage variations within and across tasks~\cite{wang2024comprehensive}. 


\subsubsection{Continual Model Refinement~(CMR)}
\label{sec:cft-cmr}
The concept of model editing was initially explored in \cite{sinitsin2020editable}, which introduced a \emph{``reliability-locality-efficiency''} principle and proposed a gradient descent editor to address it efficiently. Subsequent research, such as \cite{de2021editing} and \cite{fast_edit}, extended this principle to edit factual knowledge in BERT-based language models and larger models like GPT-J-6B~\cite{gpt-j} and T5-XXL~\cite{raffel2020exploring}, respectively, using gradient decomposition. These approaches typically update a subset of model parameters to alter the labels of specific inputs. Additionally, memory-based models, as discussed in \cite{mitchell2022memory} and \cite{hartvigsen2023aging}, incorporate editing through retrieval mechanisms.
%%%%%%%%% FULL
% \textbf{Model Refinement as a Special Form of Continual Learning.}\quad 
% Although model refinement and continual learning are typically treated as separated fields, they share fundamental similarities. The principle of \emph{locality} in model refinement, ensuring that new knowledge doesn't disrupt responses to other inputs, aligns with the non-forgetting objective of continual learning. Furthermore, the principles of \emph{reliability}—effectively updating the model—and \emph{efficiency} are crucial in both model refinement and continual learning Thus, the problem of model refinement can be viewed as akin to continual learning, where a small batch of updated samples $\{(\vx_e, \hat{y}_e)\}$ represents a new task. Alternatively, one can consider the stream of such samples for updating as a specialized form of online continual learning~\cite{ying2021mitigating,mai2022online,prabhu2023online}.
% It is important to note that, although there are similarities between them, the difference in the volume of knowledge updates distinguishes the approaches used in the fields of continual learning and model editing. Continual learning typically involves learning new tasks or knowledge from a large dataset, while model editing aims to refine current knowledge with only a few samples. Therefore, for model refinement, efficiently localizing a sub-structure of the network and making minor modifications to it is often efficient and effective. Methods used in large-scale continual learning may not be suitable in this context.

% \textbf{Model Refinement as a Special Form of Continual Learning.}\quad 
% Although model refinement and continual learning are typically treated as distinct areas of study, they share fundamental similarities. The principle of \emph{locality} in model refinement, ensuring that new knowledge doesn't disrupt responses to other inputs, aligns with the non-forgetting objective of continual learning. Additionally, the principles of \emph{reliability} (successfully update the model) and \emph{efficiency} in model refinement are also pertinent to continual learning. Thus, the problem of model refinement can be viewed as akin to continual learning, where a small batch of updated samples $\{(\vx_e, \hat{y}_e)\}$ represents a new task. Alternatively, one can consider the stream of such samples for updating as a specialized form of online continual learning~\cite{mai2022online,prabhu2023online}.
% It is important to note that, although there are similarities between them, the difference in the volume of knowledge updates distinguishes the approaches used in the fields of continual learning and model editing. Continual learning typically involves learning new tasks or knowledge from a large dataset, while model editing aims to refine current knowledge with only a few samples. Therefore, for model refinement, efficiently localizing a sub-structure of the network and making minor modifications to it is often efficient and effective. Methods used in large-scale continual learning may not be suitable in this context.

% \textbf{Continual Model Refinement~(CMR).}\quad 
% Recent works have combined continual learning with model refinement, resulting in a new problem termed continual model refinement~(CMR). 
Continual Model Refinement~(CMR) extends model refinement horizontally, presenting updated sample pairs ${(\vx_e, y_e, \hat{y}_e)}^{e=1}_N$ sequentially as a stream. \cite{lin2022continual} initially introduces this idea, evaluating various CL methods with a dynamic sampling algorithm. Many CMR methods employ a retrieval mechanism. For instance, \cite{hartvigsen2023aging} uses hidden activations of the language model as a ``key'' to activate updated parameters only when input $x_0$ resembles updated sample pairs; 
\cite{yu2023melo} improves this approach's efficiency by integrating LoRA \cite{hu2021lora}; \cite{das2024larimar} augments the LLM with an external episodic memory, modeling CMR as an ongoing memory refresh. 
Meanwhile, some methods focus solely on updating a subset of model parameters. For example, \cite{hu2024wilke} addresses the issue of ``toxicity buildup and flash'' in single-editing methods like ROME \cite{meng2022locating}, adapting it to the CL context with a knowledge-aware layer selection algorithm.
WISE~\cite{wang2024wise} addresses the ``impossible triangle'' of reliability, locality, and generalization in existing lifelong model refinement methods. It introduces a side memory system that enables knowledge sharding and merging, successfully achieving all three objectives simultaneously.

% Recent works have combined continual learning with model refinement, resulting in a new problem termed continual model refinement~(CMR). This concept extends model refinement horizontally, presenting updated sample pairs ${(\vx_e, y_e, \hat{y}_e)}^{e=1}_N$ sequentially as a stream. \cite{lin2022continual} initially introduces this idea as continual model refinement, evaluating various continual learning methods with a dynamic sampling algorithm. \cite{hartvigsen2023aging} uses hidden activations of the language model as a "key" to activate updated parameters only when input $x_0$ resembles updated sample pairs, preserving other predictions. 
% \cite{yu2023melo} improves this approach's efficiency by integrating LoRA \cite{hu2021lora}. 
% Meanwhile, \cite{das2024larimar} augments the LLM with an external episodic memory, modeling CME as an ongoing memory refresh. 
% \cite{hu2024wilke} addresses the issue of ``toxicity buildup and flash'' in single-editing methods like ROME \cite{meng2022locating}, adapting it to a continual context with a knowledge-aware layer selection algorithm.

% \subsubsection{OLD: Continual Model Refinement~(CMR)}
% \label{sec:cft-cmr-old}
% Like humans, LLMs cannot avoid making mistakes. They may produce inaccurate translations due to domain-specific terminology or incorrectly inform that the last global pandemic was swine flu rather than COVID-19, due to outdated training data \cite{de2021editing}. Directly fine-tuning the model to correct such mistakes could be time-consuming and may adversely affect the model's performance on other samples. To address these challenges, model editing is proposed, aiming to correct the model's mistakes without altering its performance on other inputs and using only moderate computing resources.
% % structure of the section
% In the remaining section, we first introduce the formal definition of model editing and a quick overview of its current research. Then we introduce the idea of viewing model editing from the perspective of continual learning, and the currently understudied challenging setting of continual model editing. 

% % \textbf{Formal Definition of Model Editing \cite{de2021editing}.}\quad 


% Model editing was first proposed by \cite{sinitsin2020editable}. Subsequently, \cite{de2021editing} applied it to edit factual knowledge in BERT-based language models, and \cite{fast_edit} extended its application to larger language models, including GPT-J (6B) and T5-XXL (11B), through gradient decomposition. Most methods, including \cite{de2021editing}, \cite{fast_edit}, \cite{hase2021language}, and \cite{huang2023transformer}, focus on updating a fraction of model parameters to successfully change the labels of specific inputs. Meanwhile, memory-based models, such as those discussed in \cite{mitchell2022memory} and \cite{hartvigsen2023aging}, apply editing through retrieval mechanisms.

% \textbf{Model Editing as a Special Form of Continual Learning.}\quad 
% It is important to note that while continual learning and model editing both update models while preserving the original output for unchanged inputs, they differ significantly. Continual learning usually focuses on learning new tasks or knowledge from a large dataset, whereas model editing concentrates on refining current knowledge with just a few samples. However, model editing can still be seen as a special form of continual learning, as each updated sample pair $(x_e, \hat{y_e})$ can be regarded as a new task.

% \textbf{Continual Model Editing (CME).}\quad 
% Recent works haved combined continual learning with model editing to introduce a new problem setting termed continual model editing (CME). It is considered a "horizontal generalization" of model editing, where updated sample pairs ${(x_e, y_e, \hat{y_e})}^{e=1}_N$ are presented sequentially as a stream. \cite{lin2022continual} first introduced this concept as continual model refinement, assessing several continual learning methods with a dynamic sampling algorithm. \cite{hartvigsen2023aging} utilized the hidden activations of the language model as a "key" to activate updated parameters only when input $x_0$  resembles updated sample pairs, thus keep other predictions unchanged. \cite{yu2023melo} enhanced the efficiency of this approach by using LoRA \cite{hu2021lora}. Meanwhile, \cite{das2024larimar} augmented the LLM with an external episodic memory, modeling CME as an ongoing memory refresh. \cite{hu2024wilke} addressed the issue of "toxicity buildup and flash" in single-editing methods like ROME \cite{meng2022locating}, adapting it to a continual context with a knowledge-aware layer selection algorithm.


% draft
% \cite{lin2022continual} further generalize it to continual model refinement, where 

% \cite{lin2022continual}.
% % \cite{li2023continual}.
% \cite{yu2023melo}.
% \cite{das2024larimar}.
% \cite{hu2024wilke}.

While all these works pioneer research in CMR, the exploration of CMR of LLMs remains open. \cite{hase2023does} highlights a potential problem: the location for storing the fact may not coincide with the best place for editing it. This challenges the classical ``locate and edit'' paradigm used by several existing methods~\cite{meng2022locating, meng2022mass}, and could become a significent concern for CMR~\cite{hu2024wilke}. Other questions, including whether such problem setting fits LLMs and whether more memory/computationally efficient methods of CMR could be developed for LLMs, are yet to be answered.

% 1. ROHIT那篇是只在一层的某个weight上做,但是这个weight选择和层选择都是靠Tracing
% 第一篇文章揭露这个层选不科学,第二篇文章就有一个动态层选

% might not coincide with the best place for editing it

% Like humans, LLMs cannot avoid making mistakes. They may produce inaccurate translations due to domain-specific terminology or incorrectly inform that the last global pandemic was swine flu rather than COVID-19, due to outdated training data. Directly fine-tuning the model to correct such mistakes could be time-consuming and may adversely affect the model's performance on other samples. To address these challenges, model editing is proposed, aiming to correct the model's mistakes without altering its performance on other inputs and using only moderate computing resources.

% Formally, suppose we have a model $f(x,\theta)$, parameterized by $\theta$ and taking data $x$ (e.g., natural language queries) as inputs. Consider a size-$N$ set ${ (x_e, y_e, \hat{y_e})}^{e=1}_N$, where the model incorrectly predicts $y_e$ for $x_e$, and we aim to efficiently update the model from $f$ to $f'$ such that it correctly predicts $\hat{y_e}$. For other inputs $x_0 \notin {x_e}$, we desire $f(x_0) = f'(x_0)$. This is the basic problem setting of model editing.

% It is important to note that while continuous learning and model editing both update models while preserving the original output for unchanged inputs, they differ significantly. Continuous learning usually focuses on learning new tasks or knowledge from a large dataset, whereas model editing concentrates on refining current knowledge with just a few samples. However, model editing can still be seen as a special form of continuous learning, as each updated sample pair $(x_e, \hat{y_e})$ can be regarded as a new task.

% Model editing was first proposed by \cite{sinitsin2020editable}. Subsequently, \cite{de2021editing} applied it to edit factual knowledge in BERT-based language models, and \cite{fast_edit} extended its application to larger language models, including GPT-J (6B) and T5-XXL (11B), through gradient decomposition. Most methods, including \cite{de2021editing}, \cite{fast_edit}, \cite{hase2021language}, and \cite{huang2023transformer}, focus on updating a fraction of model parameters to successfully change the labels of specific inputs. Meanwhile, memory-based models, such as those discussed in \cite{mitchell2022memory} and \cite{hartvigsen2022aging}, apply editing through retrieval mechanisms.

% A current work, \cite{hartvigsen2022aging}, focuses on combining continual learning and model editing into what is termed "Continual Model Editing." This approach can be seen as a "horizontal generalization" of model editing. Compared with previous works, it introduces two novelties: first, the updated sample pairs are presented in a streaming manner, requiring the model to update its prediction for the current input $x_e^T$ to $\hat{y_e^T}$ while keeping all previous modifications $f'(x_e^t) = \hat{y_e^t}$ for $1 \leq t < T$ unchanged; second, it operates without access to the model's previous training dataset. Yet, it still fulfills the requirement that for inputs $x_0 \notin {x_e}$, $f(x_0) = f'(x_0)$. This work utilizes the hidden activations of the language model as the "key" to "trigger" the updated parameters when a new example's hidden states fall within a range of the key. When inputs $x_0$ are far from the updated sample pairs, this method keeps the main model frozen, thereby not affecting its predictions.

% While this work pioneers research in continual model editing, the exploration of continual model editing for large language models (LLMs) remains open. Questions about whether such a paradigm fits LLMs and whether more memory/computationally efficient methods of continual model editing could be developed for LLMs are yet to be explored and answered.





% Like humans, large language models cannot avoid making mistakes. It may produce inaccurate translation due to the domain-specific terminology, or it may tell you that the last global pandemic is swine flu rather than covid-19 due to the outdated training data. If we directly fine-tune the model to correct such mistakes, it would be very time consuming, with a possibility that affects the model's performance on other samples. To address these issues, model editing is proposed. Generally, it aims to correct model's mistakes without altering model's performance on other inputs, and only use moderate computing resources.

% Formally, suppose we have a model $f(x,\theta)$, which is parameterized by $\theta$ and takes data $x$ (e.g., natural language queries) as inputs. We have a size-N set $\{ (x_e, y_e, \hat{y_e})\}^{e=1}_N$, where the model gives wrong predictions $y_e$ given $x_e$, and we want to efficiently update the model from $f$ to $f'$ such that it gives correct prediction $\hat{y_e}$. For other inputs $x_0 \notin \{x_e\} $, we want to keep $f(x_0) = f'(x_0)$. 
% % We also define an set $I(x_e)$ that contains all the semantic-identical samples of $x_e$, for instance, images for the same animal but in different views, or the paraphrases of the  

% Notice that even though continue learning and model editing both udate moodels while keep the original output the same, they differs in that continue learning usually focus on learning the new task or new knowledge from a large amount of data, while model editing usually focus on refinement of the current knowledge with only a few samples. However, model editing can still be seen as a special form of continue learning as each updated sample pairs $(x_e, \hat{y_e})$ can be seen as a new task.

% Model editing was first proposed by \cite{sinitsin2020editable}. \cite{de2021editing}  applied it to edit factual knowledge of Brt-based language models, and \cite{fast_edit} further incorporated gradient decomposition to apply it to large language models including GPT-J (6B) and T5-XXL (11B). Most methods \cite{de2021editing, fast_edit, hase2021language, huang2023transformer} focus on updating a fraction of model parameters to successfully change the labels of specific input, while there also exist memory-based models \cite{mitchell2022memory, hartvigsen2022aging} that apply editing through retrieval.

% There is a current work \cite{hartvigsen2022aging} that focuses on combining both continual learning and model editing, which give rise to the setting of "Continual Model Editing". We can seen as this combination as a "horizontal generalization" of model editing. Compared with previous works on model editing, it has 2 difference: First, the updated sample pairs come in a streaming way. At each time step $T$ given input $x_e^T$, the method needs to update the model's prediction to $\hat{y_e^T}$, while keeps all the previous modification $f'(x_e^t) = \hat{y_e^t}$ with $1 \leq t < T$ unchanged. Second, it has no access to the previous training dataset of the model. Remember that it still needs to satisfy the requirement that for other inputs $x_0 \notin \{x_e\} $, $f(x_0) = f'(x_0)$. This work use the hidden activations of the language model as the "key" and "trigger" the updated parameters when the new example's hidden states is within a range of the key. When inputs $x_0$ far away from the updated sample pairs, this method will keep the main model frozen and thus does not effect their prediction.

% While this work has pioneered the research of continual model editing, there are still sno works that focus on continual model editing for LLM. Would such paradigm fit with LLM, and would there be more memory/computational effecint methods of continual model editing on LLM? All these questions are waited for us to explore and answer.


\subsubsection{Continual Model Alignment~(CMA)}
\label{sec:cft-cma}
% Model Alignment~(MA) ensures AI systems' actions and outputs align with human values, ethics, and preferences~\cite{ouyang2022rlhf,rafailov2024dpo}.
% % It can be defined as the process of adjusting the objectives and functioning of an AI system to achieve such goals, involving a combination of mathematical models, algorithmic adjustments, and iterative feedback to refine AI behavior.
% MA can be broadly categorized into two types: Reinforcement Learning-based (RL-based) and Supervised Learning-based (SL-based). RL-based approaches~\cite{wu2022survey,ouyang2022rlhf,schulman2017proximal} are trained to make decisions reinforced by human feedback, using a reward system to guide them towards desirable outcomes. Conversely, the SL-based approaches~\cite{hendrycks2023aligning, rafailov2024dpo,ji2024ai} directly train models on datasets of human preferences, aligning their output with demonstrated human values. Both approaches leverage a combination of algorithmic learning techniques and human feedback to progressively refine the model behavior.
% \todo{Done: A paragraph about the Alignment Tax, describing the phenomenon of the performance degrade when performing model alignment: even a single stage of the MA can cause this problem.}
When LLMs undergo the phase of MA, vertical forgetting of previous knowledge usually occurs. 
In \cite{lin2024mitigating}, the authors refer to this phenomenon of catastrophic forgetting induced caused by MA as the ``Alignment Tax.'' 
% This term describes the performance degradation that occurs 
Notably, even a single stage of MA can diminish the model's performance capabilities, as it restricts the model's responses to a narrower subset of the training distribution. 
% This paper focuses on understanding and reducing the impact of the Alignment Tax in the process of refining AI models, highlighting the balance between alignment and maintaining broad model capabilities.

% \textbf{Continual Model Alignment~(CMA).}\quad 
Continual Model Alignment~(CMA) aims to continuously refine LLMs to align with evolving human values, ethics, and data. 
% The significance of CMA lies in its capacity to ensure that LLMs retain their relevance, accuracy, and ethical alignment over time. This ongoing adjustment is essential to navigate the complexities introduced by concept drift, the evolution of data, and shifts in societal values. 
The static nature of LLM training on historical data sets can lead to discrepancies between the models' outputs and current factual accuracies, societal norms, and standards, making CMA a crucial process for maintaining their adaptability and alignment with contemporary contexts. 
% Despite its importance, CMA faces several challenges: 
% (i)~\emph{scalability and resource intensity issue}: continually updating LLMs requires significant computational resources and human oversight, posing scalability issues for the practitioners; 
% (ii)~\emph{ensuring ethical alignment}: balancing model updates to reflect societal changes without introducing or perpetuating biases remains a complex issue; 
% (iii)~\emph{data privacy and security}: continuously integrating new data into LLMs raises concerns regarding data privacy, security, and the potential for misuse.
% Emerging strategies in CMA emphasize the optimization of LLMs for enhanced adaptability, addressing the necessity for these models to accommodate continuous changes in language use, information validity, and societal expectations. Notably, innovative research efforts such as those detailed in~\cite{lin2024mitigating}, and~\cite{puthumanaillam2024moral}, highlight the development of methodologies designed to minimize the challenges associated with the continuous realignment of LLMs. These studies underscore the importance of implementing specific optimization strategies to foster the adaptability of LLMs, ensuring their outputs remain ethically attuned and factually relevant in the face of dynamic societal and informational landscapes. 
Likewise, there are two types of CMA frameworks: RL-based and SL-based. 
% \todo{More analysis, conclusions, and findings about the currently available works, especially about the forgetting, CL techniques observed in the papers. done}
In the realm of RL-based CMA, two significant contributions have been noted.  \cite{lin2024mitigating} identifies the conflicts between the existing CL techniques and RLHF, and proposes Adaptive Model Averaging~(AMA), adaptively finding appropriate ratios for the combination of model layers to gain maximal rewards with minimal tax;  Continual Proximal Policy Optimization~(CPPO)~\cite{zhangcppo} proposes a weighting strategy for different examples deciding its usage of policy enhancement or knowledge retention, mitigating the alignment tax over time. 
% Employing reinforcement learning for CMA focus on developing efficient reinforcement learning techniques that integrate continual learning principles and human feedback to dynamically maintain alignment with evolving human values while minimizing computational overhead and mitigating forgetting.
For SL-based CMA, Continual Optimal Policy Fitting~(COPF)~\cite{zhang2023copf} presents a solution adapted from the Direct Policy Optimization~(DPO)~\cite{rafailov2024direct}, solving its potential risks of sub-optimal policy fitting and over-optimization in the context of CMA.
% presents an innovative approach by integrating supervised continual learning techniques with the process of aligning AI systems through direct training on evolving datasets of human preferences. This methodology promises a more sustainable and adaptable model for maintaining alignment with human values over extended periods and across various contexts. The use of supervised continual learning suggests a focus on preventing catastrophic forgetting, a common challenge where a model loses its ability to perform previously learned tasks upon learning new ones.

% \todo{a short comment on what is lacking in the current research status. done}
% In summary, both RL-based and SL-based frameworks need to address forgetting, especially in scenarios where the model continuously integrates new information. Effective strategies might include techniques (such as EWC~\cite{kirkpatrick2017overcoming}, experience replay~\cite{chaudhry2019tiny}, and dynamic re-weighting~\cite{lin2024mitigating}), which help the model retain old knowledge while integrating new insights.
% Future research in CMA aims to develop more efficient, automated processes for model updates, better mechanisms for ethical oversight, and innovative solutions to balance model relevance with privacy and security concerns. Existing streams of research highlight the importance of developing adaptable, efficient, and robust AI systems that can continually align with human values without substantial losses in performance or increased computational costs. Future research could explore hybrid models that combine RL-based and SL-based approaches, potentially offering a more holistic framework for continual model alignment.


        
% \subsubsection{OLD: Continual Model Alignment~(CMA)}
% \label{sec:cft-cma-old}
% % \haizhou{Structure of the CMA: 
% %     \begin{itemize}
% %         \item General Introduction to Model Alignment. 
% %         \begin{itemize}
% %             \item Formal definition of MA, including notations and mathematical definitions, using references below. 
% %             \item Two general types of MA: RL-based (ref: RLHF https://arxiv.org/abs/2203.02155) and SL-based (ref: DPO https://arxiv.org/abs/2305.18290). (We need some definitions, equations to better illustrate this.)
% %         \end{itemize}  
% %         \item Continual Model Alignment (CMA). 
% %             \begin{itemize}
% %                 \item Highlighting the difficulties in changing preferences (ref: Mitigating the Alignment Tax of RLHF https://arxiv.org/abs/2309.06256) (ref A Moral Imperative: The Need for Continual Superalignment of Large Language Models https://arxiv.org/abs/2403.14683). 
% %                 \item RL-based, two work, possible application of the CRL techniques. (ref Mitigating the Alignment Tax of RLHF https://arxiv.org/abs/2309.06256, CPPO- Continual Learning for Reinforcement Learning with Human Feedback)
% %                 \item SL-based (ref COPF- Continual Learning Human Preference through Optimal Policy Fitting https://arxiv.org/pdf/2310.15694.pdf) , possible application of the supervised CL techniques.
% %             \end{itemize}
% %     \end{itemize}
% % }


% \textbf{General Introduction to Model Alignment.}
% Model Alignment (MA) is a crucial concept in the development and deployment of AI systems, ensuring that their actions and outputs align with human values, ethics, and preferences. MA can be formally defined as the process and methodologies applied to adjust the objectives and functioning of an AI system to ensure its decisions and actions are in harmony with human ethical standards and values. This process involves a combination of mathematical models, algorithmic adjustments, and iterative feedback to refine AI behavior.

% MA can be categorized broadly into two types: Reinforcement Learning-based (RL-based) and Supervised Learning-based (SL-based). The RL-based approach, as discussed in~\cite{ouyang2022rlhf}, involves training models to make decisions that are reinforced by human feedback, utilizing a reward system that guides the model towards desirable outcomes . The SL-based approach, on the other hand, as seen in~\cite{rafailov2024dpo}, involves directly training models on datasets of human preferences, aligning the model's output with the demonstrated human values . Both approaches leverage a combination of algorithmic learning techniques and human feedback to progressively refine and align model behaviors.

% % \textbf{Formal Definition of Model Alignment.} Formally, let us consider a model $g(x;\phi)$, parameterized by $\phi$ and designed to process inputs $x$ (for instance, decision-making scenarios). In the context of model alignment~\cite{lin2024mitigating}, we define an alignment dataset of size-$M$ as ${(x_a, y_a, \hat{y}_a)}^{a=1}_M$, where $y_a$ is the model's original decision for scenario $x_a$, and $\hat{y}_a$ represents the aligned decision (i.e., the decision that aligns with specified ethical guidelines or desired outcomes). Our objective is to adjust the model from $g$ to $g'$ in a way that for any given $x_a$, $g'(x_a;\phi')$ yields $\hat{y}_a$, thus aligning the model's decisions with our alignment criteria. For all other inputs $x_0 \notin {x_a}$, we seek to preserve the original behavior, such that $g(x_0;\phi) = g'(x_0;\phi')$. 





% \textbf{Continual Model Alignment (CMA).}
% Continual Model Alignment (CMA) in Large Language Models (LLMs), such as GPT-4, is a pivotal approach within the field of natural language processing (NLP), aiming to perpetually refine these models to stay in sync with the evolving landscape of human values, ethics, and data. The significance of CMA lies in its capacity to ensure that LLMs retain their relevance, accuracy, and ethical alignment over time. This ongoing adjustment is essential to navigate the complexities introduced by concept drift, the evolution of data, and shifts in societal values. The static nature of LLM training on historical data sets can lead to discrepancies between the models' outputs and current factual accuracies, societal norms, and standards, making CMA a crucial process for maintaining their adaptability and alignment with contemporary contexts \cite{taori2023alpaca}.

% Emerging strategies in CMA emphasize the optimization of LLMs for enhanced adaptability, addressing the necessity for these models to accommodate continuous changes in language use, information validity, and societal expectations. Notably, innovative research efforts such as those detailed in~\cite{lin2024mitigating}, and~\cite{puthumanaillam2024moral}, highlight the development of methodologies designed to minimize the challenges associated with the continuous realignment of LLMs. These studies underscore the importance of implementing specific optimization strategies to foster the adaptability of LLMs, ensuring their outputs remain ethically attuned and factually relevant in the face of dynamic societal and informational landscapes. %\haizhou{the background part here is repeated.}


% \textbf{RL-based CMA.}
% In the realm of RL-based CMA, two significant contributions have been noted:~\cite{lin2024mitigating} explores methodologies for reducing the overhead associated with continual alignment through efficient reinforcement learning techniques, and~\cite{zhangcppo}, which suggests a framework for applying continual learning principles to reinforcement learning with human feedback, potentially mitigating the alignment tax over time.

% \textbf{SL-based CMA.}
% For SL-based CMA,~\cite{zhang2023copf} presents an innovative approach by integrating supervised continual learning techniques with the process of aligning AI systems through direct training on evolving datasets of human preferences . This methodology promises a more sustainable and adaptable model for maintaining alignment with human values over extended periods and across various contexts.

% In conclusion, Continual Model Alignment represents a critical, ongoing challenge in the development of ethical and effective AI systems. By leveraging and innovating upon RL-based and SL-based approaches, researchers and developers can better ensure that AI systems remain aligned with human values, even as those values evolve over time.


% %separate



% CMA leverages a suite of strategies, each aimed at addressing specific aspects of model alignment:
% \begin{itemize}
%   \item \textbf{Dynamic Data Augmentation}: This approach continuously incorporates new datasets reflective of current events, slang, technological advancements, and changing societal norms into the training regime of LLMs. It ensures the models' vocabulary, knowledge base, and ethical guidelines are up-to-date.
%   \item \textbf{Feedback Loops and Human-in-the-loop (HITL)}: Implementing feedback mechanisms where model outputs are regularly reviewed by humans can correct biases, inaccuracies, and misalignments. HITL systems allow for the iterative improvement of LLMs based on human evaluations and corrections \cite{wang2022selfinstruct}.
%   \item \textbf{Reinforcement Learning from Human Preferences (RLHP)}: RLHP adjusts model outputs based on preferences expressed by humans, guiding models towards desired behaviors and away from undesirable ones. This method aligns models with evolving human values and ethics.
%   \item \textbf{Knowledge Distillation and Transfer Learning}: Techniques that enable the transfer of knowledge from newer, more aligned models to older ones, or the distillation of large-scale models into more manageable, specialized versions that can be more easily updated and maintained \cite{houlsby2019parameter}.
% \end{itemize}

% Despite its importance, CMA faces several challenges:

% \begin{itemize}
%   \item \textbf{Scalability and Resource Intensity}: Continually updating LLMs requires significant computational resources and human oversight, posing scalability issues.
%   \item \textbf{Ensuring Ethical Alignment}: Balancing model updates to reflect societal changes without introducing or perpetuating biases remains a complex issue.
%   \item \textbf{Data Privacy and Security}: Continuously integrating new data into LLMs raises concerns regarding data privacy, security, and the potential for misuse.
% \end{itemize}

% Future research in CMA aims to develop more efficient, automated processes for model updates, better mechanisms for ethical oversight, and innovative solutions to balance model relevance with privacy and security concerns. Advances in AI governance, decentralized data collection, and model auditing will play crucial roles in this endeavor.






\subsubsection{Continual Multimodal Large Language Models~(CMLLMs)}
\label{sec:cft-cmllm}
% \wenyuan{Structure of the CL for MLLM: 
%     \begin{itemize}
%         \item General Introduction to Multimodal Large Language Models.
%         \begin{itemize}
%             \item the relationship with the LLM.
%             \item major structure.
%         \end{itemize}  
%         \item the general CL type of MLLM: continual fine-tuning MLLM /  CMML . 
%             \begin{itemize}
%                 \item the reason for catastrophic forgetting and negative forward transfer()
%                 \item MLLM task in CL(instruction tuning: image classification, VQA),benchmark
%             \end{itemize}
%         \item: CL method for MLLM
%     \end{itemize}
% }

% Multi-modal LLMs integrate data of multiple modalities, like texts, images and videos to enhance real-world information comprehension. Typically, MLLMs consist of modality-specific sub-modules such as pre-trained vision encoders, large language models, and projectors for cross-model alignment. This alignment is essential for MLLMs to fuse the diverse data types and promote their comprehension. 
% For example, MiniGPT-4~\cite{zhu2023minigpt4} utilizes a linear projector to align frozen vision encoders and language models; LLaVA~\cite{liu2023visual} employs a simple linear layer to connect image features and instruction into the word embedding space. 
% Currently, all existing MLLMs are pre-trained on large scale multi-modal datasets and then fine-tuned on specific small downstream datasets, which constitutes the training process~\cite{dai2023instructblip,li2024videochat}.

Continually training multi-modal models like CLIP~\cite{radford2021learning} has been long studied~\cite{zheng2023preventing,ni2023continual}, while the problem of continually training MLLMs still remains underexplored.
Several existing studies have investigated the causes of catastrophic forgetting when continually training MLLMs. \cite{zheng2024antiforgetting} performs singular value decomposition on input embeddings, revealing a significant disparity among different input embeddings. This discrepancy causes the model to learn irrelevant information for previously trained tasks, resulting in catastrophic forgetting and negative forward transfer. 
\cite{zhai2023investigating} observes that minority collapse may lead to catastrophic forgetting, when the imbalance ratio between majority and minority classes approaches infinity during fine-tuning. It further identifies hallucination as a contributing factor to performance degradation in MLLMs. 

% \textbf{Continual Pre-Training MLLMs.}\quad 
% Training an MLLM to be updated with the changing world from scratch is resource-intensive, requiring considerable time and cost. While continual learning offers a promising solution to this problem, trivially applying past methods is not advisable, given the distinct structure of MLLMs that includes modality-specific sub-modules and cross-model alignment. Currently, there is an apparent lack of continual pre-training for MLLMs, and further exploration into continual and joint pre-training of MLLMs is necessary.

\textbf{Continual Fine-Tuning MLLMs.}\quad 
In contrast to traditional continual learning methods that involve full-model fine-tuning for new tasks, continual fine-tuning for MLLMs focuses on refining specific layers when adapting to new tasks~\cite{zhai2023investigating,he2023continual,zheng2024antiforgetting,chen2024coin,zhu2024model}. Given the strong capabilities of pre-trained models, training specific layers suffices, and can simultaneously reduce computational demands. 
\cite{zhao2024reconstruct} additionally considers an continual learning scenario, Continual Missing Modality Learning~(CMML), where different modalities are emerging throughout the incremental learning stages.
All the aforementioned studies collectively indicate that MLLMs still suffer from catastrophic forgetting, which manifests in two ways: along the direction of \emph{vertical continuity}, a performance decline on pre-trained tasks following fine-tuning for downstream tasks; and along the axis of \emph{horizontal continuity}, a performance degrade on previously fine-tuned tasks after fine-tuning for new tasks. \cite{zheng2024antiforgetting} also observes negative forward transfer, where the performance of unseen tasks degrades when learning new tasks, indicating a decline in model generalization capability.



% \textbf{Continual Learning MLLMs.}\quad
While traditional CL methods are applicable, some may not yield optimal results, as evidenced by various experiments~\cite{he2023continual,zheng2024antiforgetting}. For instance,
\cite{he2023continual} observes a consistent efficacy of replay-based and model expansion strategies across diverse scenarios of continual fine-tuning MLLMs, but regularization-based methods only perform well on models that have been jointly instruction-tuned on multiple tasks. 
Other works seek to develop ad-hoc solutions for continual learning MLLMs.
\cite{he2023continual} proposes EProj to expand the projection layer in MLLMs for each new task and utilizes task-similarity-informed regularization~(TIR) to enhance performance. \cite{zheng2024antiforgetting} introduces Fwd-Prompt, a prompt tuning method that projects prompt gradient to both the residual space and the pre-trained subspace to minimize the interference between tasks and reuse pre-trained knowledge respectively, fostering positive forward transfer without relying on previous samples. 
\cite{zhu2024model} focuses on the forgetting of the pre-trained MLLMs after fine-tuned on specific tasks and proposes model tailor to compensate the selected subset that are critical for enhancing target task performance. 
\cite{zhao2024reconstruct} presents a novel method named Reconstruct before Query~(RebQ), leveraging  the multi-modal knowledge from a pre-trained model to reconstruct the absent information for the missing modality. Recently, MoE~(Mixture-of-Experts) framework has gained attention which resembles the architecture-based methods in CL. It provides the model with
the ability to learn different intentions from distinct experts, e.g., 
\cite{chen2024coin} first introduces MoELoRA to fine-tune LLaVA, effectively mitigate the catastrophic forgetting of MLLMs in CoIN and the results demonstrate the effectiveness.

% Conclusion
% In concluding remarks on CL of MLLMs, the role of templates in instruction tuning emerges as crucial. As highlighted by \cite{chen2024coin}, employing similar templates across tasks proves more advantageous, aiding in knowledge retention and forgetting mitigation. This approach fosters task-specific learning, reducing reliance on common knowledge prone to forgetting in sequential contexts. In addition, it is noteworthy that the forgetting induced by the gap between tasks is more critical than the forgetting induced by the distributional gap between datasets. According to \cite{zhai2023investigating}, moderate fine-tuning is advantageous for non-fine-tuned tasks, excessive fine-tuning ultimately leads to catastrophic forgetting in these tasks. \cite{he2023continual} discovers the multi-task joint instruction tuning at the beginning state can facilitate the model’s continual learning ability and mitigate forgetting. 
% Overall, continual learning in MLLMs holds promise, but further research is needed to fully realize its potential.




\section{Evaluation Protocols and Datasets}
\label{sec:eval-and-data}
% \subsection{Continual LLMs' Evaluation Protocols}
\textbf{Continual LLMs' Evaluation Protocols.}\quad
LAnguage Model Analysis~(LAMA) is an evaluation framework designed to \emph{probe the world knowledge} embedded in language models~\cite{petroni2019language}.
LAMA converts each world fact into a cloze statement, which is then input into the language models to predict the correct answer. It has been extensively utilized in work on CPT under the temporal shifts~\cite{jang2022temporalwiki,jang2022towards}.
{FUAR~(Forgotten / (Updated + Acquired) Ratio)} is proposed for CPT to address the \textbf{OP}'s drawback of not able to accurately reflect the model's behavior. 
A FUAR value of 1 represents an equal trade-off between the knowledge forgetting and knowledge learning, while a FUAR less than 1 suggests high learning efficacy.
In TRACE~\cite{wang2023trace}, the authors propose a set of ``\textbf{X-Delta}'' metrics for continual instruction tuning, quantifying the forward transfer on specific abilities of LLMs, which is a straightforward extension of \textbf{FWT}. Specifically, the authors construct three sets of evaluation tasks to benchmark the ability of LLMs, including \emph{general ability}, \emph{instruction following}, and \emph{safety}. 
For more detailed introduction to these evaluation protocols, please refer to \appref{app:eval-llm}.
% In Continual Model Alignment, three prominent metrics used to evaluate different aspects of Natural Language Generation~(NLG) are BLEU-4~\cite{papineni2002bleu}, METEOR~\cite{banerjee2005meteor}, and ROUGE-L~\cite{lin2004rouge}. BLEU-4 and METEOR are mainly designed for the tasks of machine translation, while ROUGE-L is for text summarization.


\textbf{Datasets.}\quad
In this section, we provide a comprehensive review of the datasets available for benchmarking continual LLMs, as illustrated in \Tabref{tab:datasets}. 
We provide information about these datasets' types, what distributional shifts and semantic domains they include, and their sources and applications. 
We intentionally exclude datasets used for domain-adaptive pre-training LLMs in vertical domains such as legal, medical, and financial, unless they are specifically designed for continual domain-adaptive pre-training. Furthermore, we omit datasets used in general continual fine-tuning, as they have already been extensively studied in existing works~\cite{biesialska2020continual,ke2023continual}.
For details, please refer to \appref{app:data}.

\begin{sidewaystable}
% \begin{landscape}
% \begin{table*}[h]
    \centering
    \caption{
    \textbf{Summary of the existing benchmarks publicly available for Continual Learning LLMs.} 
    In the column of \textbf{Name}, we use the superscript ``$^*$'' to denote the lack of the dataset name and the name shown is that of the original paper. 
    In this table, we deliberately omit the datasets used for domain-adaptive pre-training the vertical LLMs, as their main focus of development is not on continual learning. We also omit the datasets used for general continual fine-tuning, as they are extensively discussed in other existing surveys~\cite{biesialska2020continual,ke2023continual}. 
    }
    \label{tab:datasets}
    \resizebox{1\linewidth}{!}{
\begin{tabular}{ccccc ccccc }
	\toprule[0.15em]
    \textbf{Name} & \textbf{Type} & \textbf{Shift}  & \textbf{Domain} & \textbf{\#Stages} & \textbf{Scale} & \textbf{Sources} & \textbf{Applications} & \textbf{Comment} \\
    \midrule
    \midrule
    $^*$TimeLMs~\cite{loureiro2022timelms} & CPT & Temporal & Social Media & 8 & \#Examples: 123.86M  & Tweets & \cite{loureiro2022timelms} & \href{https://github.com/cardiffnlp/timelms}{code}\\
    \midrule
    CC-RecentNews~\cite{jang2022towards} & CPT & Temporal & News & 1 & \#Tokens: $\sim$168M & Web & \cite{jang2022towards} & \href{https://github.com/joeljang/continual-knowledge-learning}{code}\\
    \midrule
    TWiki~\cite{jang2022temporalwiki} & CPT & Temporal & General Knowledge & 5 & \#Tokens: 4.7B & Wikipedia & \cite{jang2022temporalwiki} & \href{https://github.com/joeljang/temporalwiki}{code} \\
    \midrule
    $^*$DAPT~\cite{gururangan2020dont} & \makecell{CPT \\ DAP} & Content & Multi-Domain & 4 & Size: 160GB & \makecell{BioMed~\cite{lo2020s2orc}, CS~\cite{lo2020s2orc}, News~\cite{zellers2019defending}, Reviews~\cite{he2016ups}} & \makecell{\cite{gururangan2020dont} \\ \cite{qin2023recyclable} \\ \cite{qin2022elle}} & \href{https://github.com/allenai/dont-stop-pretraining}{code} \\
    \midrule
    $^*$CPT~\cite{ke2022continual-train} & CPT & Content & Multi-Domain & 4 & \#Examples: 3.12M & \makecell{Yelp~\cite{xu2019bert}, S2ORC~\cite{lo2020s2orc}, AG-News~\cite{zhang2015character}} & \cite{ke2022continual-train} & \href{https://github.com/UIC-Liu-Lab/CPT}{code} \\
    \midrule
    $^*$DEMix~\cite{gururangan2022demix} & CPT & Content & Multi-Domain & 8 & \#Tokens: 73.8B & \makecell{1B~\cite{chelba2014billion}, CS~\cite{lo2020s2orc}, Legal~\cite{caselaw2018}, Med~\cite{lo2020s2orc}\\ WebText~\cite{gokaslan2019OpenWeb}, RealNews~\cite{zellers2019defending}, Reddit~\cite{baumgartner2020pushshift}, Reviews~\cite{ni2019justifying}} & \cite{gururangan2022demix} & \href{https://github.com/kernelmachine/demix}{code} \\
    % \midrule
    % $^*$ELLE~\cite{qin2022elle} & CPT & Content & Multi-Domain & 4 & \#Tokens: $\sim$13.6B & \makecell{Bio~\cite{lo2020s2orc}, Review~\cite{he2016ups} \\ CS~\cite{lo2020s2orc}, News~\cite{zellers2019defending}} & \makecell{\cite{qin2023recyclable} \\ \cite{qin2022elle}} & \href{https://github.com/thunlp/RecyclableTuning}{code} \\
    \midrule
    $^*$DAS~\cite{ke2022continual-pre} & \makecell{CPT \\ DAP} & Content & Multi-Domain & 6 & Size: 4.16GB & \makecell{Yelp~\cite{xu2019bert}, Reviews~\cite{ni2019justifying}, Papers~\cite{lo2020s2orc}, PubMed} & \cite{ke2022continual-pre} & \href{https://github.com/UIC-Liu-Lab/ContinualLM}{code} \\
    % \cite{gupta2023continual} & CPT & Content & Mutli-Domain & 1 & - & 296.86 B & SlimPajama~\cite{cerebras2023slimpajama} & - & -\\
    \midrule
    SuperNI~\cite{wang2022supernaturalinstructions} & CIT & Content & Mutli-Domain & 16 & \makecell{\#Tasks: 1616 \\ \#Examples: $\sim$5M} & GitHub & \cite{zhang2023citb,wang2024inscl} & \href{https://github.com/allenai/natural-instructions}{code} \\

    \midrule
    CITB~\cite{zhang2023citb} & CIT & Content & Mutli-Domain & 19 & \#Tasks: 38 & SuperNI~\cite{wang2022supernaturalinstructions}  & \cite{zhang2023citb} & \href{https://github.com/hyintell/CITB}{code} \\

    \midrule
    CoIN~\cite{chen2024coin} & CIT & Content & Multi-Domain & 8 &\#Examples: $\sim$1.14M & \makecell{RefCOCO~\cite{kazemzadeh-etal-2014-referitgame},RefCOCO+~\cite{mao2016generation},RefCOCOg~\cite{mao2016generation} \\ ImageNet~\cite{imagenet_cvpr09}, VQAv2~\cite{goyal2017making}, ScienceQA~\cite{lu2022learn} \\ 
    TextVQA ~\cite{singh2019vqa}, GQA~\cite{hudson2019gqa}, VizWiz ~\cite{gurari2018vizwiz}, OCR-VQA~\cite{mishraICDAR19}} & \cite{chen2024coin} & \href{https://github.com/zackschen/CoIN}{code} \\
    \midrule
    TRACE~\cite{wang2023trace} & CIT & Content & Mutli-Domain & 8 &  \#Examples: 56,000  & \makecell{ScienceQA~\cite{lu2022learn}, FOMC~\cite{shah2023trillion}, MeetingBank~\cite{hu2023meetingbank}\\ 
    C-STANCE~\cite{zhao-etal-2023-c}, 20Minuten~\cite{kew-etal-2023-20}, CodeXGLUE~\cite{lu2021codexglue}, NumGLUE\cite{mishra2022numglue}} & \cite{wang2023trace} & \href{https://github.com/BeyonderXX/TRACE}{code} \\
    \midrule
    NATURAL-INSTRUCTION~\cite{mishra2021natural} & CIT & Content & Mutli-Domain & 6 &   \#Examples: 193k & \makecell{CosmosQA~\cite{huang2019cosmos}, DROP~\cite{dua2019drop}, Essential-Terms~\cite{khashabi-etal-2017-learning} \\ MCTACO~\cite{zhou2019goingvacationtakeslonger}, MultiRC~\cite{khashabi-etal-2018-looking}, QASC~\cite{khot2020qasc} \\ Quoref\cite{dasigi-etal-2019-quoref}~, ROPES~\cite{lin2019reasoning} , Winogrande~\cite{sakaguchi2019winogrande}} & \cite{mishra2021natural} & \href{https://github.com/allenai/natural-instructions-v1}{code} \\
    \midrule
    IMDB~\cite{maas2011learning} & CMA & Content &Social Media&1&Size: 217.35 MB& IMDB& \cite{zhang2023copf} & \href{https://huggingface.co/datasets/stanfordnlp/imdb}{code}\\
    \midrule 
    HH-RLHF~\cite{bai2022training} & CMA & Content &General Knowledge&1&Size: 28.1 MB& Human Feedback& \cite{zhang2023copf} & \href{https://github.com/anthropics/hh-rlhf}{code}\\
    \midrule
    Reddit TL;DR~\cite{volske2017tl} & CMA & Content&Social Media&2&Size: 19.6 GB&Reddit & \cite{zhang2023copf,zhangcppo} &\href{https://zenodo.org/records/1043504}{code}\\
    \midrule
    \makecell{Common Sense QA~\cite{lin2024mitigating} \\ Reading Comprehension~\cite{lin2024mitigating}\\ 
    %7787+97867+16000+107785+96000+40836876
    Translation~\cite{lin2024mitigating}} & CMA & Content &Multi-Domain&6& \#Examples: $\sim$ 41.16M&  \makecell{ARC Easy and Challenge~\cite{clark2018think}, Race~\cite{lai2017race}, PIQA~\cite{bisk2020piqa} \\ SQuAD~\cite{rajpurkar2018know}, DROP~\cite{dua2019drop} \\
    WMT 2014 French to English~\cite{bojar2014findings}
    }& \cite{lin2024mitigating} & see sources \\
    \midrule
    FEVER~\cite{fever} & CMR & Content & General Knowledge & 1 & \#Examples: 420k &  Wikipedia & \cite{de2021editing, hase2021language} & \href{https://fever.ai/resources.html}{code} \\
    \midrule
    VitaminC~\cite{vitaminC} & CMR & Content & General Knowledge & 1 & \#Examples: 450k &  Wikipedia & \cite{mitchell2022memory} & \href{https://github.com/TalSchuster/VitaminC}{code} \\
    \midrule
    zsRE~\cite{zsRE} & CMR & Content & General Knowledge & 1 & \#Examples: 120M &  Wikireading~\cite{Wikireading} & \cite{hase2021language, meng2022locating, meng2022mass, hase2023does, hartvigsen2023aging, das2024larimar} & - \\
    \midrule
    T-rex~\cite{T-rex} & CMR & Content & General Knowledge & 1 & \#Examples: 11M &  Dbpedia abstracts~\cite{Dbpedia} & \cite{li2022large, dong2022calibrating} & \href{https://hadyelsahar.github.io/t-rex/}{code} \\
    \midrule
    NQ~\cite{nq} & CMR & Content & General Knowledge & 1 & \#Examples: 320k &  Google queries, Wikipedia & \cite{hartvigsen2023aging} & \href{https://ai.google.com/research/NaturalQuestions}{code} \\
    \midrule
    CounterFact~\cite{meng2022locating} & CMR & Content & General Knowledge & 1 & \#Examples: 22k & zsRE \cite{zsRE} & \cite{meng2022locating, yu2023melo, hu2024wilke, das2024larimar} & \href{https://github.com/kmeng01/rome}{code} \\
    \midrule
    SCOTUS~\cite{scotus} & CMR & Temporal & Law & 1 & \#Examples: 9.2k &  Supreme Court Database & \cite{hartvigsen2023aging} & \href{https://github.com/coastalcph/fairlex}{code} \\
    % \multirow{2}{*}[-0.6em]{\textbf{CFT Type}} & 
    % \multirow{2}{*}[-0.6em]{\textbf{Method}} & 
    % \multirow{2}{*}[-0.6em]{\textbf{X-IL}} & 
    % \multirow{2}{*}[-0.6em]{\textbf{LLM Arch.}} & 
    % \multicolumn{4}{c}{\textbf{{Continual Learning Tech.}}} & 
    % \multicolumn{3}{c}{\textbf{{Continual Learning Eval.}}} \\
    % \cmidrule(lr){5-8}\cmidrule(lr){9-11}
    % & & & & \emph{Rehearsal} & \emph{Param. Reg.} & \emph{Arch. Exp.} & \emph{Others} & \emph{\makecell{Avg. \\Acc.}} & \emph{\makecell{Bwd. \\Trans.}} & \emph{\makecell{Fwd. \\Trans.}} \\
	\bottomrule[0.15em]
	\end{tabular}
	}
% \end{table*}
% \end{landscape}
\end{sidewaystable}

\section{Discussion}
\label{sec:discussion}
% In this section, we delve into the intersection of conventional computational patterns in continual learning and the training and deployment of large language models~(LLMs). We begin by examining intriguing properties that arise during continual learning with LLMs. Next, we explore the evolving roles of three types of incremental learning within the context of LLMs. Following this, we contrast the roles of memory in continual LLMs with those in traditional continual learning. Finally, we conclude with a concise overview of promising directions for future research in this area.


% \textbf{Intriguing Properties Emergent in Continual LLMs.}\quad
\subsection{Intriguing Properties Emergent in Continual LLMs}
\label{sec:discussion-emergent}
Beyond the well-established resilience of pre-trained large language models~(LLMs) against catastrophic forgetting compared to downstream-specific models~\cite{ke2021achieve,tao2022can,luo2023investigating,zheng2023learn,mehta2023empirical}, there is a notable lack of exploration into other intriguing properties of LLMs when trained continually. In \cite{yang2024reawakening}, it is observed that when fine-tuned sequentially and cyclically on a series of documents, large models exhibit a phenomenon known as ``\emph{anticipatory recovering}.'' This refers to the LLMs' ability to recover forgotten information on documents even before encountering them again. 
This suggests that LLMs may possess the capability of sequential memorization, which could pave the way for research into more complex structured learning environments as model parameters scale up.



% \textbf{Conventional Types of Incremental Learning.}\quad
\subsection{Conventional Types of Incremental Learning}
\label{sec:discussion-xil}
As mentioned in \Secref{sec:background-cl-types}, three types of incremental learning are prevalent~\cite{van2022three}. Among them, class-incremental learning~(CIL) has historically attracted significant attention from the community~\cite{rebuffi2017icarl,wu2019large}. However, in the context of continually pre-training and adapting large language models~(LLMs), we observe a decreased interest in CIL but an increased focus on task-incremental learning~(TIL) and domain-incremental learning~(DIL).
% A couple of reasons attribute to this drift of interests. 
Given that language models are inherently designed for content generation and are pre-trained with the pretext generative task of next-word prediction, it is natural to emphasize the patterns of generative tasks and integrate the traditional CIL paradigm into the broader framework of language modeling, discarding the incremental classification head~\cite{shao2023class,cao2024generative}. 
% For instance, in Vocabulary-Aware Label Generation~(VAG), CIL is redefined as the task of continual label generation. This approach utilizes a pre-trained encoder-decoder language model to generate class labels~\cite{shao2023class}. Meanwhile, in the Generative Multi-modal Model~(GMM) for CIL~\cite{cao2024generative}, image patches and prompts are concatenated and fed into the language model to generate classification results.
However, the declining attention to CIL does not suggest that it is not impactful in the field of continual learning for LLMs. 
% Nonetheless, many current research endeavors unwittingly employ such techniques, indicating their widespread adoption in various applications.
Techniques such as vocabulary expansion~\cite{amba2021dynamic,cossu2022continual} and learning routing function in the MoE system~\cite{chen2023lifelong} can be seen as an extension of expanding the classification head in CIL, and previously validated techniques of CIL can be directly applied.

The importance of DIL is self-evident, given the shared task definition and input-output format in continual pre-training~(CPT) and domain-adaptive pre-training~(DAP). 
% As dynamically expanding token vocabularies can pose additional challenges, it is natural to focus on understanding distributional shifts within the input corpus while keeping the vocabulary fixed. 
On the other hand, TIL attracts significant interest 
% due to its potential for personalizing LLM services. For instance, users may desire options for selecting domain-specific experts, thereby making task IDs available throughout inference time~\cite{wistuba2023continual}. 
as it plays a crucial role in instruction tuning, where instructions can be seen as natural-language-encoded task indices~\cite{scialom2022fine,huang2024mitigating,mok2023large,he2024dont,yin2022contintin,wang2023orthogonal,zhao2024sapt,wang2024inscl}.
It is worth noting that the boundary between TIL and DIL becomes somewhat blurred in continual instruction tuning. Language models demonstrate the capability to infer domain information for unseen instructions, suggesting a convergence of TIL and DIL in certain contexts.


% \textbf{Roles of Memory in Continual LLMs.}\quad
\subsection{Roles of Memory in Continual LLMs}
\label{sec:discussion-mem}
Previous continual learning research, drawing inspiration from human learning patterns, primarily emphasizes the storage efficiency of past data. However, this focus may no longer hold true in the context of continual LLMs. In the direction of relaxing memory constraints, institutions with access to training data may opt to retain full access without restricting memory size, given that the cost of memory storage is more than affordable. In such scenarios, as highlighted in \cite{verwimp2024continual}, the challenge shifts from storage efficiency to computational efficiency. To achieve continual learning goals, models must efficiently adapt to new data (efficient adaptation) and select key experiences for replay (efficient replay)~\cite{xie2023efficient,jin2024model}. Therefore, it is essential to reassess the existing memory constraint and prioritize optimizing computational efficiency for continual learning of LLMs by restricting the number of updates and FLOPs~\cite{prabhu2023computationally,wang2022sparcl}.

On the other end of the spectrum, studies with tightened memory constraints remain vital in modern continual learning of LLMs. As shown in \Figref{fig:overview}, upstream suppliers of LLMs typically do not provide training data with the released model weights. Consequently, consumers must adapt these models to downstream data without access to the actual replay data. Various rehearsal-free continual strategies are applied in this scenario, such as collecting data examples from alternate sources~\cite{rozière2024code,colombo2024saullm7b,wu2023pmc,Azerbayev2023LLEMMA}, leveraging the generative capabilities of LLMs to produce pseudo-examples for replay~\cite{qin2021lfpt5}, and implementing regularization techniques in the parameter space~\cite{ke2022continual-pre,rongali2021continual}. Continual learning under the strict memory constraint is also driven by data privacy concerns, where preserving data on the server side is prohibited. In these scenarios, researchers must rely on online continual learning methods~\cite{mai2022online,prabhu2023online}, where data examples are only utilized for training as they arrive in a stream, and numerous efforts are already underway to develop LLMs capable of operating under these constraints~\cite{bornschein2024transformers}.


% \textbf{Prospective Directions.}\quad
\subsection{Prospective Directions}
\label{sec:discussion-future}
\textbf{Theories of Continual LLMs.}\quad 
It is widely recognized that the continual learning community tends to prioritize empirical research over theoretical exploration. Nevertheless, there are efforts to establish theoretical foundations for CL. In \cite{wang2024comprehensive}, the authors utilize second-order Taylor expansions around optimal parameters to derive an inter-task generalization error bound based on the maximum eigenvalue and $l_2$-norm of parameter differences. Another line of approaches leverages task/domain discrepancies to construct a multi-task generalization bound. For instance, Unified Domain Incremental Learning~(UDIL) in \cite{shi2024unified} proposes upper bounds for intra-domain and cross-domain distillation losses, unifying various replay-based DIL techniques under a single adaptive generalization bound. However, applying these existing theories directly to continual LLMs can be imprudent, given their pre-trained, large-scale nature. Consequently, there is a notable gap in research focusing on continually learning LLMs with robust theoretical guarantees and understanding the forgetting behaviors of LLMs from a theoretical perspective.



\textbf{Efficient Replay for Knowledge Retention for Continual LLMs.}\quad
% Computational resources for training large-scale LLMs are often limited. 
While the storage budget can theoretically be infinite (\Secref{sec:discussion-mem}), replaying past experiences without specific design can lead to inefficient updates in current domain learning, resulting in slow convergence. Beyond sparse replay solutions that control data mixture ratios \cite{lin2023geogalactica,rozière2024code,Yang2023PLLaMa}, there is ongoing exploration of efficient replay for continual LLMs. 
For example, KPIG~\cite{he2024dont} enhances replay efficiency by calculating Key-Part Information Gain~(KPIG) on masked segments, enabling the dynamic selection of replay data. 
\cite{jin2024model} introduces a forgetting forecasting mechanism based on output changes during adaptation, later used for selective replay in continual model refinement~(CMR). 
More sophisticated and accurate data mixing strategies and efficient replay sample selection mechanisms are needed and hence we mark it as a significant research focus in the future.



\textbf{Continual LLMs with Controllable Memory.}\quad
The long-term memory inherent in the whole set of parameters of LLMs often lacks interpretability and explicit manipulability, which is crucial in certain application areas such as machine unlearning~\cite{bourtoule2020machine}, where the continually pre-trained models need to constantly roll back to a previous version predating the inclusion of the revoked data and retrain the model from that point onward.
This example illustrates the benefits of equipping LLMs with an external, controllable memory.
As part of continual model refinement~(CMR), memory systems for continual learning have been explored in several studies. Larimar~\cite{das2024larimar} suggests integrating the Kanerva Machine~\cite{wu2018kanerva} as an episodic memory for multi-fact model editing. This memory system supports basic operations like \emph{writing, reading, and generating}, as well as advanced operations such as \emph{sequential writing and forgetting}. It enables one-shot knowledge updates without costly retraining or fine-tuning. Other memory systems like Hopfield Networks~\cite{ramsauer2021hopfield} hold promise for future investigation as well.

\textbf{Continual LLMs with Custom Preferences.}\quad
In service-oriented contexts, users often require different trade-offs between domain expertise, ethics, values, or tones of expression. Efficiently building customized LLMs for individual users and offering flexible adjustment options is a challenging task. Early attempts in this direction include Imprecise Bayesian Continual Learning~(IBCL), which, under certain assumptions, guarantees the generation of Pareto-optimal models based on user preferences by combining two model posteriors in the parameter space~\cite{lu2023ibcl}. While empirical validation is limited in scale, this approach paves the way for future research in this area.

\section{Conclusion}
\label{sec:conclusion}
In this work, we offer a comprehensive survey on continual LLMs, summarizing recent advancements in their training and deployment from a continual learning standpoint. We categorize the problems and tasks based on their positions within our proposed broader framework of modern stratified continual learning of LLMs. While there is a widespread and growing interest in this area across the community, we also note several missing cornerstones, including algorithmic diversity and a fundamental understanding of large models' behaviors such as knowledge forgetting, transfer, and acquisition. With a holistic yet detailed approach, we aim for this survey to inspire more practitioners to explore continual learning techniques, ultimately contributing to the development of robust and self-evolving AI systems.


% In this work, we present a comprehensive survey on continually LLMs, summarizing the recent advances on training and deploying from a continual learning perspective. 
% We categorize the problems and tasks of continual learning LLMs based on the positions they belong to under the big picture of modern stratified continual learning LLMs.  
% We observe wide and increasing interests about this topic spread across the community, while we also see some missing corner stones in this field such as the diversity of the algorithms and the fundamental understanding of the large models' behavior of knowledge forgetting, knowledge transfer, and knowledge acquisition. 
% Holistic but meanwhile detailed, we hope this survey will encourage and include more practitioners to study continual learning techniques that can eventually boost the rising of a powerful and reliable AI system that can self-evolve throughout the time. 


% \newpage
% \appendix

% \begin{landscape}
% [insert table here that will be displayed horizontally]
% \end{landscape}


% \end{sidewaystable}

% \begin{sidewaystable} % final form of the table.
% \begin{landscape}
% \begin{table*}[t]
%     \centering
%     \caption{
%     \textbf{Summary of the existing studies on Horizontal Continual Domain-Adaptive Pre-Training of LLMs,} where the papers are organized in the chronological order.
%     1. Domain(s): use ``/'' and ``$\rightarrow$'' to denote the DAP is performed in parallel and sequentially to different domains, respectively. 
%     }
%     \label{tab:pre-training-big}
%     \resizebox{1\linewidth}{!}{%
% \begin{tabular}{lcccc ccccc ccccc cccc}
% 	\toprule 
% 	\multirow{3}{*}{\textbf{Method}} & 
%     \multirow{3}{*}{\textbf{Domain(s)}} & 
%     \multirow{3}{*}{\textbf{LLM Arch.}} & 
%     \multicolumn{10}{c}{\textbf{{Continual Learning Techniques}}} & 
%     \multicolumn{6}{c}{\textbf{{Continual Learning Evaluation}}} \\
%     \cmidrule(lr){4-13}\cmidrule(lr){14-19}
%     & & & \multicolumn{3}{c}{\emph{Rehearsal}} &  \multicolumn{2}{c}{\emph{Parameter Regularization}} &  \multicolumn{5}{c}{\emph{Architecture Expansion}} & \multicolumn{3}{c}{\emph{Backward Transfer}} & \multicolumn{3}{c}{\emph{Forward Transfer}} \\
%     \cmidrule(lr){4-6}\cmidrule(lr){7-8}\cmidrule(lr){9-13}\cmidrule(lr){14-16}\cmidrule(lr){17-19}
%     & & & \small{ER} & \small{KD} & \small{Pseudo-ER} & \small{\makecell{Parameter \\ Freezing}} & \small{\makecell{Imp.-based \\ Regularization}} & \small{\makecell{Vocab. \\ Expansion}} & \small{\makecell{Prompt \\ Tuning}} & \small{\makecell{Layer \\ Expansion}} & \small{LoRA} & \small{Adapter} & \small{\makecell{Perplexity / \\ Loss / Acc.}} & \small{ZS / FS} & \small{FT} & \small{\makecell{Perplexity / \\ Loss / Acc.}} & \small{ZS / FS} & \small{FT}\\
% 	%  &
%     \midrule
%     \midrule
% 	\cite{yan2023af} & BioMed & RoBERTa & \xmark & \xmark & \xmark & \cmark\club & \xmark & \xmark & \xmark & \cmark & \club & \xmark & $\mathcal{D}_0$ & \xmark & \xmark & \cmark & \xmark & \cmark \\
%     \midrule
%     \cite{rongali2021continual} & BioMed & \makecell{BERT \& \\ RoBERTa \& \\ DistilBERT} & \club & \xmark & \xmark  & \xmark & \club & \xmark & \xmark & \xmark & \xmark & \xmark & $\mathcal{D}_0$ & \xmark & $\mathcal{D}_0$ & \cmark & \xmark & \cmark \\
%     \midrule
%     \cite{guo2023continuous} & Medical & Llama2 & \xmark & \xmark & \xmark & \xmark & \xmark & \xmark & \xmark & \xmark & \xmark & \xmark & $\mathcal{D}_0$ & $\mathcal{D}_0$ & \xmark & \xmark & \cmark & \cmark \\
%     \midrule
%     \cite{gururangan2020dont} & \makecell{CS / News / \\ Reviews / BioMed} & RoBERTa &  \xmark & \xmark & \xmark & \xmark & \xmark & \xmark & \xmark & \xmark & \xmark & \xmark & $\mathcal{D}_0$ & \xmark & \xmark & \cmark & \xmark & \cmark \\
%     \midrule
%     \cite{ma2023ecomgptct} & E-Commerce & BLOOM & \cmark & \xmark & \xmark & \xmark & \xmark & \xmark & \xmark  & \xmark & \xmark & \xmark  & \xmark & $\mathcal{D}_0$ & \xmark & \xmark & \cmark & \cmark \\
%     \midrule
%     \cite{han2021econet}$^1$ & TemporalEvent & \makecell{BERT \& \\ RoBERTa} & \xmark & \xmark & \xmark & \xmark & \xmark & \xmark & \xmark &\xmark  & \xmark & \xmark & \xmark & \xmark & \xmark & \xmark & \xmark & \cmark \\
%     \midrule
%     \cite{xie2023efficient} & Financial & Pythia & \xmark & \xmark & \xmark & \xmark & \xmark & \xmark & \xmark & \xmark & \xmark & \xmark & $\mathcal{D}_0$& $\mathcal{D}_0$ & \xmark & \cmark & \cmark & \xmark \\
%     \midrule
%     \cite{zhou2020pre} & CommonSense & T5 & \xmark & \xmark & \xmark & \xmark & \xmark & \xmark & \xmark & \xmark  & \xmark & \xmark & \xmark & \xmark & \xmark & \xmark & \cmark & \cmark \\
%     \midrule
%     \cite{xie2023quert} & Travel & \makecell{BERT \& \\ ERNIE} & \xmark & \xmark & \xmark & \xmark & \xmark & \xmark & \xmark & \xmark  & \xmark & \xmark & \xmark & \xmark & \xmark & \xmark & \xmark & \cmark \\
%     % \midrule
%     % \cite{zhang2023revisit} & & & & & & & & & & & & & & & & & & 1 \\ % this pushes the limit of direct fine-tuning. 
%     % \midrule
%     % \cite{jin2024model} & & & & & & & & & & & & & & & & & & 1 \\
% 	\bottomrule
% 	\end{tabular}
% 	}
% \end{table*} 
% \footnotetext[1]{\cite{han2021econet} adopts \emph{small learning rate} and \emph{selective data sampling} to successfully avoid forgetting of general knowledge.}
% \end{landscape}
% \end{sidewaystable}

% continual fine-tuning.
% \cite{liu2020exploring} & & & & & & & & & & & & & & & & & & 1 \\

{
% \bibliographystyle{alpha}
\bibliographystyle{abbrv}
\bibliography{main}
}

% \end{document}
\appendix
\clearpage
\section*{\LARGE Supplementary Material}


\section{Preliminaries}
\label{app:preliminary}
In this section, we provide an overview of the fundamental concepts of large language models (LLMs) and continual learning~(CL)
% , ensuring clarity and comprehensibility for readers unfamiliar with these topics. 
We begin by introducing the notation used in this paper. Subsequently, we discuss the pre-training and downstream adaptation of LLMs, as well as mainstream LLM families (\appref{app:preliminary-llm}), followed by an introduction to basic continual learning techniques studied by the community (\appref{app:preliminary-cl}).


\textbf{Notation.}\quad
We denote scalars with lowercase letters, vectors with lowercase boldface letters, and matrices with uppercase boldface letters.
The $l_2$-norm of vectors and the Frobenius norm of a matrix are represented by $\|\cdot\|_2$. For a vector $\vv = [v_1, v_2, \cdots, v_n]^\top$, $\|\vv\|_2 = (\sum_{i=1}^{n} v_i^2)^{\nicefrac{1}{2}}$; for a matrix $\mA\in \mathbb{R}^{m\times n}$, $\|\mA\|_2 = (\sum_{ij} A_{ij}^2)^{\nicefrac{1}{2}}$.
We use $\epsilon_{\gD}$, $\gL_{\gD}$ to denote the error function, and loss function that is deployed for training, respectively, where the subscript is used to denote the error/loss measured by taking the expectation on the data distribution $\gD$. 
We further use $\hat{\gL}_{S}$ to represent the empirical evaluation of the loss function $\gL$ over the set of examples $S$.
Probability and expectation are denoted by $P$ and $\E$, respectively. 
We use $[m]$ to denote the set of positive integers up to $m$, $\{1, \cdots, m\}$.



\subsection{Large Language Models}
\label{app:preliminary-llm}
In the past two decades, neural language modeling has emerged as the dominant field of deep learning, marked by significant and rapid advancements. Primarily built on the transformer architecture, pre-trained language models~(PLMs) like BERT have established a universal hidden embedding space through extensive pre-training on large-scale unlabeled text corpora. 
Following the pre-training and fine-tuning paradigms, PLMs exhibit promising performance across various natural language processing tasks after being fine-tuned upon small amounts of task-specific data~\cite{devlin2018bert, liu2019roberta, raffel2020exploring}.
Research on scaling laws indicates that increasing model size enhances the capacity of language modelss~\cite{kaplan2020scaling, hoffmann2022training}.
By scaling parameters to billions or even hundreds of billions and training on massive text datasets, PLMs not only demonstrate superior language understanding and generation capabilities but also manifest emergent abilities such as in-context learning, instruction following, and multi-step reasoning, which are absent in small-scale language models like BERT~\cite{wei2022chain,wei2022emergent,yao2024tree,wei2021finetuned,min2022rethinking}. 
These larger models are commonly referred to as Large Language Models~(LLMs).


\subsubsection{Pre-Training of LLMs}
\label{app:preliminary-lmm-pt}
Pre-training is essential for language models to acquire broad language representations. Decoder-only models typically employ probability language modeling (LM) tasks during pre-training. LM, in this context, specifically refers to auto-regressive LM.
Given a sequence of tokens $\vx = [x_1, x_2, \cdots, x_N ]$, LM predicts the next token $x_t$ autoregressively based on all preceding tokens $\vx_{<t} = [x_1, x_2, \cdots, x_{t-1}]$, and trains the entire network by minimizing the negative log-likelihood:
\begin{align}\label{lm}
    \mathcal{L}_{{\rm LM}}(\vx) &\triangleq -\sum^N_{t=1} \log P( x_t | \vx_{<t} ), 
\end{align}
where $P(x_1|\vx_{<1})\triangleq P(x_1)$ is the unconditional probability estimation of the first token.
The three most popular families of decoder-only models are GPT, PaLM, and LLaMA. The GPT family, developed by OpenAI, includes models such as GPT-2~\cite{radford2019language}, GPT-3 ~\cite{brown2020language}, ChatGPT~\cite{achiam2022chatgpt}, and GPT-4 ~\cite{achiam2023gpt}. Notably, GPT-3 was the first LLM to exhibit emergent abilities not found in smaller PLMs. Another notable family, Gemini, developed by Google, is comparable to the GPT family~\cite{team2023gemini, reid2024gemini}. While both GPT and Gemini families are closed-source, LLaMA, released by Meta, is currently the most popular open-source family of LLMs~\cite{touvron2023llama, touvron2023llama2}. The weights of these models are made available to the research community under non-commercial licenses.

Masked language modeling (MLM) task serves as a common pre-training objective for encoder-only models like BERT~\cite{devlin2018bert,liu2019roberta}. 
In MLM, for the input sequence $\vx$, a subset of input tokens $m(\vx)$ are masked and replaced with the special [MASK] token. The pre-training goal is to utilize the unmasked parts $\vx_{\backslash m(\vx)}$ to predict the masked portions $m(\vx)$. 
In summary, the overarching goal of MLM is to minimize the negative log-likelihood:
\begin{align}\label{mlm}
    \mathcal{L}_{{\rm MLM}}(\vx) &\triangleq -\sum_{\hat{x} \in m(\vx)}{\rm log} \, P( \hat{x}|\vx_{\backslash m(\vx)} ).
\end{align}

Some encoder-decoder architecture models, such as T5~\cite{raffel2020exploring}, also utilize Sequence-to-Sequence MLM task as the pre-training objective. They take masked sentences as encoder inputs and utilize the decoder to sequentially predict the masked tokens.


\subsubsection{Adaptation of LLMs}
\label{app:preliminary-llm-adaptation}
LLMs are primarily trained to generate linguistically coherent text. However, this training may not align with human values, preferences, or practical needs. Furthermore, the pre-training data can be outdated, leading to knowledge cutoffs or inaccuracies. To address these issues, various computational paradigms such as Instruction Tuning~(IT)~\cite{zhang2024instruction}, Model Refinement~(MR)~\cite{de2021editing}, and Model Alignment~(MA)~\cite{ouyang2022rlhf,rafailov2024dpo} have been proposed. These approaches adapt LLMs to better meet diverse downstream tasks and user requirements.

\begin{definition}[\textbf{Instruction Tuning, IT}]\label{def:it}
    % Instruction Tuning involves fine-tuning LLMs on a dataset comprised of instructional prompts and their corresponding desired responses, which can be denoted as $D = \{(x_{text}, x_{instruct}, y_t)_{j=1}^n\}$, where $(x_{text}, x_{instruct}, y_t)_j$ is $j^{th}$ pair of the instruction and desired output pair.  This dataset is used to fine-tune the existing LLM $f(x_{text};\theta)$ with parameter $\theta$, thereby enabling the model to better perform specific tasks.
    Let $h(\vx)$ be a language model that takes as input data $\vx$, typically consisting of natural language instructions or queries. Instruction Tuning (IT) is a specialized training approach designed to enhance the model's ability to accurately and effectively respond to specific instructions. The objective of IT is to refine $h$ by adjusting its parameters using a designated set of training examples $\gI = \{(\vx_i, \hat{\vy}_i)\}_{i=1}^N$ drawn from the IT data distribution $\gD_\gI$, where $\hat{\vy}_i$ represents the desired output for $\vx$. This set is curated to target specific tasks or functionalities that require improved performance. 
    % The tuning process involves an iterative adjustment of the model’s weights to minimize a loss function that measures the discrepancy between the model's predictions and the desired outputs in \( \mathcal{E} \). 
    Formally, IT seeks to find an optimal refined hypothesis $h^*$ that satisfies:
    \begin{align}
    \label{eq:it}
        h^* &\triangleq \arg\min_{h^\prime} \E_{(\vx, \vy) \sim \gD_\gI} \left[ -\log P(\hat{\vy}|\vx, h^\prime) \right] 
        \approx \arg\min_{h^\prime} \sum_{i=1}^N -\log P(\hat{\vy}_i|\vx_i, h^\prime).
    \end{align}
    % \begin{align}
    % h'(\mathbf{x}) = 
    % \begin{cases}
    %     \hat{\mathbf{y}} & \text{if } \mathbf{x} \in \mathcal{E}, \\
    %     h(\mathbf{x}) & \text{o.w.}.
    % \end{cases}
    % \]
% The goal is to ensure that \( h' \) achieves higher accuracy and better performance on the tasks defined by \( \mathcal{E} \) while maintaining its original capabilities on inputs not covered by \( \mathcal{E} \).
\end{definition}

\begin{remark}
    The task of Model Alignment~(MA) is usually formulated in the same problem definition as IT, with an alignment dataset of size $M$ as $\gA = \{(\vx_a, \vy_a, \hat{\vy}_a)\}_{a=1}^M$, where $\vy_a$ represents the model's original decision for input $\vx_a$, and $\hat{\vy}_a$ denotes the aligned decision that adheres to specified ethical guidelines or desired outcomes.
\end{remark}
    
\begin{definition}[\textbf{Model Refinement, MR}]\label{def:mr}
    Suppose we have a model $h(\vx)$ taking data $\vx$ (e.g., natural language queries) as inputs. Consider a size-$N$ editing set $\gE = \{(\vx_e, \vy_e, \hat{\vy}_e)\}_{e=1}^N$, where $\hat{\vy}_e$ denotes the true label of $\vx_e$, but the model incorrectly outputs $\vy_e$ for $\vx_e$. Model Refinement~(MR) aims to efficiently update the model from $h$ to $h^\prime$ such that it correctly predicts the editing set $\gE$, while preserving the original outputs outside $\gE$. Formally, we aims to find $h^\prime$ satisfying
    \begin{align}
    \label{eq:mr}
        h^\prime(\vx_0) = 
        \begin{cases}
            \hat{\vy}_0 & \text{if } (\vx_0, \hat{\vy}_0) \in \gE, \\
            % h(\vx_0) & \text{if } (\vx_0, \vy_0). \notin \gE
            h(\vx_0) & o.w.
        \end{cases}
    \end{align}
    % such that it correctly predicts $\hat{\vy}_e$. For other inputs $\vx_0 \notin \{\vx_e\}$, we desire $h(\vx_0) = h^\prime(\vx_0)$. This is the basic problem setting of model editing.
\end{definition}


% \begin{definition}[\textbf{Model Alignment, MA}]\label{def:ma}
% Consider a model $g(x; \phi)$, parameterized by $\phi$, designed to process inputs $x$ in decision-making scenarios. Define an alignment dataset of size $M$ as $\{(x_a, y_a, \hat{y}_a)\}_{a=1}^M$, where $y_a$ represents the model's original decision for input $x_a$, and $\hat{y}_a$ denotes the aligned decision that adheres to specified ethical guidelines or desired outcomes. The objective is to modify $g$ into $g'$ such that for any $x_a$ in the alignment dataset, $g'(x_a; \phi')$ yields $\hat{y}_a$, aligning the model's decisions with the alignment criteria. For all other inputs $x_0 \not\in \{x_a\}$, the goal is to preserve the original behavior, ensuring $g(x_0; \phi) = g'(x_0; \phi')$.
% \end{definition}

% \begin{definition}[\textbf{Model Alignment, MA}]\label{def:ma}
% Consider a model $h(\vx)$ designed to process inputs $\vx$ in decision-making scenarios. Define an alignment dataset of size $M$ as $\gA = \{(\vx_a, \vy_a, \hat{\vy}_a)\}_{a=1}^M$, where $\vy_a$ represents the model's original decision for input $\vx_a$, and $\hat{\vy}_a$ denotes the aligned decision that adheres to specified ethical guidelines or desired outcomes. The objective of Model Alignment~(MA) is to modify $h$ into $h^\prime$ such that for any $\vx_a$ in the alignment dataset, $h^\prime(\vx_a)$ yields $\hat{\vy}_a$, aligning the model's decisions with the alignment criteria. Formally, 
%     \begin{align}
%     \label{eq:ma}
%         h^\prime(\vx_0) = \hat{\vy}_0, \quad \forall (\vx_0, \hat{\vy}_0) \in \gA.
%         % \begin{cases}
%         %     \hat{\vy}_0 & \text{if } (\vx_0, \hat{\vy}_0) \in \gA, \\
%         %     % h(\vx_0) & \text{if } (\vx_0, \vy_0). \notin \gE
%         %     h(\vx_0) & o.w.
%         % \end{cases}
%     \end{align}
% \end{definition}

% \begin{remark}
% It is still an open problem to include the constraint of preventing catastrophic forgetting of the general knowledge for IT, and reducing the Alignment Tax~\cite{lin2024mitigating} in the optimization objective of MA. 
% A simple extension from the constraint of model refinement in \Eqref{eq:mr}, $h^\prime(\vx_0) = h(\vx_0), \forall (\vx_0, \hat{\vy}_0) \notin \gA$, might be too strong in this case, as we certainly want the preference represented by $\gA$ can generalize to other similar while not the same inputs. 
% \end{remark}

% Instruction tuning and model alignment are two major methods for further adapting pre-trained LLMs to specific objectives. After pre-training, LLMs are tuned on a collection of formatted instances composed of task descriptions and corresponding outputs, a method called instruction tuning. This approach can enhance model performance~\tocite, improve generalization to unseen tasks~\tocite, and enable the model to acquire domain-specific knowledge, such as in medicine, finance, and law~\tocite. 
% % The model alignment refers to aligning the behavior of the model with human values, preferences, and principles, thereby meeting human expectations. 
% The model alignment refers to utilizing human feedback to iteratively modify the model, aligning its behavior with human values, preferences, and principles, in order to meet human expectations.
% % RLHF (Reinforcement Learning from Human Feedback)~\cite{christiano2017deep} is a popular approach that utilizes a reward model to provide guidance signals learned from human feedback, and employs reinforcement learning algorithms such as PPO (Proximal Policy Optimization)~\cite{schulman2017proximal} to optimize pre-trained model using signals from the reward model. 
% Through instruction tuning and model alignment, ChatGPT's powerful abilities in instruction-following and conversation have sparked discussions and research trends at a phenomenon level.




% \subsection{Large Language Models}
% \label{app:preliminary-llm}
% Neural language modeling has become the mainstream approach in the field of language modeling in the past two decades, achieving a series of rapid and significant advancements. Mainly based on the transformer architecture, pre-trained language models (PLMs) represented by BERT learn a universal hidden embedding space through pre-training on web-scale unlabeled text corpora~\cite{devlin2018bert, liu2019roberta, raffel2020exploring}. Following the pre-training and fine-tuning paradigms, PLMs have shown promising performance on a variety of natural language processing tasks after fine-tuning on small amounts of task-specific data. Research on scaling law indicates that model scaling can improve the capacity of language models~\cite{kaplan2020scaling, hoffmann2022training}. By scaling parameters to billions or even hundreds of billions and training on massive text data, PLMs not only demonstrate stronger language understanding and generation capabilities, but also surprisingly exhibit emergent abilities such as in-context learning, instruction following, and multi-step reasoning that small-scale language models like BERT do not have~\cite{wei2022emergent}. Such models are referred to as Large Language Models (LLMs).

% % GPT, PaLM, and LLaMA are currently the three most popular LLMs families. The GPT family, developed by OpenAI, includes various models such as GPT-2~\cite{radford2019language}, GPT-3~\cite{brown2020language}, ChatGPT~\cite{achiam2022chatgpt}, and GPT-4~\cite{achiam2023gpt}.
% % GPT-3 is the first LLM to demonstrate emergent abilities not present in smaller PLMs. ChatGPT (Chat Generative Pre-trained Transformer) is further fine-tuned through Reinforcement Learning from Human Feedback (RLHF) to better follow human instructions. Its powerful abilities in instruction-following and conversation have sparked a phenomenon-level discussion and research trend. 
% % PaLM (Pathways Language Model) family, developed by Google, is another LLMs family comparable to the GPT family~\cite{chowdhery2023palm, anil2023palm}. 
% % The two LLM families mentioned above are closed-source, while LLaMA released by Meta is currently the most popular open-source LLM family~\cite{touvron2023llama, touvron2023llama2}. The model weights are released to the research community under non-commercial licenses.

% % The model weights are released to the research community under non-commercial licenses, facilitating rapid development with support from the research community.

% \textbf{Pre-Training of LLMs.}\quad
% Pre-training plays a crucial role in enabling language models to learn general language representations. 
% % Decoder-only models, such as GPT, PaLM, and LLaMA, typically utilize probability language modeling (LM) task as their objective of pre-training. 
% Decoder-only models typically utilize probability language modeling (LM) task as their objective of pre-training. 
% LM generally refers to auto-regressive LM. Given a text sequence $\vx = [x_1, x_2, \cdots , x_N ]$, LM predicts the target token $x_t$ autoregressively based on all preceding tokens before the target token $\vx_{<t} = [x_1, x_2, \cdots, x_{t-1}]$, and trains the entire network minimizing the negative log-likelihood:
% \begin{equation}\label{lm}
% \mathcal{L}_{{\rm LM}}(\vx) = -\sum^N_{t=1} \log P( x_t | \vx_{<t} ), 
% \end{equation}
% where $P(x_1|\vx_{<1})\triangleq P(x_1)$ is the unconditional probability estimation of the first token. 

% The representatives of decoder-only models are the three current most popular LLMs families: GPT, PaLM, and LLaMA.
% The GPT family, developed by OpenAI, includes various models such as GPT-2~\cite{radford2019language}, GPT-3~\cite{brown2020language}, ChatGPT~\cite{achiam2022chatgpt}, and GPT-4~\cite{achiam2023gpt}.
% GPT-3 is the first LLM to demonstrate emergent abilities not present in smaller PLMs. 
% % ChatGPT (Chat Generative Pre-trained Transformer) is further fine-tuned through Reinforcement Learning from Human Feedback (RLHF) to better follow human instructions. Its powerful abilities in instruction-following and conversation have sparked a phenomenon-level discussion and research trend. 
% PaLM (Pathways Language Model) family, developed by Google, is another LLMs family comparable to the GPT family~\cite{chowdhery2023palm, anil2023palm}. 
% The two LLMs families mentioned above are closed-source, while LLaMA released by Meta is currently the most popular open-source LLMs family~\cite{touvron2023llama, touvron2023llama2}. The model weights are released to the research community under non-commercial licenses.

% Masked language modeling (MLM) task is widely used as pre-training objective for encoder-only models, such as BERT. In MLM, parts of the input sequence tokens are masked, denoted as $m(\vx)$, and the unmasked parts $\mathbf{x}_{\backslash m(\vx)}$ are used to predict the masked parts. Similar to the LM, the general objective is to minimize the negative log-likelihood:
% \begin{equation}\label{mlm}
% \mathcal{L}_{{\rm MLM}}(\vx) = -\sum_{\hat{x} \in m(\vx)}{\rm log} \, P( \hat{x}|\vx_{\backslash m(\vx)} ).
% \end{equation}

% Some encoder-decoder architecture models, such as T5~\cite{raffel2020exploring}, also utilize Sequence-to-Sequence MLM task as the pre-training objective. They take masked sentences as encoder inputs and utilize the decoder to sequentially predict the masked tokens.

% \textbf{Adaptation of LLMs.}\quad

% \haizhou{Brief Intro to the downstream adaptation.}

% Instruction tuning and model alignment are two major methods for further adapting pre-trained LLMs to specific objectives. After pre-training, LLMs are tuned on a collection of formatted instances composed of task descriptions and corresponding outputs, a method called instruction tuning. This approach can enhance model performance~\tocite, improve generalization to unseen tasks~\tocite, and enable the model to acquire domain-specific knowledge, such as in medicine, finance, and law~\tocite. 
% % The model alignment refers to aligning the behavior of the model with human values, preferences, and principles, thereby meeting human expectations. 
% The model alignment refers to utilizing human feedback to iteratively modify the model, aligning its behavior with human values, preferences, and principles, in order to meet human expectations.
% % RLHF (Reinforcement Learning from Human Feedback)~\cite{christiano2017deep} is a popular approach that utilizes a reward model to provide guidance signals learned from human feedback, and employs reinforcement learning algorithms such as PPO (Proximal Policy Optimization)~\cite{schulman2017proximal} to optimize pre-trained model using signals from the reward model. 
% Through instruction tuning and model alignment, ChatGPT's powerful abilities in instruction-following and conversation have sparked discussions and research trends at a phenomenon level.

\subsection{Continual Learning}
\label{app:preliminary-cl}
% Contemporary machine learning models differ from human learning processes. 
Humans gradually accumulate knowledge and skills across tasks without significant performance decline on previous tasks~\cite{mcclelland1995there,kandel2000principles,pallier2003brain,mccaffary2021towards}. In contrast, machine learning models are usually data-centric, minimizing the training loss on the subsequent tasks will cause the model fail on the old ones, which phenomenon is phrased as \emph{``catastrophic forgetting''}. Addressing this challenge is a focal point in continual learning research. The problem of efficiently adapting models to a sequence of tasks without forgetting is extensively studied in the continual learning community~\cite{pentina2016theoretical,chen2018lifelong,van2022three,wang2024comprehensive}. These studies are typically conducted under the following memory constraint of CL.

% Memory constraints definition
\begin{definition}[\textbf{Memory Constraint of Continual Learning}]\label{def:memory}
Suppose $T$ sets of observations $\{S_t\sim \gT_t\}_{t=1}^T$ come in as a sequence, where $\{\gT_t\}_{t=1}^T$ denotes the $T$ task distributions . At the learning stage $t>1$, the sets of observations $\{S_i\}_{i=1}^{t-1}$ are not accessible (\textbf{strong}) or only partially accessible (\textbf{relaxed}).
\end{definition}

\begin{remark}
    In early stages of CL, works mostly focused on the strong memory constraint~\cite{kirkpatrick2017overcoming,li2017learning,aljundi2018memory,lomonaco2020rehearsalfree}; as the research field progresses, more focus was put on relaxing the memory constraint to a small buffer for replay~\cite{rebuffi2017icarl,chaudhry2019tiny,buzzega2020dark,shi2024unified}; some modern CL works completely discard the memory constraint but put focus on the computational budget~\cite{prabhu2023online,verwimp2024continual}.
\end{remark}


\subsubsection{Three Types of Continual Learning}
There are three outstanding types of continual learning scenarios: task-incremental learning~(TIL), domain-incremental learning~(DIL), and class-incremental learning~(CIL).
To establish a groundwork for subsequent discussions (as illustrated in \Tabref{tab:cft} and \Secref{sec:discussion-xil}), we adhere to the conceptual framework proposed by \cite{van2022three,kim2022theoretical,wang2024comprehensive} and offer formal definitions for these three continual learning scenarios.

% TIL definition
\begin{definition}[\textbf{Task-Incremental Learning, TIL}]\label{def:til}
Suppose $T$ task distributions $\{\gT_t\}_{t=1}^T$ come in as a sequence, where $\gT_t$ denotes the joint distribution over the $t$-th task's input space and the label space $(\gX_t, \gY_t)$. Denote $\gX \triangleq \bigcup_{t=1}^T \gX_t$ and $\gY \triangleq \bigcup_{t=1}^T \gY_t$ as the union of the input and label spaces, respectively.
Under the memory constraint defined in \Defref{def:memory}, Task-Incremental Learning~(TIL) aims to find the optimal hypothesis $h^*: \gX \times [T] \rightarrow \gY$ that satisfies:
\begin{align}
    h^* &= \arg\min_{h} \sum_{t=1}^{T} \E_{(\vx, y)\sim \gT_t} \left[ \mathbbm{1}_{h(\vx, t)\neq y} \right].
\end{align}
\end{definition}



\begin{definition}[\textbf{Domain-Incremental Learning, DIL}]\label{def:dil}
Suppose $T$ domain distributions $\{\gD_t\}_{t=1}^T$ come in as a sequence, where $\gD_t$ denotes the $t$-th joint distribution over the shared input space and label space $(\gX, \gY)$. 
Under the memory constraint defined in \Defref{def:memory}, Domain-Incremental Learning~(DIL) aims to find the optimal hypothesis $h^*: \gX \rightarrow \gY$ that satisfies:
\begin{align}
    h^* &= \arg\min_{h} \sum_{t=1}^{T} \E_{(\vx, y)\sim \gD_t} \left[ \mathbbm{1}_{h(\vx)\neq y} \right].
\end{align}
\end{definition}


\begin{definition}[\textbf{Class-Incremental Learning, CIL}]\label{def:cil}
Suppose $T$ task distributions $\{\gT_t\}_{t=1}^T$ come in as a sequence, where $\gT_t$ denotes the joint distribution over the $t$-th task's input space and the label space $(\gX_t, \gY_t)$. Denote $\gX \triangleq \bigcup_{t=1}^T \gX_t$ and $\gY \triangleq \bigcup_{t=1}^T \gY_t$ as the union of the input and label spaces, respectively.
Under the memory constraint defined in \Defref{def:memory}, Class-Incremental Learning~(CIL) aims to find the optimal hypothesis $h^*: \gX \rightarrow [T] \times \gY$ that satisfies:
\begin{align}
    h^* &= \arg\min_{h} \sum_{t=1}^{T} \E_{(\vx, y)\sim \gT_t} \left[ \mathbbm{1}_{h(\vx) \neq (t,y)} \right].
\end{align}
\end{definition}

\begin{remark}
    In TIL, it is common to have a shared input space $\gX = \gX_t, \forall t \in [T]$, but the space of the label distribution $\gY_t$ can be distinct~($\gY_i \cap \gY_j = \emptyset, \forall i\neq j$), partially shared~($\gY_i \cap \gY_j \neq \emptyset, \exists i\neq j$), or shared across different tasks~($\gY = \gY_t, \forall t \in [T]$). 
    In DIL, the tasks are defined in the same format, i.e., same input space $\gX$ and same output space $\gY$. During the inference, no task IDs are provided for the hypothesis, which means the continual learning model needs to capture the pattern between the domain-invariant features and the labels. DIL is commonly perceived as more difficult than TIL.
    CIL is commonly viewed as the most challenging continual learning scenario, as the model needs to infer the label and the task ID at the same time. Another possible formulation of CIL is to represent it as DIL but the output label spaces are disjoint, $\gY_i \cap \gY_j = \emptyset, \forall i\neq j$.
\end{remark}

% Categorization of the continual learning techniques
\subsubsection{Techniques of Continual Learning}

% As outlined in the three definitions provided earlier, t
The objective of CL is to find a hypothesis that minimizes risk across all tasks/domains. Consider DIL as an example~\cite{shi2024unified}, at $t$-th learning stage, the ideal training objective $\gL(h)$ is defined as
\begin{align}
    \gL(h) &\triangleq \underbrace{\sum_{i=1}^{t-1} \gL_{\gD_i}(h)}_{\text{past domains}} + \underbrace{\gL_{\gD_t}(h) \vphantom{\sum_{i=1}^{t-1} \gL_{\gD_i}(h)}}_{\text{current domain}}.
\end{align}
The objectives for past domains are often challenging to measure or optimize due to the memory constraints~(\Defref{def:memory}). Therefore, the core of designing CL algorithms lies in identifying a proxy learning objective for the first term without violating the memory constraint.
Existing CL techniques can be roughly categorized into 5 groups: (i)~replay-based, (ii)~regularization-based, (iii)~architecture-based, (iv)~optimization-based, and (v)~representation-based~\cite{wang2024comprehensive}.
Here, we provide a concise yet comprehensive introduction to the first three categories of continual learning techniques, as they find extensive application in continually learning large language models.



\textbf{Replay-Based Methods.}\quad 
Replay-based methods adopt the relaxed memory constraint by keeping a small buffer of observed data $\{M_i\}_{i=1}^{t-1}$ for each task $\gT_i$. Formally, they seek to optimize the following empirical training objective:
\begin{align}
    \hat{\gL}_{\text{replay}}(h) &\triangleq \underbrace{\sum_{i=1}^{t-1} \hat{\gL}_{M_i}(h)}_{\substack{\text{proxy for past domains}}} + \underbrace{\hat{\gL}_{S_t}(h) \vphantom{\sum_{i=1}^{t-1} \hat{\gL}_{M_i}(h)}}_{\text{current domain}},
\end{align}
where $\hat{\gL}_{S}$ denotes the empirical loss term evaluated on the set of examples $S$.
Often regarded as a simplistic solution to CL, replay-based methods may theoretically lead to loose generalization bounds~\cite{shi2024unified}. Despite this, they are valued for their simplicity, stability, and high performance, even with a small episodic memory~\cite{chaudhry2019tiny,riemer2018learning}. For instance, DER++~\cite{buzzega2020dark} demonstrates consistent performance enhancement by replaying a small set of past examples along with their logits~(known as dark experience replay). ESM-ER~\cite{sarfraz2023error} introduces error sensitivity modulation~(ESM) to mitigate abrupt representational drift caused by high-error new examples.
A significant focus in replay-based CL is enhancing sample efficiency for buffer maintenance. For instance, ~\cite{rebuffi2017icarl} prioritizes exemplar selection based on herding to accurately model class mean throughout class-incremental learning. 
\cite{zhao2022memory} propose storing low-fidelity examples to achieve memory-efficient exemplar set maintenance. 
% RM~(Rainbow Memory)~\cite{bang2021rainbow} introduces diversity-aware memory updates based on per-sample uncertainty estimation and data augmentation for class-incremental learning.




\textbf{Regularization-Based Methods.}\quad
Suppose $h_{\vtheta_{t-1}}$ is the hypothesis yielded after the $t-1$-th stage of training, parameterized by $\vtheta_{t-1}$. Regularization-based methods utilize a regularization term as a proxy for past domain losses, determined by the distance in the parameter space.
\begin{align}
    \hat{\gL}_{\text{reg}}(h_\vtheta) &\triangleq \underbrace{\lambda \cdot \left\| \vtheta - \vtheta_{t-1}\right\|_\mSigma}_{\substack{\text{proxy for past domains}}} + \underbrace{\hat{\gL}_{S_t}(h_\vtheta) \vphantom{}}_{\text{current domain}},
\end{align}
where $\|\vv\|_\mSigma = \vv^\top \mSigma \vv$ is the vector norm evaluated on a positive-semi-definite matrix $\mSigma$, and $\lambda$ is the regularization coefficient, a hyper-parameter introduced to balance the past knowledge retention and current knowledge learning. 
The matrix $\mSigma$ introduced is to measure the different level of importance of each parameters and their correlations in retaining the past knowledge. 
In practice, to reduce computational overhead, diagonal matrices are often designed to encode only the importance of each parameter.
For example, Elastic Weight Consolidation (EWC)~\cite{kirkpatrick2017overcoming} adopts a Bayesian perspective, using diagonal values from the Fisher Information Matrix (FIM) as an approximation for the Hessian matrix of parameters. This forms a sequential Maximize A Posteriori (MAP) optimization for continual learning. Memory Aware Synapses (MAS)~\cite{aljundi2018memory} computes parameter importance in an online and unsupervised manner, defining importance by accumulated absolute gradient during training.
It is also worth noting that when $\mSigma=\mI$ degenerates to an identity matrix, the regularization term simplifies to a basic $l_2$-penalty term, equally penalizing each parameter, which can be surprisingly effective in some cases of continual LLMs~\cite{rongali2021continual}.
% In Table~\ref{tab:cpt-small},~\ref{tab:dap}, and \ref{tab:cft}, we use \emph{``param. reg.''} to denote this category of continual learning methods.


\textbf{Architecture-Based Methods.}\quad 
Expanding the network architecture dynamically to assimilate new knowledge is deemed the most efficient form of CL~\cite{wang2022learning,wang2022dualprompt}. This method primarily tackles adaptation challenges and can achieve zero-forgetting when task IDs are available during inference or can be correctly inferred~\cite{gururangan2022demix,wistuba2023}. 
However, due to the difficulty of task ID inference, architecture expansion is predominantly utilized in TIL but is scarcely explored in DIL or CIL.
Progressive Neural Networks~(PNN)~\cite{rusu2016progressive} proposes learning laterally connected neurons as new tasks arise, ensuring non-forgetting and enabling transfer of previously learned neurons for future tasks. In conjunction with pre-trained backbone large models like ViT~\cite{dosovitskiy2020image}, CoLoR~\cite{wistuba2023} trains various low-rank adaptation (LoRA)~\cite{hu2021lora} modules for different tasks. It estimates and stores prototypes for each task and utilizes the natural clustering ability of the pre-trained model during testing to infer task IDs, selecting the corresponding LoRA component for prediction generation.
In the domain of continual LLMs, architecture expansion has resurged in popularity following the rise of parameter-efficient fine-tuning (PEFT)~\cite{shazeer2017outrageously,hu2021lora,dettmers2023qlora}, a topic we will delve into shortly~\cite{yang2024moral,wang2023orthogonal,li2024examining,jang2022towards,jin2022lifelong,paul2024ircoder,yan2023af,wu2024llama}.

% Addressing general knowledge forgetting in Large Language Models (LLMs) is a significant challenge. As these models are incrementally updated or fine-tuned on new datasets, they often suffer from catastrophic forgetting - the loss of previously learned information. This phenomenon can severely impact the model's performance across previously well-performing domains. Techniques such as Elastic Weight Consolidation (EWC) and Progressive Neural Networks have been proposed to mitigate this issue by preserving important parameters related to old tasks while learning new ones. Moreover, employing rehearsal techniques, where a model is intermittently re-exposed to older data, can also help maintain general knowledge over time. These strategies are crucial for maintaining a broad and consistent knowledge base in LLMs, ensuring their utility across a wide range of applications without significant regression in performance on previously learned tasks \cite{kirkpatrick2017overcoming, rusu2016progressive, robins1995catastrophic}. \haizhou{This background intro might be too general, can be omitted.}


\section{Evaluation Protocols and Datasets}
\label{app:eval-and-data}

In \appref{app:eval}, we review common continual learning evaluation metrics and provide formal definitions.
In \appref{app:eval-llm}, we introduce metrics designed specifically for continual LLMs. 
Finally, in \appref{app:data}, we outline the datasets available for each discussed topic.


\subsection{Evaluation Metrics of Continual Learning}
\label{app:eval}
In the realm of conventional continual learning, where task streams take the form of classification, many metrics rely on the concept of Accuracy Matrix~\cite{lopez2017gradient,shi2024unified}.
Extending this notion to the context of continually learning LLMs, we introduce the \textbf{Performance Matrix} $\mP\in\mathbb{R}^{T\times T}$, where $T$ represents the total number of training stages.
Each entry of $\mP$ corresponds to a performance metric evaluated on the models, such as perplexity on pre-training data~\cite{jin2022lifelong,chen2023lifelong,gupta2023continual}, zero-shot/few-shot evaluation metrics on downstream data without fine-tuning~\cite{colombo2024saullm7b,wu2023pmc,Azerbayev2023LLEMMA,deng2023learning,nijkamp2022codegen,rozière2024code}, fine-tuned accuracies on downstream tasks~\cite{amba2021dynamic,qin2023recyclable,chen2023lifelong,jang2022towards}, and probing accuracies from fine-tuning add-on components evaluated on downstream tasks~\cite{tao2022can,luo2023investigating,zheng2023learn}.
In $\mP$, $P_{i,j}$ denotes the model's performance after training on task $i$ and evaluating on task $j$. With this Performance Matrix definition, we introduce the primary evaluation protocols widely adopted.

\textbf{Overall Performance~(OP)}.\quad 
The Overall Performance~(OP)~\cite{ke2021achieve,zhang2022continual,zhang2023copf} is a natural extension of the concept of Average Accuracy~\cite{lopez2017gradient,shi2024unified}. The OP measured up until training stage $t$ is the average performance of the model trained right after the stage $t$. Denote it as $\operatorname{OP}_t$ and we have: 
\begin{align}
    \operatorname{OP}_t &\triangleq \frac{1}{t} \sum_{i=1}^{t} P_{t,i}.
\end{align}
As noted in \cite{shi2024unified}, the OP corresponds to the primary optimization objective defined in \Defref{def:til}, \ref{def:dil}, and \ref{def:cil}. In much of the continual learning literature, once all $T$ tasks are completed, the final $\operatorname{OP}$~($\operatorname{OP}_T$) is reported, with the subscript $_T$ often omitted for brevity. 
In some works, OP is weighted by the importance of tasks $\tilde{\operatorname{OP}} \triangleq \frac{1}{T}\sum_{i=1}^T w_i P_{t,i}$, where $w_i = N_i / \sum_{j=1}^T N_j$ represents the ratio of data. In some literature, $\tilde{\operatorname{OP}}$ is referred to as ``example accuracy''~\cite{chen2024parameterizing}, ``whole accuracy''~\cite{song2023conpet}, or ``edit success rate'' in  CMR~\cite{hartvigsen2023aging}. 

\textbf{Forgetting (F).}\quad
Define $F_t$ as the forgetting up to task $t$, which represents the largest performance drop observed throughout the training process, averaged over $t$ training stages:
\begin{align}
    F_t &\triangleq \frac{1}{t-1}\sum_{j=1}^{t-1} \left[ \max_{l\in [t-1]} \{P_{l,j} - P_{t,j}\} \right].
\end{align}
Typically, researchers report the average forgetting $F=F_T$ at the end of the entire training process.
Forgetting quantifies the impact of learning new tasks on previously acquired knowledge. Ideally, a robust continual learning framework should achieve \textbf{Backward Transfer~(BWT)}, where learning new tasks enhances performance on prior tasks. This enhancement is typically measured by negating the forgetting, thus indicating an improvement in performance on earlier tasks. The concepts of Forgetting and Backward Transfer underpin various evaluation metrics, such as knowledge retention~\cite{jin2022lifelong}, performance on unchanged knowledge~\cite{jang2022temporalwiki}, average increased perplexity~(AP$^+$)~\cite{qin2022elle}, and test and edit retention rate in CMR~\cite{hartvigsen2023aging}.


\textbf{Forward Transfer (FWT).}\quad 
Forward Transfer measures the generalization ability of the continual learning algorithms. Formally, forward transfer $\operatorname{FWT}_t$ up to training stage $t$ is defined as
\begin{align}
    \operatorname{FWT}_t &\triangleq \frac{1}{t-1} \sum_{i=2}^{t}P_{i-1, i} - b_i,
\end{align}
where $b_i$ is the baseline performance of the model evaluated on task~$i$ before undergoing continual learning. Strictly speaking, the definition of $b_i$ is not the same as defined in the previous work~\cite{lopez2017gradient,shi2024unified}, where it is used to denote the performance of a random initialization of the model.
Additionally, we extend the notation of forward transfer in the vertical direction to represent the performance improvement on downstream tasks resulting from domain-adaptive pre-training~(see \Tabref{tab:dap}). 
Forward Transfer is alternatively referred to as temporal generalization~\cite{jin2022lifelong} or knowledge transfer~\cite{lazaridou2021mind} in some literature.
In this section, we introduce the evaluation protocols and datasets for continul LLMs. 


\subsection{Continual LLMs' Evaluation Protocols}
\label{app:eval-llm}
\textbf{LAnguage Model Analysis~(LAMA).}\quad 
LAnguage Model Analysis (LAMA) is an evaluation framework designed to \emph{probe the world knowledge} embedded in language models~\cite{petroni2019language}.
It converts each world fact into a cloze statement, which is then inputted into the language models to predict the correct answer.
LAMA has been extended for continual pre-training, particularly for those under the temporal shifts~\cite{jang2022temporalwiki,jang2022towards}. In CKL, three LAMA benchmarks are constructed for different dimensions: InvariantLAMA assesses knowledge retention on time-invariant facts, UpdatedLAMA focuses on knowledge update, and NewLAMA evaluates knowledge acquisition~\cite{jang2022towards}. 

\textbf{Forgotten / (Updated + Acquired) Ratio~(FUAR).}\quad
As the performance of a pre-trained LLM is decomposed into a fine-grained set in CKL~\cite{jang2022towards}, OP becomes a too general metric and cannot accurately reflect the balance and trade-offs of the model's behavior. 
To address this issue, CKL proposes a joint evaluation metric FUAR~(Forgotten / (Updated + Acquired) Ratio) for continual pre-training. 
A FUAR value of 1 represents an equal trade-off between the knowledge forgetting and knowledge learning: for each piece of updated or acquired knowledge, one piece of time-invariant knowledge is forgotten on average. A FUAR less than 1 suggests high learning efficacy, where more than one piece of knowledge is acquired at the expense of forgetting one piece of time-invariant knowledge.


\textbf{X-Delta.}\quad
In TRACE~\cite{wang2023trace}, the authors propose a set of ``X-Delta'' metrics for continual instruction tuning, quantifying the forward transfer on specific abilities of LLMs. Let's denote a set of $M$ datasets $\{X_1, X_2, \cdots, X_M\}$ for task X. The baseline performances of the pre-trained LLM evaluated on these tasks are denoted as $\{b_1^X, \cdots, b_M^X\}$. The model undergoes continuous fine-tuning on a different set of tasks, distinct from those used for evaluation. Throughout the sequential training process, the performance of the model after learning task $t$ on evaluation tasks $X_i$ is $R_{t,i}^X$. The X-Delta $\Delta R_{t}^X$ after learning task $t$ is defined as:
\begin{align}
    \Delta R_t^X &\triangleq \frac{1}{M}\sum_{m=1}^M (R_{t,i}^X - b_i^X).
\end{align}
In the public TRACE benchmark, the authors construct three sets of evaluation tasks to benchmark the ability of LLMs, including \emph{general ability}, \emph{instruction following}, and \emph{safety}~\cite{wang2023trace}.


% \todo{More evaluation metrics in CMR and CMA (CMA done).}

\textbf{NLG Score.}\quad In continual model alignment, three prominent metrics used to evaluate different aspects of Natural language generation (NLG) are BLEU-4~\cite{papineni2002bleu}, METEOR~\cite{banerjee2005meteor}, and ROUGE-L~\cite{lin2004rouge}. BLEU-4~\cite{papineni2002bleu}, designed for machine translation (MT), evaluates the precision of n-grams between the machine-generated and reference texts, focusing especially on four-word sequences to gauge fluency and adequacy. METEOR~\cite{banerjee2005meteor} also targets MT but aims to improve correlation with human judgment by considering synonyms and stemming, thus providing a more nuanced assessment of translation quality. On the other hand, ROUGE-L~\cite{lin2004rouge} is commonly applied in summarization tasks, assessing the longest common subsequence between the generated summary and a set of reference summaries, effectively measuring the recall of essential content. Each metric has its strengths and is tailored to specific kinds of language processing tasks, reflecting different dimensions of text generation quality. 

% \textbf{rPMS.}\quad The reference PM score (rPMS) quantifies the alignment of a model's outputs with human preferences as captured by a reference Preference Model (PM). Specifically, for a given task $t$, the rPMS is defined as:
% \begin{align}
% \text{rPMS}_t = \text{rPM}(M_t, D_{\text{test}, t}),
% \end{align}
% where $M_t$ is the model being evaluated, $D_{\text{test}, t}$represents the test dataset for task $t$, and rPMS is the function implemented by the reference PM that scores model outputs based on their alignment with human preferences. Higher rPMS values indicate that the model $M_t$ aligns closely with human preferences, suggesting effective learning and retention of the desired behaviors across tasks in a continual learning scenario. The metric is crucial for evaluating the extent to which models can maintain or improve performance relative to human preference standards without substantial forgetting.



\subsection{Datasets}
\label{app:data}
In this section, we provide a comprehensive review of the datasets available for benchmarking continual LLMs, as illustrated in \Tabref{tab:datasets}. We intentionally exclude datasets used for domain-adaptive pre-training LLMs in vertical domains such as legal, medical, and financial, unless they are specifically designed for continual domain-adaptive pre-training. Furthermore, we omit datasets used in general continual fine-tuning, as they have already been extensively studied in existing works~\cite{biesialska2020continual,ke2023continual}.


\textbf{Datasets for Continual Pre-Training~(CPT) and Domain Adaptive Pre-Training~(DAP).}\quad
% Temporal-level
Current research lacks a widely recognized benchmark for evaluating continual pre-training LLMs under temporal shifts.
TimeLMs utilizes a series of Twitter corpora collected until 2022, sequentially pre-training RoBERTa models quarterly~\cite{loureiro2022timelms}.
CC-RecentNews, adopted as unlabeled pre-training data for LMs in CKL~\cite{jang2022towards}, consists of recent news and serves as a single-stage dataset. 
Additionally, CKL introduces InvariantLAMA, NewLAMA, and UpdatedLAMA to assess the principles of continual knowledge learning.
TWiki, a dataset derived from the articles of Wikipedia between August and December 2021, is curated and cleaned in TemporalWiki~\cite{jang2022temporalwiki}. This dataset facilitates the exploration of incremental learning by providing the Diffsets between neighboring snapshots.
% Content-level
For works that study the content-level distributional shifts in CPT and DAP, researchers often resort to a similar set of publicly available datasets~\cite{lo2020s2orc,xu2019bert,ni2019justifying} to construct their own test beds for continual learning algorithms. 
The $^*$DAPT dataset, developed by \cite{gururangan2020dont}, comprises four domains: BioMed and Computer Science from S2ORC \cite{lo2020s2orc}, News from \cite{zellers2019defending}, and Reviews from \cite{he2016ups}. In $^*$DAPT's original study, each domain undergoes individual domain adaptive pre-training stages to demonstrate the universality of DAP's effectiveness. Subsequent works, such as ELLE~\cite{qin2022elle} and Recyclable Tuning~\cite{qin2023recyclable}, follow suit by employing these domains for multi-stage CPT.
DEMix~\cite{gururangan2022demix} presents another large-scale dataset, featuring eight semantic domains with over 73.8 billion tokens. Alongside the training set, it includes eight additional datasets for validating the generalization ability of LLMs. 
On a smaller scale, $^*$CPT~\cite{ke2022continual-train} and $^*$DAS~\cite{ke2022continual-pre} datasets consist of four and eight domains, respectively, with approximately 3.12 million examples and a size of 4.16GB each. These datasets are constructed similarly to the aforementioned ones.



\textbf{Datasets for Continual Instruction Tuning.}\quad
Measuring the effectiveness of CIT is crucial, particularly because traditional evaluation metrics may not be suitable for LLMs: many of them are overly simplistic and fail to comprehensively assess the model's ability to learn continually. New benchmarks and metrics are required to evaluate both the retention of old knowledge and the integration of new instructions. TRACE~\cite{wang2023trace} stands as a continual learning benchmark designed specifically for LLMs, encompassing diverse tasks such as multilingual capabilities, code generation, and mathematical reasoning. CITB~\cite{zhang2023citb} represents another benchmark for CIT, incorporating both learning and evaluation protocols. It in addition demonstrates that replay generally yields the best performance across all methods. CoIN~\cite{chen2024coin} extends the benchmark to MLLMs, incorporating a balanced and diverse set of instructions from vision-language datasets. 

% Measuring the effectiveness of CIT is crucial, given that traditional evaluation metrics may not suitable for LLMs. Many of them are too simple and fail to comprehensively assess the model's ability to learn continually. New benchmarks and metrics are needed to evaluate both the retention of old knowledge and the integration of new instructions. Trace\cite{wang2023trace} is a continual learning benchmark designed for LLMs, including diverse tasks like multilingual capabilities, code generation, and mathematical reasoning. CITB\cite{zhang2023citb} is another benchmark for CIT, consists both learning and evaluation protocols, it demonstrates that Replay generally have the best performance for all methods. COIN\cite{chen2024coin} extend the benchmark to Multimodal Large Language Models (MLLMs), includes a balanced and diverse set of instructions from vision-language datasets. It reveals that MLLMs still suffer from catastrophic forgetting.


\textbf{Datasets for Continual Model Refinement.}\quad
% The effect of model refinement on LLM is usually evaluated by question answering. 
Most datasets for continual model refinement can be categorized into two types~\cite{mazzia2023survey}: fact checking and question answering. For fact checking, models are asked to verify the truthfulness of certain claims, typically modeled as a classification task. Key datasets include FEVER~\cite{fever} (used by~\cite{de2021editing, hase2021language}) and VitaminC~\cite{vitaminC} (used by~\cite{mitchell2022memory}), both sourced from Wikipedia. For question answering, models are tasked with providing specific answers instead of choices. Zero-shot Relation Extraction (zsRE)~\cite{zsRE} is the most widely employed dataset for this purpose ~\cite{hase2021language, meng2022locating, meng2022mass, hase2023does, hartvigsen2023aging, das2024larimar}, alongside Natural Questions (NQ) ~\cite{nq} and T-rex ~\cite{T-rex}. \cite{meng2022locating} adapted zsRE with additional counterfactuals to create the more challenging CounterFact dataset, used by ~\cite{yu2023melo, hu2024wilke, das2024larimar}. Beyond these two categories, SCOTUS~\cite{scotus} is also utilized~\cite{hartvigsen2023aging} in the assessment of continual model refinement through a document classification task for U.S. Supreme Court cases into 11 topics.



% Most datasets for continual model refinement can be categorized into 2 types: fact checking and question answering. For fact checking, models are asked to judge whether certain claims are true. This is usually modeled as a classification task. Typical datasests include FEVER ~\cite{fever} (used by ~\cite{de2021editing, hase2021language}) and VitaminC ~\cite{vitaminC} (used by ~\cite{mitchell2022memory}), which are both extracted from Wikipedia. For question answering, models are asked to provide a specific answer instead of a choice. Zero-shot Releation Extraction (zsRE) ~\cite{zsRE} is the most widely used question answering dataset ~\cite{hase2021language, meng2022locating, meng2022mass, hase2023does, hartvigsen2023aging, das2024larimar}, while Natural Questions(NQ) ~\cite{nq} and T-rex ~\cite{T-rex} are also been used. \cite{meng2022locating} adapts the zsRE with additional counter facts and create a more challenging dataset CounterFact, used by \cite{yu2023melo, hu2024wilke, das2024larimar}. Besides the previous 2 categorys, SCOTUS ~\cite{scotus} is also used ~\cite{hartvigsen2023aging} in the assessment of Continual Model refinement. This is a document classification task for U.S. Supreme Court into 11 topics. 

% Compared with previous datasets, 



% PARAREL dataset. SCOTUS. NQ. zsRE. OpenWebText

% meng et al use Counter fact derived from pararel datasets, others incldues ... (from ageing with grace) 



\textbf{Datasets for Continual Model Alignment.}\quad
In the domain of reinforcement learning with human feedback (RLHF), several datasets are commonly employed across different studies to evaluate the adaptation and effectiveness of models under varying scenarios and continuous learning conditions. The IMDB~\cite{maas2011learning} and HH-RLHF~\cite{bai2022training} dataset, as introduced in~\cite{zhang2023copf} within their study on continual learning through optimal policy fitting, leverages data gathered from interactive RL scenarios to model human preferences dynamically. Similarly, the Reddit TL;DR dataset~\cite{volske2017tl} used by~\cite{zhangcppo,zhang2023copf} is focused on text summarization, providing a robust platform for testing the longevity and adaptability of learning algorithms under evolving conditions. Lastly, Common Sense QA~\cite{clark2018think, lai2017race, bisk2020piqa}, Reading Comprehension~\cite{rajpurkar2018know,dua2019drop}, and Translation~\cite{bojar2014findings}, which are utilized in~\cite{lin2024mitigating} are selected to assess the challenges of aligning RL agents with human expectations without incurring significant performance penalties. Each of these datasets is pivotal in advancing the understanding of continual learning and the interplay between human feedback and machine learning adaptation.


\textbf{Datasets for Continual Multimodal Large Language Models.}\quad
Following LLaVA~\cite{liu2023visual}, many MLLMs adopt the pattern of instruction tuning to enable assessing alignment with human intention and knowledge preservation for reasoning. Thus, traditional tasks like image classification can be transformed to VQA tasks to evaluate the ability of MLLMs, which are otherwise challenging to assess using conventional methods. Several benchmarks have been proposed to evaluate the CL method for MLLMs. MCIT~\cite{he2023continual} proposes the first continual instruction tuning benchmarks, Benchmark1 and Benchmark2. The difference between benchmark1 and benchmark2 is that benchmark2 includes Multi-task Joint Instruction Tuning, which aims to explore whether multi-task joint instruction tuning improves the model’s continual learning ability.
\cite{zhai2023investigating} proposes EMT, the first classification evaluation framework to investigate catastrophic forgetting in MLLMs. \cite{chen2024coin} presents a comprehensive benchmark CoIN, spanning 8 task categories and evaluating MLLMs from two perspectives:  Instruction Following and General Knowledge, which assess the alignment with human intention and knowledge preserved for reasoning, respectively. 
\cite{zhao2024reconstruct} constructs two datasets, UPMC-Food101-CMML and MM-IMDb-CMML to benchmark the novel CMML task, which means the data of certain modalities is missing during continual fine-tuning. UPMC-Food101-CMM contains 101 food categories and 61,142 training, 6,846 validation, and 22,716 test image-text pairs. MM-IMDb-CMML is a multi-label classification dataset across 27 distinct movie genres, consisting of 15,552 training, 2,608 validation and 7,799 test image-text pairs.

\end{document}
\endinput
%%
%% End of file `sample-manuscript.tex'.